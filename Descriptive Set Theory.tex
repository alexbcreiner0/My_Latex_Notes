\documentclass{article}
\usepackage[utf8]{inputenc}
\usepackage[margin=1in]{geometry}
\usepackage{amsmath}
\usepackage{amsthm}
% Package for making turing machine diagrams %
\usepackage{tikz}
\usetikzlibrary{chains,fit,shapes}
% Packages for algorithms %
\usepackage{algorithm}
\usepackage{algorithmic}
% Package which has the nice looking empty set symbol (\varnothing)
\usepackage{amssymb}
\usepackage{bm}

\title{Descriptive Set Theory Notes}
\author{Alex Creiner}

\theoremstyle{definition}
\newtheorem{definition}{Definition}[section]
\newtheorem{problem}{Problem}[section]
\newtheorem{lemma}{Lemma}[section]
\newtheorem{theorem}{Theorem}[section]
\newtheorem{fact}{Fact}[section]
\newtheorem{corollary}{Corollary}[section]

\theoremstyle{plain}
\newtheorem{example}{Example}

\begin{document}
\maketitle
\section{Polish Spaces}
A \textbf{Polish space} is a complete, separable metric space. A Polish space with no isolated points is called \textbf{perfect}. \textbf{Baire space} is the perfect Polish space 
\begin{align}
    \mathcal{N} = \omega^{\omega} = \prod_{n \in \omega}\omega := {}^{\omega}\omega
\end{align}
Where $\omega$ refers to the first infinite ordinal space (i.e. just $\mathbb{N}$ with the discrete topology.
This is the set of all infinite sequences of integers. Since this is the product topology, basic open sets are of the form 
\begin{align}
    \mathcal{B} = \{\prod_{n \in \omega} U_n: U_n \subseteq \omega \textrm{ is open} \}
\end{align}
Where each $U_i$ is open but all but finitely many of the $U_i$ are the entire factor space $\omega$. Since the singleton sets make up a basis for each factor space we can shrink our basis for Baire space to be sets of sequences such that finitely many terms are fixed to be a specific value, and since product spaces are homeomorphic to permutations of it's factor spaces, we can in fact take the basis to be sets of sequences where \textit{the first $n$ terms are fixed}. I.e. we take basic neighborhoods to be of the form.
\begin{align}
    N(k_0,...,k_n) := \{\alpha \in \mathcal{N}: \alpha(0) = k_0,...,\alpha(n) = k_n \}
\end{align}
To confirm that Baire space is indeed a perfect Polish space, we need to show that it is complete under some metric, and that it has a countable dense set. (We already know that it is metric because a countable product of metric spaces is metric, but the completeness remains to be seen. $\mathcal{N}$ has no isolated points, because clearly any basic open set containing $x$ also contains infinitely many other points.)
\begin{fact}
    $\mathcal{N}$ is generated by the metric
    \begin{align}
        d(\alpha,\beta) = \begin{cases}
                            0 & \textrm{ if } \alpha = \beta \\
                            \frac{1}{2^{\min\{n: \alpha(n) \neq \beta(n)\}+1}} & \textrm{ if } \alpha \neq \beta \\
                          \end{cases}
    \end{align}
    and is complete under this metric. Furthermore, the collection of eventually constant sequences is a countable dense set in $\mathcal{N}$
\end{fact}
\begin{corollary}
$\mathcal{N}$ is a perfect Polish space.
\end{corollary}
\begin{proof}
TBD 
\end{proof}
\begin{theorem}
    Baire space is homeomorphic to the space of irrationals $Irr$ as a subspace of the real numbers with the standard topology $\mathbb{R}_{std}$.
\end{theorem}
\begin{proof}
TBD
\end{proof}
We thus refer to members of Baire space as irrationals. Another important fact about Baire space:
\begin{theorem}
    For every Polish space $\mathbb{M}$, there exists a continuous surjection 
    \[\pi: \mathcal{N}: \to \mathcal{N} \]
\end{theorem}
\begin{proof}
TBD
\end{proof}
Recall that a metric space is compact iff it is complete with respect to every compatable metric. (Show) Recall also that any compact metric space is separable. Gunna prove this as an exercise:
\begin{proof}
    Let $(X,d)$ be a compact metric space. Fix an integer $m$, and note that 
    \begin{align}
        \{B(x,\frac{1}{m}): x \in X, n \in \omega \}
    \end{align}
    is an open cover of $X$, so it has a finite subcover, i.e. there exists a finite set of points $\mathcal{D}_m := \{x_{1m},x_{2m},...,x_{nm}\}$ such that
    \begin{align}
        \bigcup_{i=1}^n B(x_i,\frac{1}{m}) = X
    \end{align}
    Thus, consider the set 
    \begin{align}
        \mathcal{D} = \bigcup_{m=1}^\infty \mathcal{D}_m
    \end{align}
    Then this is clearly a countable set. Then for any basic open set $B(x,\epsilon)$, $\exists m \in \omega$ such that $\frac{1}{m} < \epsilon$. Then since $\mathcal{D}_m$ covers X, there exists a $y \in \mathcal{D}_m$ such that $x \in B(y,\frac{1}{m})$. But then $|x-y| < \frac{1}{m} < \epsilon$, so $y \in B(x,\epsilon)$, i.e. $B(x,\epsilon) \cap \mathcal{D} \neq \varnothing$. Thus $\mathcal{D}$ is dense in $X$.
\end{proof}
Thus, any compact metric space is Polish. Next consider the \textbf{Cantor space} 
\begin{align}
    \mathcal{C} = 2^{\omega} = \{0,1\}^{\omega} = \prod{n \in \mathbb{N}}\{0,1\}
\end{align}
\begin{definition}
	A \textbf{tree} on a set $X$ is a subset of $X^{<\omega}$ which is closed under initial segments. That is, if $s \in T$ and $m < lh(s)$, then $s \restriction m \in T$. A \textbf{branch} or \textbf{path} through $T$ is an $f:\omega \to T$ such that for all $n$, $f \restriction n \in T$. We let $[T]$ denote the set of branches through $T$. 
\end{definition}
\section{The Borel Hierarchy}
\begin{definition}
A \textbf{pointclass} is a collection of sets $\Lambda$ such that each $P \in \Lambda$ is a subset of \textit{some} product space $\mathcal{X}$.  
\end{definition}
We will think of pointsets sometimes as sets and sometimes as relations, so that to write $P(x)$ is to imply $x \in P$. We refer to negation and complementation as equivalent and write $\neg A$ to refer to the complement of $A$, an element of a pointclass. I.e. if $P \subseteq \mathcal{X}$ is any pointset (that is, element of a pointclass), define 
\[ \neg P := \mathcal{X}-P \]
and 
\[ \neg \Lambda := \{ \neg P: P \in \Lambda \} \]
We call this the \textbf{dual pointclass}. Similarly. if $P \in \mathcal{X} \times \omega$ for some product space $\mathcal{X}$, we let
\[ \exists^{\omega}P := \{x \in \mathcal{X}: \exists n P(x,n)\}\]
We refer to this operation as \textbf{projection along $\omega$}, or \textbf{existential number quantification}. Similarly to $\neg$, we extend this to an operation on entire pointclasses as well, as the set of projections of sets in the class:
\[\exists^{\omega}\Lambda := \{\exists^{\omega} P: P \in \Lambda, P \subseteq \mathcal{X} \times \omega \textrm{ for some } \mathcal{X}\} \]
Note that projection along $\omega$ is only defined for products of a product space $\mathcal{X}$ with $\omega$.
\begin{definition}
    We define the \textbf{Borel pointclasses of finite order} for ordinals as follows:
       \[ \bm{\Sigma}_1^0 = \textrm{all open pointsets} = \{P: P \subseteq \mathcal{X} \textrm{ for some product space } \mathcal{X} \textrm{ with $P$ open} \} \]
       \[  \bm{\Sigma}_{n+1}^0 = \exists^{\omega}\neg \bm{\Sigma}_n^0 \]
       The \textbf{dual Borel pointclasses} are defined by 
       \[ \bm{\Pi}_n^0 = \neg \bm{\Sigma}_n^0 \]
       and the \textbf{ambiguous Borel pointclasses} are defined by
       \[ \bm{\Delta}_n^0 = \bm{\Sigma}_n^0 \cap \bm{\Pi}_n^0 \]
\end{definition}
So $\bm{\Pi}_1^0$ is all of the closed pointsets, and $\bm{\Delta}_1^0$ is all of the clopen sets. Consider $\bm{\Sigma}_2^0 = \exists^{\omega} \bm{\Pi}_1^0$. $P$ is in this set if there is a closed subset $F \subseteq \mathcal{X} \times \omega$ such that for all $x$, $P(x) \iff \exists t F(x,t)$. 
\begin{theorem}
	For any space $X$, the Borel sets are the smallest $\sigma$-Algebra in $\mathcal{P}(X)$ containing the open sets (I.e. they are the smallest subfamily which is closed under complements and countable unions/intersections.)
\end{theorem}
\begin{lemma}[Moschovakis Exercise 1B.3] \leavevmode
\begin{enumerate}
    \item A pointset P is $\bm{\Sigma}_2^0$ iff 
    \[ P = \bigcup_{i=0}^{\infty} F_i \]
    Where each $F_i$ is closed.
    \item Similarly, P is $\bf{\Pi}_2^0$ iff
    \[ P = \bigcup_{i=1}^{\infty} G_i \]
    Where each $G_i$ is open.
\end{enumerate}
\end{lemma}
\begin{proof}
    For the first claim, let $P = \bigcup_{i \in \omega} F_i$, with each $F_i$ closed. Define the set $F \subseteq \mathcal{X} \times \omega$ by $(x,i) \in F \iff x \in F_i$. We need to show that $F$ is closed. To see this, let $x_n = (a_n,b_n)$ be a sequence in $F$ converging to a point $(a,b)$. Then the sequences $a_n$ and $b_n$ converge in their respective coordinate spaces. Since $\omega$ has the discrete topology, and convergent sequence is eventually constant. Thus there exists an index $m$ such that for all $n \geq m$, $b_n = b$, and thus $x_n = (a_n,b)$. But $(a_n,b) \in F$ means that $a_n \in F_b$ for all $n \geq m$, and since $F_b$ itself is closed in $\chi$, it must be that $a \in F_b$. Thus $(a,b) \in F$, meaning that convergent sequences in $F$ must converge to points which are themselves in $F$, i.e. $F$ is closed. 
    \par Conversely, let $P \in \bm{\Sigma}_2^0$. Then there exists a closed $F \subseteq \mathcal{X} \times \omega$ such that for all $x$, $P(x) \iff \exists t F(x,t)$. For each $i \in \omega$, define the set $F_i := \{ x: (x,i) \in F \}$. Again it is clear that this collection defines $P$ in the desired way, i.e. it is plain to see that $P = \bigcup_{i\in \omega} F_i$. It remains to show that $F_i$ is closed for each $i$. As before, let $x_n \subseteq F_i$ be a sequence converging to a point $x \in \chi$. Note that $x_n \in F_i$ for each $n$ means that $(x_n,i) \in F$ for each $n$, i.e. $(x_n,i)$ is itself a sequence in $F$ converging to $(x,i)$. Since $F$ is closed, $(x,i)$ must be in $F$, meaning that $x \in F_i$, as desired. So each $F_i$ is closed, and we have our equivalence for $\bm{\Sigma}_2^0$.
    \par The $\bm{\Pi}_2^0$ equivalence follows immediately now: $P \in \bm{\Pi}_2^0 \iff P^c \in \bm{\Sigma}_2^0$ iff there exists a collection of closed $F_i$ such that $P^c = \bigcup_{i=0}^{\infty} F_i \iff P = \bigcap_{i=0}^{\infty}F_i^c$. Defining $G_i = F_i^c$, we have our equivalence. 
\end{proof}
Thus for a fixed space $X$, the $\bm{\Sigma}^0_2$ sets are precisely the $F_{\sigma}$ sets, and $\bm{\Pi}^0_2$ sets are precisely the $G_{\delta}$ sets. A simple induction extends this to show that the $\bm{\Sigma}^0_n$ sets are really just the $F_{\sigma \delta}$ sets, and so forth (I'm not sure if the order is right but you get it.)
\begin{lemma}
	For any $n \in \omega$, a set $P \in \bm{\Sigma}^0_{n+1}$ iff $P = \bigcup_{i \in \omega}F_i$ where $F_i \in \bm{\Pi}^0_n$. Similarly, $P \in \bm{\Pi}^0_{n+1}$ iff $P = \bigcap_{i \in \omega}G_i$ with $G_i \in \bm{\Sigma}^0_n$. 
\end{lemma}
\begin{proof}
	The base case is done already by the Moschovakis exercise. The inductive case is nearly trivial. Note that $P \subset X \in \bm{\Sigma}^0_{n+2}$ iff $\exists F \subset X \times \omega$ such that $P(x) \iff \exists t F(x,t)$. As before, define $F_i \{x: (x,n) \in F\}$, so that $P = \bigcup_{i \in \omega} F_i$. 
\end{proof}
\begin{example}
    The set \[ I_1 = \{x \in \mathcal{C}: \textrm{$x$ has infinitely many ones}\} \] is in $\bm{\Pi}_2^0$
\end{example}
\begin{proof}
    Note that \[x \in I_1 \iff \forall x \in \omega \exists k \in \omega (x_{n+k} = 1)\]. Thus if, noting that $\pi_m^{-1}(\{1\})$ is clearly a basic open set in $\mathcal{C}$, we have that for each $n$, $\bigcup_{k \in \omega} \pi_{n+k}(\{1\})$ is open, and thus \[ I_1 = \bigcap_{n \in \omega} \bigcup_{k \in \omega} \pi_{n+k}(\{1\})\] is clearly a $\bm{\Pi}_2^0$ set, by the above lemma. Note how this lemma facilitated a 'conversion process' of going from $\forall$ quantifiers to $\bigcap$ intersections, and from $\exists$ quantifiers to $\bigcup$ unions.
\end{proof}

We make some observations, generalizations, and notation conventions before going forward. Typically Greek variables $\alpha, \beta, \gamma,...$ will denote sequences of integers, that is to say, elements of Baire space $\mathcal{N}$, while integers themselves will get typical English letters $i,j,k,n,...$. We can view logical symbols as denoting operations on pointsets. In general, a \textbf{$k$-ary pointset operation} is a function $\Phi$ whose domain is some set of $k$-tuples of pointsets (that is, it acts on elements of a pointclass, i.e. sets) and whose codomain is other pointsets. 
\par For example, \textbf{conjunction} $\wedge$ can be viewed a binary $k$-ary pointset operation. For a pair of pointsets $P,Q$ in some shared pointclass $\mathcal{X}$, $P \wedge Q$ is a new pointset, defined in terms of the actual logical conjunction as follows:
\begin{align}
	(P \wedge Q)(x) \iff P(x) \wedge Q(x) 
\end{align}
Obviously, this is also just the intersection $P \cap Q$. \textbf{Disjunction} is similar. \textbf{Negation} is most conveniently regarded as a \textit{collection} of operations $\neg_{\mathcal{X}}$, one for each product space $\mathcal{X}$, with 
\[ \neg_{\mathcal{X}}P = \mathcal{X} - P = \{ x \in \mathcal{X}: \neg P(x) \} \]
In practice, we will simply write $\neg P$ and it will be clear from context which space is being used. The binary pointset operation $\implies$ can be defined in terms of disjuction and negation in the usual way.
\par Next we turn to quantifiers, by extending the existential projection operator over $\omega$ from earlier. If $P \subseteq \mathcal{X} \times \mathcal{Y}$, then define
\[ \exists^{\mathcal{Y}}P = \{x \in \mathcal{X}: \exists y P(x,y)\} \]
We call this operation \textbf{projection along $\mathcal{Y}$} or \textbf{existential quantification on} $\mathcal{Y}$. Next, if $P \subseteq \mathcal{X} \times \mathcal{Y}$, define 
\[ \forall^{\mathcal{Y}}P = \neg \exists^{\mathcal{Y}} \neg P \]
We call this operation \textbf{dual projection along} $\mathcal{Y}$ or \textbf{universal quantification on} $\mathcal{Y}$. Finally, for pointsets $P \in \mathcal{X} \times \omega$, we have the \textbf{bounded number quantifiers}: 
\[ \exists^{\leq}P(x,n) \iff (\exists m \leq n)P(x,m) \]
\[ \forall^{\leq}P(x,n) \iff (\forall m \leq n)P(x,m) \]

\begin{definition}
	Let $\Lambda$ be a pointclass, and $\Phi$ a $k$-ary pointset operation. We say that $\Lambda$ is \textbf{closed} under $\Phi$ if whenever $P_1,...,P_k$ are pointsets in $\Lambda$ and $\Phi(P_1,...,P_k)$ is defined, we have that $\Phi(P_1,...,P_k)$ is also in $\Lambda$. We say that $\Lambda$ is \textbf{closed under continuous substitution} if for every continuous function $f:\mathcal{X} \to \mathcal{Y}$ and every $P \in \Lambda$ with $P \subseteq \mathcal{Y}$, we have that $f^{-1}[P] \in \Lambda$. (Note we are using brackets to indicate and note that "inverse image of" is being viewed as a unary pointset operation.)   
\end{definition}

\begin{lemma}
	Suppose that $\Lambda$ is a pointclass closed under continuous substitution. Let $f_1:\mathcal{X} \to \mathcal{Y_1},...,f_m:\mathcal{X}\to \mathcal{Y_m}$ be continuous functions, and $Q \subseteq \mathcal{Y_1} \times \ldots \times \mathcal{Y_m}$ a pointset in $\Lambda$. Then the pointset $P$ defined by 
	\[P(x) \iff Q(f_1(x),...,f_m(x)) \]
is also in $\Gamma$
\end{lemma}
\begin{proof}
	todo
\end{proof}

\begin{theorem}
	Each Borel pointclass $\bm{\Sigma}^0_n$ (for $n \geq 1$) is closed under continuous substitution, conjunction and disjunction, bounded quantification and unbounded existential quantification. Each dual pointclass $\bm{Pi}_n^0$ is closed under continuous substitution, conjunction and disjunction, bounded number quantification and unbounded universal quantification. Finally, each ambiguous Borel pointclass $\bm{\Delta}_n^0$ is closed under continuous substitution, conjunction and disjunction, negation, and bounded number quantification.
\end{theorem}
\begin{proof}
	todo
\end{proof}


\begin{definition}
We will define a very general notion of \textbf{parametrization}: Any surjective function $\pi: I \to \mathcal{S}$ can be viewed as a parametrization of $\mathcal{S}$ on $I$, with $I$ sometimes called a \textbf{code set}. Here we are interested in the case where $\mathcal{S}$ is the restriction of a given pointclass $\Gamma$ to some particular product space $\mathcal{X}$, i.e. 
\[ \mathcal{S} = \Gamma  \restriction \mathcal{X} = \{P \subseteq \mathcal{X}: P \in \Gamma\} \]
In particular we are interested in parametrizations of $\Gamma \restriction \mathcal{X}$ on product spaces. If $P \subseteq \mathcal{Y} \times \mathcal{X}$ and $y \in \mathcal{Y}$, define the \textbf{$y$-section of} $P$ as $P_y = \{x \in \mathcal{X}: P(y,x)\}$. (I.e. $P_y$ is the \textit{slice} of $\mathcal{X}$ determined by $y$.) 
Now, we say that a pointset $G \subseteq \mathcal{Y} \times \mathcal{X}$ is \textbf{universal} for $\Gamma \restriction \mathcal{X}$ if $G \in \Gamma$ and the map $y \mapsto G_y$ is a parametrization of $\Gamma \restriction \mathcal{X}$ in $\mathcal{Y}$. i.e. for $P \subseteq \mathcal{X}$, we have 
\[ P \in \Gamma \iff \textrm{for some $y \in \mathcal{Y}$,} P = G_y \]
So basically, $\mathcal{Y}$ parametrizes $\Gamma \restriction \mathcal{X}$ in the conventional calculus sense, insofar as that for and set $P \in \Gamma \restriction \mathcal{X}$, there is a $y \in \mathcal{Y}$ which gives us $P$ as the  $y$-slice $G_y$. 
\par A pointclass $\Gamma$ is \textbf{$\mathcal{Y}$-parametrized} if for every product space $\mathcal{X}$ there is some $G \subseteq \mathcal{Y} \times \mathcal{X}$ which is universal for $\Gamma \restriction \mathcal{X}$.  
\end{definition}
Let $N_0,N_1,...$ be an enumeration of a basis for the topology of some product space $\mathcal{X}$ (recall all of the spaces that we care about are Polish and thus separable) and define $O \subseteq \mathcal{N} \times \mathcal{X}$ by 
\[ O(\epsilon,x) \iff \exists n[x \in N_{\epsilon(n)}] \]
Then clearly $O$ is some union of basic neighborhoods in $\mathcal{X}$ (no it isn't), and is open, and for each $P \subseteq \mathcal{X}$, it of course must be the case that there exists a $\epsilon \in \mathcal{N}$ such that 
\[ P = O_{\epsilon} = \bigcup_n N_{\epsilon(n)} \]
\section{The Kleene Hierarchy}
\begin{definition}
    Suppose $\mathcal{M}$ is a separable metric space with distance function $d$. A \textbf{recursive presentation} of $\mathcal{M}$ is a countable dense set (recursively enumerable?) in $\mathcal{M}$ $D = \{r_n\}_{n \in \omega}$ such the decision problem:
    \begin{center}
        Given two elements $r_i,r_j \in D$ and a positive rational $q \in \mathbb{Q}$, is $d(r_i,r_j) \leq q$?
    \end{center}
    is decidable, as is the problem in which we replace $\leq$ with $<$. To state this slightly more formally a recursive presentation is a sequence $r_1,r_2,...$ in $\mathcal{M}$ such that the following hold:
    \begin{itemize}
        \item[(1)] The set is dense in $\mathcal{M}$.
        \item[(2)] The relations 
        \[ P(i,j,m,k) \iff d(r_i,r_j) \leq \frac{m}{k+1} \]
        \[ Q(i,j,m,k) \iff d(r_i,r_j) < \frac{m}{k+1} \]
    \end{itemize}
    are both recursive. 
\end{definition}
All of the interesting spaces that we care about admit a recursive presentation. In the case of $\omega$, in which we have the trivial distance function 
\[ d(i,j) = \begin{cases}
                0 & \textrm{ if $i=j$} \\
                1 & \textrm{ if $i \neq j$}
            \end{cases}\]
we can take the identity sequence $r_i = i$, and note that 
\[ d(i,j) \leq \frac{m}{k+1} \iff (i=j)\vee(k+1 \leq m) \]
\[ d(i,j) < \frac{m}{k+1} \iff (i=j) \vee (k+1 < m) \]
Since the sequence is dense by virtue of literally being the entire space, and these relations being obviously recursive, the identity sequence is a recursive presentation of the discrete space $\omega$.
\par As a recursive presentation of Baire space, we can take the set of all eventually $0$ sequences of integers, i.e. $R_i(n) = (i)_n$. [Show this is a recursive presentation.]
\par [do the rest of them]
\par Fix, once and for all, a recursive presentation for each basic space, and subsequently a recursive presentation of each finite product space. If $X$ is a basic space, $x_0 \in X$, and $p$ a nonnegative rational number, then let $B(x_0,p)$ denote the open ball in $X$ with center $x_0$ and radius $p$. Take $r_i$ to be the recursive presentation of $X$. For each $s \in \omega$, let 
\[ B_s = B(X,s) = B(r_{(s)_0},\frac{(s)_1}{(s)_2+1}) \]
where recall $(s)_i$ is the recursive decoding function for the $i^{th}$ entry of whatever sequence of integers $s$ is a coding of. Note while we're here that $(0)_1 = 0$, so we always have that $B_0 = \varnothing$. Thus by way of the partial recursive decoding function (see computability theory notes), the sequence $B_0,B_1,...$ defines an effective numbering of a basis for any basic space being talked about. For a product space $\mathcal{X} = X_1 \times X_2 \times ... \times X_k$ and an $s \in \omega$, let 
\begin{align}
    N_s = N(\mathcal{X},s) &= B(X_1,(s)_1) \times B(X_2,(s)_2) \times ... \times B(X_k,(s)_k) \\
                &= \{(x_1,...,x_k): x_1 \in B_{(s)_1} \wedge ... \wedge x_k \in B_{(s)_k}\}
\end{align} 
Then by the same argument the sequence $N_0,N_1,...$ is an effective numbering of a neighborhood basis for the topology of $\mathcal{X}$. Note that it is natural to read $N(\mathcal{X},s)$ as "the $s^{th}$ basic neighborhood of the space $\mathcal{X}$." 
\par For a basic space $X$, the basic open ball $B_s$ has center $r_{(s)_0}$. Similarly, for a product of two spaces $X_1 \times X_2$, we use the original index $s$ by taking the first entry $(s)_1$, and using that as the index of the triple for the factor ball in $\mathcal{X}$, i.e. the first coordinate of $N_s$ is actually $r^1_{((s)_1)_0}$, and similarly the second coordinate is $r^2_{((s)_2)_0}$. (The superscripts distinguish between the recursive presentations of $X_1$ and $X_2$ respectively. Thus we can define
\begin{align}
    center(N_s) = r_i := (r^1_{(i)_1},...,r^k_{(i)_k})
\end{align}
where
\begin{align}
    (i)_1 = ((s)_1)_0, (i)_2 = ((s)_2)_0,...,(i)_k = ((s)_k)_0
\end{align}
Note that the center of a ball in any product space is computable in the sense that we can decide a point in our 'dense presentation' is within some desired distance of the center. We can also define something of a radius, by
\begin{align}
    radius(N_s) = max\{p_1,...,p_k\}
\end{align}
where
\begin{align}
    p_j = \frac{((s)_j)_1}{((s)_j)_2+1}, j \in \{1,...,k\}
\end{align}
Note that clearly, given any $s$, both the numerator and the denominator of the radius of the nbhd $N_s$ are recursive functions of $s$, and so we can always perform calculations involving these numbers, just given the index of the ball $s$.
A small technical issue will prove to be confusing later if not addressed directly now. Let $X$ be a basic space. For this, the sequence $B(X,0),B(X,1)...$ defines a numbering of a basis for $X$. However, if we think of $X$ as a single factor product space $\mathcal{X} = X$ (which should for all practical purposes make no difference) then the sequence $N(\mathcal{X},0),N(\mathcal{X},1),...$ defines a \textit{different numbering} of the same basis, since for any $s$, $N(\mathcal{X},s) = B(X,(s)_1)$, which is likely a different ball than $B(X,s)$. Notice also that, in the other direction, $B(X,s) = N(X,\langle 0,s \rangle)$ for all $s$. Two lemmas will help to make these kinds of quips more manageable:
\begin{lemma}
	For each two product spaces $\mathcal{X},\mathcal{Y}$, there are recursive functions $f,g,h$ such that 
	\[ N(\mathcal{X},s) \times N(\mathcal{Y},t) = N(\mathcal{X} \times \mathcal{Y}, f(s,t)) \]
	\[ N(\mathcal{X} \times \mathcal{Y},s) = N(\mathcal{X},g(s)) \times N(\mathcal{Y},h(s)) \]
\end{lemma}
\begin{proof}
	If $\mathcal{X} = X_1 \times \ldots \times X_k$ and $\mathcal{Y} = Y_1 \times \ldots \times Y_l$, then 
	\[ N(\mathcal{X} \times \mathcal{Y},s) = B(X_1,(s)_1) \times \ldots \times B(X_k,(s)_k) \times B(Y_1,(s)_{k+1}) \times \ldots \times B(Y_l,(s)_{k+1}) \]
	From this, simply take
	\[ f(s,t) = \langle 0,(s)_1,...,(s)_k,(t)_1,...,(t)_l \rangle \]
	\[ g(s) = \langle 0,(s)_1,...,(s)_k \rangle \]
	\[ h(s) = \langle 0,(s)_{k+1},...,(s)_{k+l} \rangle \] 
	These are primitive recursive, and a moments thought confirms that these work as desired.
\end{proof}
We need one more technical lemma in order to make it manageable to deal with these codings of neighborhoods:
\begin{lemma}
	For each product space $\mathcal{X}$, there is a recursive function $f$ such that 
	\[ N(\mathcal{X},s) \cap N(\mathcal{X},t) = \bigcup_n N(\mathcal{X},f(s,t,n)) \]
	Also, there is a recursive function $g$ such that
	\[ \bigcap_{i \leq m} N(\mathcal{X},(u)_i) = \bigcup_n N(\mathcal{X},g(u,m,n)) \]
\end{lemma}
\begin{proof}
	We begin with the second assertion. Let $X$ be a basic space with the recursive presentation $\{r_0,r_1,...\}$, and let $B(x_0,p_0),...,B(x_m,p_m)$ be $m+1$ open balls in $X$. 
	Note that
	\begin{align}
		x \in & B(x_0,p_0) \cap \ldots \cap B(x_m,p_m) \\
		 &\iff \exists i \exists k [d(r_i,x) < \frac{(k)_1}{(k)_2+1} 
		\wedge d(x_0,r_i) < p_0 - \frac{(k)_1}{(k)_2+1} \wedge \ldots 
		\wedge d(x_m,r_i) < p_m - \frac{(k)_1}{(k)_2+1}]
	\end{align}
The implication from right to left is because we are claiming that there is a point which is close enough to $x$ as well as all of the other centers that $x$ itself must be, and from left to right follows from the fact that recursive presentations are dense in their spaces, so assuming the intersection is nonempty means that the intersection is a nonempty open set, and must have a point in the dense set inside of it, in turn meaning there is an open ball around that point contained in the open set, and that would imply that the conditions described on the right side are met. 
\par Now, note that if $s = \langle s_0,...,s_m \rangle$ is a collection of indices for balls being intersected, with radii $p_0,...,p_m$ and centers $r_0,...,r_m$, and $n$ is the index of some other ball $n$ with radius $p_n$ and center $r_n$, then relation
\[ P(s,n) \iff \left[ (d(r_0,r_n) < p_0 - p_n) \wedge (d(r_1,r_n) < p_1 - p_n) \wedge \ldots \wedge (d(r_m,r_n) < r_m - p_n) \right] \]
is recursive, and furthermore by the previous observation we have 
\[ x \in B(X,s_0) \wedge \ldots \wedge B(X,s_m) \iff \exists n [x \in B(X,n) \wedge P(s,m,n)] \]
(I honestly don't see why the second argument is necessary considering the existence of the $lh$ function which returns the lengths of coded strings, but this gets so messy that I decided to roll with it and keep the argument.) Thus,
\[ \bigcap_{i \leq m}B(X,s_i) = \bigcup_{n: P(s,m,n)} B(X,n) \]
Now, let $\mathcal{X} = X_1 \times \ldots \times X_k$, and consider an intersection
\begin{align}
	\bigcap_{i \leq m} N(\mathcal{X},(u)_i) &= \bigcap_{i \leq m} [B(X_1,((u)_i)_1) \times \ldots \times B(X_k,((u)_i)_k)] \\
		&= \left[ \bigcap_{i \leq m} B(X_1,((u)_i)_1)\right] \times \ldots \times \left[ \bigcap_{i \leq m} B(X_k,((u)_i)_k)\right]
\end{align} 
(where the second equality follows from the intersection of a product being the product of the intersections). By the earlier results, we can let $P_1,...,P_k$ be the recursive relations guaranteed to exist for each factor. Then for each $j = 1,...,k$,
\begin{align}
	\bigcap_{i \leq m} B(X_j,((u)_i)_j) &= \bigcup_{i \leq m} \{B(X_j,n): P_j(\langle ((u)_0)_j,...,((u)_m)_j \rangle,n) \}\\
		&= \bigcup_{n: P^*_j(u,m,n)} B(X_j,n)
\end{align}
where $P^*_j$ is just $P_j$ in which the first argument has been substituted with the obviously recursive mapping $u \mapsto \langle ((u)_0)_j,...,((u)_m)_j \rangle$. Hence,
\begin{align}
	\bigcap_{i \leq m} N(\mathcal{X},(u)_i) &= \left[ \bigcup_{n: P_1^*(u,m,n)} B(X_1,n) \times \ldots \times \bigcup_{n: P^*_k(u,m,n)} B(X_k,n)\right] \\
		&= \bigcup_n \{ B(X_1,(n)_1) \times \ldots \times B(X_k,(n)_k): P^*_1(u,m,(n)_1) \wedge \ldots \wedge P^*_k(u,m,(n)_k) \} \\
		&= \bigcup_{n: P^*(u,m,n)} N(\mathcal{X},n)
\end{align} \
where $P^*$ is the obviously recursive relation obtained by wedging together all of the $P^*_j$ relations in the previous step. Since $P^*$ is a recursive relation, we have that the function $g(u,m,n) := n\chi_{P^*}(u,m,n)$ is a recursive function, and from this we can plainly see that this function satisfies the second claim of the lemma. The first claim follows immediately by setting $f(s,t,n) := g(\langle s,t \rangle,1,n)$.
\end{proof}
Next, we define the "partial recursive sets" of a space.
\begin{definition}
    A pointset $G \subseteq \mathcal{X}$ is \textbf{semirecursive} (or \textbf{partial recursive}) if it is the recursively enumerable union of basic neighborhoods. To state this more formally, $G$ is semirecursive if there exists a recursive function $\epsilon:\mathbb{N} \to \mathbb{N}$ such that
    \[ G = \bigcup_{n \in \omega} N(\mathcal{X},\epsilon(n)) \]
    Note that $\epsilon$ can alternatively be seen as an element of Baire space which is itself recursive.
\end{definition}
The collection of all partial recursive sets for a topological space can be seen in some sense as a recursive approximation of the space itself. Thus we refer to the family of all semirecursive sets in $\mathcal{X}$ as the \textbf{recursive topology} on $\mathcal{X}$, despite it not being closed under arbitrary unions. The closure properties that it does have should be quite familiar however, and we state these next.
\par Before doing so, we define what it means for a function to be \textbf{trivial}. Intuitively, a function is trivial if all of it's output entries are just projections of the input. I.e. the mapping $(x_1,x_2,x_3,x_4) \mapsto (x_2,x_1,x_1)$ is trivial. More formally, let $\mathcal{Y} = Y_1 \times Y_2 \times ... \times Y_l$. A function $f: \mathcal{X} \to \mathcal{Y}$ is trivial if there exist projection functions $f_1:\mathcal{X} \to Y_1,...,f_l:\mathcal{X} \to Y_l$ such that $f(x) = (f_1(x),...,f_l(x))$. A moments thought shows that this definition captures our intuition.
\begin{theorem}
    The pointclass of semirecursive sets contains the empty set, every product space $\mathcal{X}$, every recursive relation on $\omega^k$ for each $k$, every basic neighborhood $N(\mathcal{X},s)$, and the \textbf{basic neighborhood relation} defined by $(x,s) \in R \iff x \in N(\mathcal{X},s)$. Furthermore, the semirecursive sets are closed under substitution of trivial functions, logical operators $\wedge$, $\vee$, bounded universal quantification, and unbounded existential quantification.
\end{theorem}
\begin{proof}
    Before we begin, it will be helpful to note the following: If $P^*$ is a partial recursive relation, $f$ a recursive function,and we define
    \[ P = \bigcup_{n: P^*(n)} N(\mathcal{X},f(n)) \]
    then $P$ is partial recursive, since the function
    \[ \epsilon(n) = \begin{cases}
    					f(n) & \textrm{ if } P^*(n) \\
    					0 & \textrm{ else}
    				 \end{cases} \]
    is recursive, and noting that $N(\mathcal{X},0) = \varnothing$, it is clear that
    \[ P = \bigcup_{n \in \omega} N(\mathcal{X},\epsilon(n)) \]
    Now, clearly we already noted that $\varnothing = \bigcup_n N(\mathcal{X},0)$ (i.e. the recursive function we are using is the zero function). Similarly trivial are that $\mathcal{X} = \bigcup_n N(\mathcal{X},n)$ (corresponding to the identity function), and that $N(\mathcal{X},s) = \bigcup_n N(\mathcal{X},s)$ for any $s$ (using the constant functions).
    \par Towards showing that all recursive relations on $\omega^k$ are semirecursive, in the new sense we defined, we first note that singleton sets in $\omega$ are basic neighborhoods: for any integer $i$,
    \[ B(\omega, \langle i,1,1 \rangle ) = \{ m: d(m,i) < \frac{1}{2} \} = \{i\} \] 
    Similarly for any singleton sets $(n_1,...,n_k) \in \omega^k$, we have that
    \[ (n_1,...,n_k) = N(\omega^k, \langle 0,\langle n_1,1,1 \rangle,...,\langle n_k,1,1 \rangle \rangle) \]
    Therefore, if $R \subseteq \omega^k$ is partial recursive, then 
    \[ R = \bigcup_{\vec{n}:R(\vec{n})} N(\omega^k,\langle 0,\langle n_1,1,1 \rangle,...,\langle n_k,1,1 \rangle \rangle)  \]
    Since the coding function is recursive, we have by our observation from earlier that $R$ is partial recursive. 
    \par Next we turn towards the basic nbhd relation. Note first that for any singleton $\{s\} \subseteq \omega$, $\{s\} = B(s,\frac{1}{2}) = B(\omega,\langle s,1,1 \rangle ) = N(\omega, \langle 0,\langle s,1,1 \rangle \rangle )$, and thus  
\begin{align}
 	N(\mathcal{X},s) \times \{s\} &= N(\mathcal{X},s) \times N(\omega,\langle 0, \langle s,1,1 \rangle \rangle ) \\
 		&= N(\mathcal{X} \times \omega, f(s,\langle 0,\langle s,1,1 \rangle \rangle )
\end{align}
Where $f$ is a recursive function whose existence was assured by our technical lemma from earlier. Observe now then that
\[x \in N(\mathcal{X},s) \iff (x,s) \in \bigcup_n N(\mathcal{X} \times \omega, f(s,\langle 0,\langle s,1,1 \rangle \rangle)  \]
So the basic nbhd relation is semirecursive. 
\par now for the closure properties. First suppose that 
\[ P = \bigcup_{n \in \omega} N(\mathcal{X},\alpha(n)), Q = \bigcup_{m \in \omega} N(\mathcal{X},\beta(n)) \]
with $\alpha,\beta$ recursive. Then clearly the function $\sigma$ which alternates back and forth between $\alpha$ and $\beta$ outputs is also recursive, and so $P \cup Q = \bigcup_t N(\mathcal{X},\sigma(t))$ is semirecursive. For intersections, we see that 
\begin{align}
	P \cap Q &= \left[ \bigcup_n N(\mathcal{X},\alpha(n))\right] \cap \left[ \bigcup_m N(\mathcal{X},\beta(m))\right] \\
			&= \bigcup_{m,n} \left[ N(\mathcal{X},\alpha(n)) \cap N(\mathcal{X},\beta(m)) \right] \\
			&= \bigcup_{n,m,s} N(\mathcal{X},f(\alpha(n),\beta(m),s))
\end{align}
Where $f$ is the recursive function whose existence is assured by our earlier lemma about intersection. Thus,
\[ P \cap Q = \bigcup_t N(\mathcal{X},f(\alpha((t)_0),\beta((t)_1),(t)_2)) \]
and so the intersection is semirecursive. Moving to unbounded existential quantification, let $\epsilon(n)$ be a recursive sequence so that $Q = \bigcup_n N(\mathcal{X} \times \omega, \epsilon(n))$ is semirecursive, and define $P = \exists^{\omega} Q$, i.e.
\[P(x) \iff \exists m Q(x,m) \]
Then we have
\begin{align}
	P(x) &\iff (\exists m)(\exists n)[(x,m) \in N(\mathcal{X} \times \omega, \epsilon(n))] \\
		&\iff (\exists m)(\exists n)[x \in N(\mathcal{X},g(\epsilon(n)) \wedge m \in N(\omega,h(\epsilon(n)))] 
\end{align}
Where the $g,h$ are recursive as assured by our little coding lemma. Define the relation 
\[ R(m,n) \iff m \in N(\omega,h(\epsilon(n))) \]
This is clearly recursive, since $h \circ \epsilon$ is recursive, and so this is a recursive relation on $\omega^2$ which coincides with a recursive pointset as we've shown earlier in this proof. Then
\[ P(x) \iff (\exists t)[x \in N(\mathcal{X},g(\epsilon((t)_0))) \wedge R((t)_1,(t)_0)] \]
Clearly this is enough to see that $P$ is semirecursive. Moving next to trivial substitution, suppose $f:X_1 \times \ldots \times X_k \to \mathcal{Y}$ is trivial, i.e. $f(x_1,...,x_k) = (x_{i_1},...,x_{i_l})$, where each index $i_j$ is between $1$ and $k$. (So really $\mathcal{Y}$ is shorthand for $X_{i_1} \times \ldots \times X_{i_l}$.) Let $Q = \bigcup_n N(\mathcal{Y},\epsilon(n))$ be an arbitrary semirecursive set in $\mathcal{Y}$. Define $P \subseteq X_1 \times \ldots \times X_k$ by $x \in P \iff Q(f(x))$. Then
\begin{align}
	P(x_1,...,x_k) &\iff (\exists n)[(x_{i_1},...,x_{i_l}) \in N(\mathcal{Y},\epsilon(n))] \\
				   &\iff (\exists n)[x_{i_1} \in B(X_{i_1},(\epsilon(n))_1) \wedge \ldots \wedge x_{i_l} \in B(X_{i_l},(\epsilon(n))_l)]
\end{align}
For a fixed $j$, we of course have
\begin{align}
	x_j \in B(X_j,m) &\iff (\exists t)[x_1 \in B(X_1,(t)_1) \wedge \ldots \wedge x_j \in B(X_j,m) \wedge \ldots \wedge x_k \in B(X_k,(t)_k)]\\
					 &\iff (\exists t)[(x_1,...,x_k) \in N(\mathcal{X},f_j(m,t))]
\end{align}
where $f_j$ is obviously recursive. Since the basic nbhd relation is semirecursive as we showed already, it is easy to see that \[ R_j(x_1,...,x_k) \in N(\mathcal{X},f_j(m,t)) \]
is semirecursive (HE'S CHEATING TOO! THIS IS BS, THIS LITERALLY ASSUMES CLOSURE UNDER TRIVIAL SUBSTITUTION, GOD DAMN THIS IS TEDIUS) so by closure under $\exists^{\omega}$ we have 
\[ P(x_1,...,x_k) \iff (\exists n)[R^*_{i_1}(x_1,...,x_k,n) \wedge R^*_{i_l}(x_1,...,x_k,n)] \]
with suitable semirecursive $R^*_{i_1},...,R^*_{i_l}$ and $P$ semirecursive by closure under conjunction and projection (I'm just copying him word for word at the end of this one, fuck it. This is dumb and obvious.)
\par Finally we turn to bounded quantification. Let $Q \subseteq \mathcal{X} \times \omega$ be semirecursive, and define 
\[ P(x,n) \iff (\exists i \leq n)Q(x,i) \]
Then if we define $R(x,n,i) \iff i \leq n$, and $S(x,n,i) \iff Q(x,i)$ then these are both semirecursive by closure under trivial substitution (using $(x,n,i) \mapsto (i,n)$ and $(x,n,i) \mapsto (x,i)$) as well as the semirecursiveness of $\leq$ and $Q$, then we have 
\begin{align}
	P(x,n) &\iff (\exists i)[i \leq n \wedge Q(x,i)] \\
		&\iff (\exists i)[R(x,n,i) \wedge S(x,n,i)]
\end{align}
Giving us closure under bounded existential quantification. For bounded universal quantification, considering $P := \forall^{\leq}Q$, i.e.
\[ P(x,n) \iff (\forall i \leq n)Q(x,i) \]
then note
\begin{align}
	P(x,n) &\iff (\forall i \leq n)(\exists m)[(x,i) \in N(\mathcal{X} \times \omega, \epsilon(m))] \\
		   &\iff (\exists s)(\forall i \leq n)[(x,i) \in N(\mathcal{X},\epsilon((s)_i))] \\
		   &\iff (\exists s)(\forall i \leq n)[x \in N(\mathcal{X},f_1(s,i)) \wedge i \in N(\omega,f_2(s,i))]
\end{align}
where $f_1$ and $f_2$ are assured to exist by our little coding lemma. More symbol pushing, and introducing another quantifier:
\begin{align}
	P(x,n) \iff (\exists s)(\exists u)[(\forall i \leq n)[f_1(s,i) = (u)_i] \wedge (\forall i \leq n)[x \in N(\mathcal{X},(u)_i)] \wedge (\forall i \leq n)[i \in N(\omega,f_2(s,i)]]
\end{align}
I don't have a god damn clue what he's doing in these last two steps. God this is the worst. Apparently you use the the lemma about intersections and rearrange
\end{proof}
\begin{definition}
    We will say that a product space $\mathcal{X}$ is of \textbf{type 0} if $\mathcal{X} = \omega^k$ for some integer $k$. That is, the type $0$ spaces are precisely the discrete product spaces. A space $\mathcal{X} = X_1 \times ... \times X_k$ is of \textbf{type 1} if each $X_i$ is either $\omega$ or Baire space $\mathcal{N}$, and at least one of these is the latter. (This means that type 0 and type 1 are mutually exclusive properties.) We will also refer to pointsets which are subsets of these spaces as either type 0 or type 1.   
\end{definition}
Note that our definition of semirecursiveness seems to depend on the particular choice of recursive presentation which we have fixed, as well as the way that we coded the basic neighborhoods (remember that this depended on an enumeration of triples of integers). While the former is true, the latter is an illusion, as the next result shows:
\begin{lemma}[Moschovakis Exercise 3C.12]
    The collection of semirecursive sets is the smallest pointclass which contains all recursive pointsets of type 0 and the relation $P^X \subseteq X \times \omega^3$ for each basic space $X$:
    \[ P^X(x,i,m,k) \iff d(r_i,x) < \frac{m}{k+1} \]
    and which is closed under trivial substitutions, logical operations $\wedge$, $\vee$, bounded universal quantification, and unbounded existential quantification.
\end{lemma}
\begin{proof}
    todo
\end{proof}
Since this would be true regardless of the coding we chose of triple to number the basic neighborhoods, we now have that we obtain the same family no matter what, so that aspect of the construction was arbitrary. 
\par Note while we have defined what it means for general pointsets of arbitrary product spaces to be semirecursive, we haven't yet defined what it means to be recursive. Note however that for type 0 spaces, recursiveness is already defined! It would be natural to say that a pointset $G$ is recursive if both $G$ and $\neg G$ are recursive. However, we can only safely do this if the definition aligns with the old one in the special case of type 0 spaces. Fortunately, it does.
\begin{theorem}
    A pointset $P \subseteq \omega^k$ of type 0 is semirecursive iff there is a recursive relation $R \subseteq \omega^{k+1}$ such that 
    \[ P(x) \iff \exists n R(x,n) \]
    Moreover P is recursive iff both $P$ and $\neg P$ are semirecursive.
\end{theorem}
\begin{proof}
    todo
\end{proof}
Thus, we say that $G$ is \textbf{recursive} if both $G$ and $\neg G = G^c$ are recursive, and note that this generalizes the original definition of recursiveness to arbitrary product spaces of interest. We have the expected closure properties of recursive sets, and then a few new ones:
\begin{theorem}
    The pointclass of recursive sets contains the empty set, every product space $\mathcal{X}$, every recursive relation on $\omega$ (book just says $\omega$, but I'm guessing it's supposed to be $\omega^k$), and the basic neighborhood relation for spaces which are of type 0 or type 1. Furthermore, it is closed under substitution of trivial functions, logical operators $\neg, \wedge$, and $\vee$, and bounded quantification. Finally, the pointset 
    \[\{(\alpha,n,w): \alpha(n) = w\} \]
    is recursive (where $\alpha$ is an element of Baire space).
\end{theorem}
\begin{proof}
    todo
\end{proof}
\begin{theorem}
    A pointset $P \subseteq \mathcal{X}$ of type 0 or type 1 is semirecursive iff there is a recursive $R \subseteq \mathcal{X} \times \omega$ such that
    \[ P(x) \iff \exists n R(x,n) \]
\end{theorem}
\begin{proof}
    todo
\end{proof}
The more general analog of the above theorem needs only a small addendum:
\begin{theorem}
    A pointset $P \subseteq \mathcal{X}$ (regardless of type) is semirecursive iff there is a semirecursive $P^*\subseteq \omega$ such that 
    \[ P(x) \iff \exists s (x \in N(\mathcal{X},s) \wedge P^*(s)) \]
    More generally, $P \subseteq \mathcal{X} \times \mathcal{Y}$ is semirecursive iff there is a semirecursive $P^* \subseteq \omega^2$ such that 
    \[ P(x) \iff \exists s \exists y (x \in N(\mathcal{X},s) \wedge y \in N(\mathcal{Y},t) \wedge P*(s,t)) \]
    More specifically, $P \subseteq \omega \times \mathcal{X}$ is semirecursive iff there is a semirecursive $P^* \subseteq \omega^2$ such that 
    \[ P(n,x) \iff \exists s (x \in N(\mathcal{X},s) \wedge P^*(n,s)) \]
\end{theorem}
Some exercises which are interesting results and I should definitely do eventually:
\begin{example}[Exercise 3C.6]
    For any space $\mathcal{X}$, the relation $\neq$ is partial recursive. 
\end{example}
\begin{example}[Exercise 3C.11]
    The $<$ relation on $\mathbb{R}_{std}$ is partial recursive.
\end{example}
\begin{example}[Exercise 3C.13]
    A pointset $P \subseteq \mathcal{X}$ is open iff there is a semirecursive $Q \subseteq \mathcal{X} \times \mathcal{X}$ and a sequence $\{x_n\} \in \mathcal{N}$ such that $P(x) \iff Q(\{x_n\},x)$. 
\end{example}
With all of this, we now know what we must do. 
\begin{definition}
    We define the \textbf{lightface hierarchy} (otherwise known as the \textbf{Kleene hierarchy} or the \textbf{arithmetic hierarchy}) as follows:
    \[ \Sigma_1^0 = \textrm{ all semirecursive pointsets} \]
    \[ \Sigma^0_{n+1} = \exists^{\omega} \neg \Sigma^0_n \]
    \[ \Pi_n^0 = \neg \Sigma^0_n \]
    \[ \Delta^0_n = \Sigma^0_n \cap \Pi^0_n \]
\end{definition}
Note that by the quantifier based characterizations which we proved for type 0 spaces, it is clear that the restriction of this hierarchy to the pace $\omega$ is \textit{precisely} the arithmetic hierarchy from logic and recursion theory. We now see that hierarchy as the 'effective version' of the topological Borel hierarchy of $\omega$ built out of open sets. Although this hierarchy is trivial, the arithmetic hierarchy itself very much isn't, and we now have the notion of the 'effective' version of the Borel hierarchy for general spaces of interest. There is an 'effective version' of the Borel hierarchy for Baire space, Cantor space, and so forth, and these all satisfy the basic properties one would hope for.  
\begin{definition}
    Let $f:\mathcal{X} \to \mathcal{Y}$ be a function. The \textbf{neighborhood diagram} of $f$ is a relation $G^f \subseteq \mathcal{X} \times \omega$, defined by $(x,s) \in G^f \iff f(x) \in N(\mathcal{Y},s)$. (i.e. $G^f$ is the collection of pairs $(x,s)$ such that $f(x)$ is in the $s^{th}$ neighborhood of $\mathcal{Y}$.)
    We say that $f$ is \textbf{recursive} if it's neighborhood diagram is partial recursive. For a (lightface!) pointclass $\Gamma$, we say that $f$ is \textbf{$\Gamma$-recursive} if it's neighborhood diagram is in $\Gamma$. 
\end{definition}
We should make sure that this lines up with our definition in the special case $f:\omega \to \omega$. Recall that our canonical recursive presentation for $\omega$ was simply the identity sequence, so to call $f$ recursive in the sense above would be to say that the relation $G^f = \{(m,n): m = f(n)\} = \{(f(n),n): n \in \omega\}$, i.e. the neighborhood diagram for functions of natural numbers is just the graph, and it is known that a function is recursive iff it's graph is partial recursive, so this definition is indeed a generalization of what we had originally. 
\par Note that if $f$ is recursive in the sense described above, then we can effectively compute arbitrarily good approximations of $f(x)$ for any $x$, in the following way. For each $s$, first compute $radius(N_s)$ and check if it is less than $\frac{1}{2^n}$, where $n$ is least such that $\frac{1}{2^n} < \epsilon$, where $\epsilon$ is the desired accuracy. If this check is passed, next check using the recursiveness of the neighborhood diagram whether $f(x) \in N_s$. If this check is met, return $center(N_s)$. I assumed the neighborhood diagram was fully recursive, but partial recursive will do just fine, using a dovetail search. Assuming properties of the spaces we care about (which, specifically?) a good enough approximation will eventually be obtained. This observation points out how, in a vague sense(?), recursiveness of function is the "effective refinement" of the topological notion of continuity. We traded the Borel hierarchy for the arithmetic hierarchy, and now we trade continuity for recursiveness.
\par Not all constant functions are recursive - only the ones whose constant value can be effectively approximated to any desired accuracy, which might even vary in meaning depending on the space. However, all of the constant functions which are constantly points $r$ in the the recursive presentation of a space are certainly recursive. It makes no sense to talk about a function being recursive "at a point $x$". The analog between recursiveness and continuity only applies globally.
\begin{lemma}[Dellacherie]
    Let $\Gamma$ be a (lightface!) $\Sigma$-pointclass. A function $f:\mathcal{X} \to \mathcal{Y}$ is $\Gamma$-recursive iff for every semirecursive $P \subseteq \omega \times \mathcal{Y}$, the relation $P^f \subseteq \omega \times \mathcal{X}$ defined by $(n,x) \in P^f \iff P(n,f(x))$ is in $\Gamma$
\end{lemma}
\begin{proof}
    todo
\end{proof}
\begin{theorem}
    Let $\Gamma$ be a lightface $\Sigma$-pointclass. 
    \begin{itemize}
        \item[(1)] A function $f:\mathcal{X} \to \mathcal{Y}$ is $\Gamma$-recursive iff the graph of $f$ is in $\Gamma$
        \item[(2)] If $Q \subseteq Y_1 \times ... \times Y_l$, and 
        \[ P(x) \iff Q(f_1(x),f_2(x),...,f_l(x)) \]
        where each $f_i$ is either trivial or $\Gamma$-recursive into $\omega$. If $Q$ is in $\Gamma$, then so is $P$.
    \end{itemize}
\end{theorem}
\begin{proof}
    todo
\end{proof}
\begin{theorem}
    Let $\Gamma$ be a $\Sigma$-pointclass 
    \begin{itemize}
        \item[(1)] A function $f\mathcal{X} \to \mathcal{N}$ is $\Gamma$-recursive iff the associated function $f^*:\mathcal{X} \times \omega \to \omega$ defined by 
        \begin{align}
            f^*(x,n) = f(x)(n) \textrm{ i.e. the $n^{th}$ entry of the sequence $f(x)$} 
        \end{align} 
        is $\Gamma$-recursive.
        \item[(2)] A function $f:\mathcal{X} \to \mathcal{Y}$ where $\mathcal{Y} = Y_1 \times ... \times Y_l$ is $\Gamma$-recursive iff 
        \[ f(x) = (f_1(x),...,f_l(x)) \]
        with suitable $\Gamma$-recursive functions $f_1,...,f_l$.
    \end{itemize}
\end{theorem}
\begin{proof}
    todo
\end{proof}
\begin{theorem}
    Let $\Gamma$ be a $\Sigma$-pointclass
    \begin{itemize}
        \item Every trivial function $f:\mathcal{X} \to \mathcal{Y}$ is recursive.
        \item If $f:\mathcal{X} \to \mathcal{Y}$ is $\Gamma$-recursive and $g:\mathcal{Y} \to \mathcal{Z}$ is recursive, then the composition $h(x) = g(f(x))$ is $\Gamma$-recursive. In particular, the class of recursive functions is closed under composition.
    \end{itemize}
\end{theorem}
\begin{proof}
    todo
\end{proof}
\begin{theorem}
    The pointclass $\Sigma_1^0$ (this one specifically and no others!) is closed under recursive substitution.
\end{theorem}
\begin{proof}
    todo
\end{proof}
\begin{definition}
    For any pointclass $\Gamma$, define the \textbf{ambiguous} part of $\Gamma$ to be $\Delta := \Gamma \cap \neg \Gamma$.
\end{definition}
\begin{theorem}
    Let $\Gamma$ be a $\Sigma$-pointclass. A set $P \subseteq \mathcal{X}$ is in $\Delta$ iff its characteristic function $\chi_P$ is $\Gamma$-recursive.  
\end{theorem}
\begin{proof}
    todo
\end{proof}
Next, we generalize the notion of relativization. 
\begin{definition}
    Let $\Gamma$ be a pointclass, and $z \in \mathcal{Z}$. Define the \textbf{relativization} $\Gamma(z)$ of $\Gamma$ to $z$ as follows: $P \subseteq \mathcal{X}$ is in $\Gamma(z)$ iff there exists some $Q \subseteq \mathcal{Z} \times \mathcal{X}$ in $\Gamma$ (unrelativized) such that 
    \[ P(x) \iff Q(z,x) \]
    In particular, the sets $\Sigma^0_1(z)$ are called \textbf{semirecursive in $z$} and the functions which are $\Sigma^0_1(z)$-recursive are called \textbf{recursive in $z$}.
\end{definition}
Next we define what it means for \textit{points} in a space to be recursive.
\begin{definition}
    A point $x \in \mathcal{X}$ is \textbf{$\Gamma$-recursive} if the set of codes of neighborhoods of $x$ is in $\Gamma$, i.e. if 
    \[ \mathcal{U} := \{s \in \omega: x \in N(\mathcal{X},s)\} \]
    is in $\Gamma$. (So, informally, and in the case of regular partial recursiveness, we would be saying a point is partial recursive if the sequence indexing the collection of all basic neighborhoods which contain $x$ is recursive.) We will often call these simply the \textbf{points in $\Gamma$} (as opposed to the point\textit{sets} in $\Gamma$), and occasionally we will consider them to be actual members of $\Gamma$. I.e. when we claim that a point $x \in \Gamma$, what we really mean is that $x$ is $\Gamma$-recursive. We call the points in $\Sigma^0_1$ \textbf{recursive}, and we call the points in $\Sigma^0_1(z)$ \textbf{recursive in $z$}. 
\end{definition}
\begin{theorem}
    Let $\Gamma$ be a $\Sigma$-pointclass. (Starting to question if I know what this actually means)
    \begin{itemize}
        \item[(1)] For eah point $z$, $\Gamma(z)$ is a $\Sigma$-pointclass.
        \item[(2)] A point $x$ is $\Gamma$-recursive iff for each $\mathcal{Y}$, the constant function $y \mapsto x$ is $\Gamma$-recursive
        \item[(3)] If $x$ is recursive in $y$ and $y$ is $\Gamma$-recursive, then $x$ is $\Gamma$-recursive.
        \item[(4)] If $f:\mathcal{X} \to \mathcal{Y}$ is $\Gamma$-recursive, then for each $x \in \mathcal{X}$, $f(x)$ is $\Gamma(x)$-recursive. In particular, if $f:\mathcal{X} \to \mathcal{Y}$ is recursive and $x$ is recursive, then $f(x)$ is $\Gamma(x)$-recursive.
    \end{itemize}
\end{theorem}
\begin{proof}
    todo
\end{proof}
\begin{theorem}
    Let $\Gamma$ be a pointclass in the arithmetic hierarchy, and $\bm{\Gamma}$ be the corresponding analogous boldface pointclass. Then for each product space $\mathcal{X}$, there is a pointset $G \subseteq \mathcal{N} \times \mathcal{X}$ in (lightface) $\Gamma$ which is \textit{universal} for $\bm{\Gamma}\restriction \mathcal{X}$, the class of subsets of $\mathcal{X}$ in $\bm{\Gamma}$.
    \par In particular, $P \subseteq \mathcal{X}$ is in $\bm{\Gamma}$ iff $P \in \Gamma(\epsilon)$ for some $\epsilon \in \mathcal{N}$, i.e. iff
    \[ P(x) \iff P^*(\epsilon,x) \]
    for some $P^* \in \Gamma$.
\end{theorem}
\begin{proof}
	For $\Sigma_1^0$, take 
	\[ G(\epsilon,x) \iff \exists n[x \in N(\mathcal{X},\epsilon(n))] \]
	Note that the relation in brackets is just the basic nbhd relation, which we already know to be recursive, and so $G$ is semirecursive in $\mathcal{N} \times \mathcal{X}$, i.e. $G \in \Sigma^0_1$. Now for any open set $U$ in any topological space $\mathcal{X}$, $U$ must be equal to a union of basic nbhds, so there exists a sequence $\epsilon \in \mathcal{N}$ which enumerates these neighborhoods, i.e. $P = \bigcup_{n \in \omega} N(\mathcal{X},\epsilon(n))$. But then
	\begin{align}
		x \in P &\iff \exists n [x \in N(\mathcal{X},\epsilon(n))] \\
				&\iff (\epsilon,x) \in G \\
				&\iff x \in \{y \in \mathcal{X}: G(y,\epsilon)\} = G_{\epsilon}
	\end{align}
	I.e. $G$ is universal for $\bm{\Sigma}^0_1 \restriction \mathcal{X}$. Note that this effectively means that a set $P$ is open if and only if, given the sequence enumerating the basic open sets of $P$, the task of confirming if a point is in $P$ is recursively enumerable - simply perform a dovetail search through all of the open sets, repeatedly calling the recursive function which decides the basic nbhd relation. Note also that immediately we also have related results for all the other pointclasses: A  set is clopen iff the same task is not just recursively enumerable but recursive. A set is closed iff the problem is \textbf{coRE}, and so forth. \textit{Topological complexity of sets in a topological space coincides with the computational complexity of deciding membership in a set.}
\end{proof}
So for the special case of, say, $\Sigma^0_1 \restriction \mathcal{C}$, the claim is that $P$ is open in Cantor space iff there exists a sequence of integers $\{x_n\}$ and a partial recursive $P^* \in \Sigma^0_1$ such that $P(x) \iff P^*(\{x_n\},x)$. Recall this in turn means that there is a recursively enumerable sequence of indices $\{i\}_n$ such that $P^*(\{x_n\},x) \iff \exists k ((\{x_n\},x) \in N(\mathcal{N} \times \omega,i_k)$
\\

\par Fix a finite alphabet $\Sigma$, and enumerate the finite strings over that alphabet lexicographically. Let $N: \omega \to \Sigma^{\#}$ Then any decision problem $L$ can be viewed as an element of Cantor Space $L \in \mathcal{C}$, in the sense that for any string $x$, $x \in L \iff L(N^{-1}(x)) = 1$. I.e. the sequence of ones and zeroes codes whether or not the $n^{th}$ string is or isn't in the language. Viewed more simply, we are replacing the \textit{set} $L$ with it's characteristic function $\chi_L$, and noting that there is no loss of information in doing so. $L$ might as well \textit{be} it's characteristic function. Define the set
\[ \mathcal{O} := \{L: \textbf{P}^L \neq \textbf{NP}^{L}\} \]
Then this set exists somewhere in the Borel hierarchy. We wish to identify where exactly it resides. To this end, let $M_i$ be an admissible numbering of all Turing machines, (via partial recursive codings of the transition functions), and let $N_i$ be the more general numbering of all nondeterministic Turing machines. We wish to look at a few predicates, one at a time. First, the predicate 
\[ P(i,n,t) \iff \textrm{$M_i(x)$ halts in time $t$} \]
is clearly recursive. 


\end{document}