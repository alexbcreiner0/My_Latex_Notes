\documentclass{article}
\usepackage[utf8]{inputenc}
\usepackage[margin=1in]{geometry}
\usepackage{amsmath}
\usepackage{amsthm}
% Package for making turing machine diagrams %
\usepackage{tikz}
\usetikzlibrary{chains,fit,shapes}
% Packages for algorithms %
\usepackage{algorithm}
\usepackage{algorithmic}
% Package which has the nice looking empty set symbol (\varnothing)
\usepackage{amssymb}
% Package with the ceiling function
%\usepackage{mathtools}
%\DeclarePairedDelimiter{\ceil}{\lceil}{\rceil}
\usepackage{bm}
\usepackage{unitsdef}
\usepackage[makeroom]{cancel}
\title{Value Theory Thoughts/Notes}
\author{Alex Creiner}

\theoremstyle{definition}
\newtheorem{definition}{Definition}[section]
\newtheorem{problem}{Problem}

\theoremstyle{plain}
\newtheorem{example}{Example}[section]

\theoremstyle{theorem}
\newtheorem{fact}{Fact}[section]
\newtheorem{lemma}{Lemma}[section]
\newtheorem{theorem}{Theorem}[section]
\newtheorem{corollary}{Corollary}[section]
\newtheorem{claim}{Claim}[section]
\newtheorem{conjecture}{Conjecture}[section]


\begin{document}
\maketitle
\section{The Discrete Time Stochastic Model (Farjoun and Machover)}
The main takeaway from Marx's metaphysical reflections and explorations of the three volumes of Capital are, in my opinion, that capital is in a \emph{real social fluid}. 
Just like a real \emph{material} fluid, it flows like in a river, or maybe even a circuit. The flows of capital intersect and criss cross. At some locations, the flow is wider and more intense, at others it's only a trickle. Just like a river, society subsists off of the flow of capital. In order to obtain the necessities of daily life, you have to either control part of the flow, or work against it to siphon off a small amount of it. Siphoning off fluid requires the application of labor - not exactly work in the physics sense, but rather socially necessary human labor. In any case, physical work is involved, and the aggregated physical labor of the working population acts as a massive motor force which is responsible for animating the flow in the first place. This bears repeating: In working against the flow to siphon off their subsistence, workers are \emph{producing} and perpetuating that flow in the first place. It is in a fetishistic reversal that we mistake the flow of capital as the flow of these sheets of paper called money, or more directly as something radically disconnected from social labor, something which is mystically bestowed it's power over us as individuals by natural law. This is much as the peasants who gave the king of a feudal regime his social power were eventually convinced that the king's power came from god. \par 
Just as a flow of water, the fluid of capital is understood to be in actuality an aggregate collection of moving particles. Just like an ideal gas cloud, the entity of the cloud, and it's observable dynamics, are the statistical results of the particles interactions. Crucial takeaways result from this conception:
\begin{itemize}
	\item[(1)] To effectively understand the basic mechanics of a capitalist system is to understand the basic statistical dynamics of the flow of capital.
	\item[(2)] The study of the flow of capital will heavily resemble the study of fluids and gas clouds, and so it will resemble the study of statistical mechanics.
\end{itemize}
\newunit{\auw}{auw}
\newunit{\aup}{aup}
\newunit{\dollars}{d}
I will assume for simplicity that our capitalist economy has a single currency, that is, a single medium of exchange which is traded for commodities. I will call this currency dollars, with units denoted by $d$. Our model economy will not be closed in the sense of having a fixed selection of commodities, as many economic models seem to be. However, the number of \textbf{firms} within our economy will be closed. Define the \textbf{firm space} for a model capitalist economy to be the set
	 \[ \mathcal{F} = \{ f_1,f_2,...,f_{n_F} \} \]
I wish to define in an objective way the amount of \textbf{capital} employed by a firm at a particular time $t$. I want to be very careful about this. Fix an interval of time, $T$. During this time, a firm will make use of a definite number of commodities as inputs to it's production (one of which is always human labour-power). This can be seen as a basket of goods, and this basket of goods itself, \textit{not it's dollar price or any other quantitative measure of these goods}, is the capital of a firm. That said, I want to view the capital of a firm as a number, and to do this, a numeraire must be specified. The simple thing to do would be to just use dollars. And this is what we will do, but this requires a disconnection in some sense from the objectivity of the situation, because while the inputs are a definite set of physical objects, their prices are the result of transactions occurring on the free market, on which anyone can sell anything at any price. In other words, prices are the product of a chaotic system of social decision making, and so to attach the numeraire of price to the commodities one must acknowledge the way in which our physical system of production is directly engaging with and reacting to a separate system of exchange, one in which endlessly complex social processes are taking place which can only be seen in purely physical terms as stochastic processes - processes \textit{which can only be observed statistically in aggregate}. Thus the price of a commodity must be seen as a random variable, however narrow or nearly degenerate that random variable might be. The upshot of this is that the \textit{physical collection of capital goods} is definite, but the \textit{value of that capital itself is the product of randomness}. 
\par 
However, the firm in question must purchase those materials that it employs in production as commodities, and despite their prices being random, that firm, in order to be using them, must at least have agreed to pay a definite price for those commodities. This agreed price of the commodities used during the interval $T$ (some of which will be partial uses, but this is fine, simply take the appropriate fraction of the price in those cases) is what we will define as the \textbf{capital} of the $i^{th}$ firm, which we will denote $K(f_i,t)$. (Where $t$ might as well be arbitrary or can be assumed known without specifying it, we will simply write  $K(f_i)$.) \par 
Note that since the number of firms is assumed finite, the \textbf{total social capital} $\mathcal{K}$ circulating in society is itself finite (at any specific moment in time). We define it as
\[ \mathbb{K} = \sum_{i =1}^{n_F}K(f_i) \]
We also use $\mathbb{K}$ do denote the total capital as a physical collection of goods, and distinguish this from the quantity $\mathcal{K}$. Using this capital function, a probability measure can be defined on firm space (one which changes with time). Simply define the probability of "drawing" a specific firm (or if one would rather, the \textbf{weight} of the $i^{th}$ firm) to be the proportion of capital in that firm versus the whole economy, i.e. 
\[ p(f_i) = \frac{K(f_i)}{\mathcal{K}}K(f_i) \]
The capital function itself then becomes a random variable on firm space, in the sense that
\[ P(k) = \sum_{i \in K^{-1}(k)}p(f_i) \]
Thus the probability of seeing a specific level of capital is precisely the proportion of capital which that bracket occupies out of the total social capital. The dynamics of a capitalist economy are precisely the dynamics of the distribution of capital. \par 
With these definitions, we have the following: The particles in our system are not firms, but capitals. What we have is essentially an ideal gas cloud, that is, a collection of identical particles in space. The space is a one dimensional line, made out of the firms (ordered arbitrarily), and the particles are the \textit{individual} capitals. Each capital is it's own entity. It's dynamics are the dynamics of capital, and the dynamics of the cloud are the dynamics of capitalism as a total system. This is not a perfect analogy. Capitals are created moment to moment by the prices agreed upon, and so the total social capital is slowly changing over time - particles are winking in and out of existence. Additionally, the total social capital need not be a whole number - single "unit" capitals might split in half and move in different directions. Nonetheless, the analogy will be extremely useful to keep in mind. \par 
Continuing on the gas cloud analogy, while the capitals might be the same, they by no means have a uniform distribution in space, that is, among the firms. At any moment in time, $70$ percent of all $\mathbb{K}_G$ many particles might be occupying the same location (i.e. one single firm might be dominating production to the point where they are employing seventy percent of all of the capital). In the situation that individual particles have the same mass, the location where population is most dense is the location that will have the most effect on gravity. The same would be true of the distribution of charge in an electrodynamical system, and the same is true of capital in a capitalist system. The dynamics of the economy revolve and are dictated by the firms operating with the most operating capital, and defining a probability measure via the distribution of capital means that all future random variables defined will be contingent on this distribution of capital, being influenced by firms proportionally based on the proportion of capital within the firm itself. It should be emphasized that there is no causal claim being made here. This is a choice, not a guess, of what quantities are believed to be central in the sense that we are concerned how the capitalist economy relates to these central quantities. In the case of capitalism and the dynamics of capital, this choice is obvious. As the reader will see shortly, it won't always be that way.
 \par 
Central to our human economy is the labour which participates in it. During the period $T$ specified above, there is a portion of the physical goods which are purchased by the workers in exchange for bills they receive from their employer. Note here I am stressing the \textit{physical goods} as the wages, not the actual money, just as I did with capital. For now it suffices and is crucial to view the wages as a definite collection of physical goods and services, unquantified. (The remainder of the goods produced are the \textbf{surplus}, and the quantification of this surplus will be profit.) \par 
From the perspective of the capitalist, human labour power is purchased like any other commodity. However, it is a very peculiar commodity in the following sense. Suppose I'm a capitalist, and I want to enter the cardboard box making industry. How would I go about this? I would ask around how \textit{cardboard boxes are already made}. That is to say, I would look at how my competitors are doing it, and either copy them, or copy them and add on some very small innovation to what they were doing. Either way, we can conclude that commodities generally have a "standard method of production". There is at any moment in time $t$, a standard set of procedures, machines, quantities of inputs, style of work being employed, etcetera, which is remarkably stable and changing usually very slowly. Perhaps more realistically there are two or three methods, depending on whether we're talking about a global system or a national system, but that is manageable by taking averages (we will return to this in a bit). \par 
This is the case for all commodities \textit{except}, arguably, for labour power, since people reproduce themselves \textit{outside} of the capitalist economy. The reproduction of labour power \textit{does} require labour, but from the perspective of capital this labour is not productive of profit, because if it were, then that would mean there is a capitalist firm selling the product of this reproductive labour for profit, and slavery is illegal. Thus it must be acknowledged right away that the system of humans is actually two interconnected systems of labour organization. In the \textbf{domestic economy}, humans take care of themselves and perhaps a few of their family members, reproducing their labour power. From the perspective of a profit seeking \textbf{capitalist economy}, the domestic economy is simply the non-profit industry in which humans reproduce units of labour power for sale on the market as inputs to production.  \par 
I say arguably because Marx himself would disagree. In Capital, Marx defines precisely this quantity, which he calls the \textbf{means of subsistence} - the ideal price of reproduction of labour power, determined by society's average daily (or weekly or whatever the period in question is) consumed goods. Just as a realistic model must swear off from ideal prices, we should do the same here and reject an ideal wage. Nonetheless we will denote this proposed basket of goods $\mathbb{V}_0$, for later comparison with Marx's own model. 
Despite not having a standard method of production, the production of labour power, like all other commodities, \textit{does} consume, for each period of time $T$, a fixed collection of commodities, which we will denote $\mathbb{V}$, and that collection of goods is precisely what I defined earlier as wages. These wages are paid by the firm, and must be equal to a portion of the capital. In the spirit of Marx, we will call this the \textbf{variable capital} of the $i^{th}$ firm. The remaining basket of non-labour goods used in production during the period $T$ is denoted $\mathbb{I}$, and is called the \textbf{constant capital}. Thus, in a loose sense (these are physical collections of goods, not numbers to be added) we can say 
\[ \mathbb{K} = \mathbb{V}+\mathbb{I} \]
There is one more physical collection of commodities which we should define and give a special symbol for - the commodities which are actually produced by the capitalist economy after the period $T$ has ended. This we denote $\mathbb{P}$. \par 
Before turning to the rate of profit, we should introduce two more spaces. For the same fixed period $T$, there are a definite number of transactions which occur, in which money is exchanged for commodity. We make no reference or restriction to commodity type. If I pay seventy dollars my groceries at he store, then that is a single transaction, and the bundle of groceries I leave the store with are a single commodity. We denote \textbf{market space} as an indexing of these transactions:
\[ \mathcal{M} = \{c_1,c_2,...,c_{n_M}\} \]
The only transactions left out of this set are those in which the commodity sold is labour power. For this, we fix a unit of measurement of labour, usually as the period length $T$, but for the sake of conceptual simplicity one can think in terms of hours. The reason $T$ makes for a good unit is that it means that the total amount of labour sold is always precisely the number of workers in the capitalist economy (assuming each worker sells the same amount of labour, which is quite a strong claim to make, but even if it is off, it will still be the case that it is a close approximation.) Then for the period $T$, the number of units of labour power sold is definite, and we index these units in \textbf{labour power space}:
\[ \mathcal{L} = \{l_1,l_2,...,l_{n_L}\} \]
We will also denote $N$ as the total number of workers in he economy, and subject to choice of units of time, in the case of unit labour time $T$, we will often substitute $N$ for $n_L$. The probability measure on labour power space will simply be the uniform measure:
\[ p(l_i) = \frac{1}{n_L} \]
Let $S$ denote the total number of dollars paid for all of this labour power. From $S$ as well as the total labour done in the period $T$, we will define a unit price from which we will measure prices in general. Define the \textbf{average unit wage} to be the quantity:
\[ 1 \auw = \frac{S}{n_L} \dollars \]
We will in general measure prices in units of $\auw$, rather than simply dollars. If, on average, it costs $16$ dollars to purchase one worker hour, and I pay $2$ dollars for coffee, then the price of the coffee is $.125\auw$. That is to say, what I paid for a cup of coffee is an eighth of what it would cost, on average, to purchase an hour of labour (or a unit of labour, more generally). \par 
With $\auw$ defined, we can begin defining the main random variables in question. First of all, for each unit of labour power sold in the economy, $l_i$, a particular wage is paid, in dollars. This wage, expressed in $\auw$, is the \textbf{wage} paid for the $l_i^{th}$ unit of labour, and is denoted $W(l_i)$. Thus the wage is a random variable on labour-power space. Note that because we are measuring in $\auw$, that it is always the case that $E(W) = 1$.
\begin{proof}
	Enumerate each particular wage amount $w_1,w_2,...,w_k$. Then
	\begin{align}
		E(W) = \sum_{i=1}^k w_i p(W = w_i) &= \sum_{i=1}^k w_i \sum_{j \in W^{-1}(w_i)}p(l_j) \\
			&= \sum_{i=1}^k w_i \sum_{j \in W^{-1}(w_i)}\frac{1}{n_L} \\
			&= \frac{1}{n_L}\sum_{i=1}^k w_i |W^{-1}(w_i)| \\
			&= \frac{1}{n_L}\sum_{i=1}^{n_L}w_i \\
			&= \frac{S}{n_L} = 1 \auw
	\end{align}  
\end{proof}
Now, let us turn back to market space. We still need to define the probability measure for this space. Just as with firm space, we want to view market space as a distribution not of the transactions themselves, but of more indivisible units which just happen to occupy certain transactions, as a particle would occupy a location. Here, though, the choice of how to do this is not obvious, but somewhat arbitrary. I'm going to give the definitions first, and then talk about them once the important random variables to the discussion have been laid out. \par 
The individual equally weighted particles in market space will be individual units of labour. Thus, transactions will be weighted proportionally to how many units of labour went into the physical objects being traded. To be clear, I've already mentioned that it is reasonable to assume a standard method of producing any commodity other than labour power in a capitalist economy (the claim would \textit{not} be true necessarily for modes of production other than capitalism). Given that there is a standard method, there is also a standard basket of commodities which is employed in the creation of whatever is defined as a "unit" of the commodity in question (the choice of which is arbitrary). 
\\
\\
elaborate on this
\\
\\
Let $\lambda(c_i)$ denote the \textbf{labour content} of the commodity $c_i$. Note that since the set of transactions is finite, the total collection of commodities is itself a definite basket, which I'll denote $\mathbb{C}$. Note that $\mathbb{C}$ is a very similar basket to the basket of all commodities produced $\mathbb{P}$, but not necessarily identical. Some commodities produced will not be sold until later, and some commodities sold were produced during an earlier period. Nonetheless, they will usually be very similar to each other, so for the record I'll write:
\[ \mathbb{C} \approx \mathbb{P} \]
Baskets of commodities have a total labour content as well. I'll employ some polymorphism and write $\lambda(\mathbb{C})$ to denote the total labour content of all goods and services sold during the period $T$ (although since the commodities involved in the transactions might be collections of things bought at the same time, the distinction is mostly meaningless). From this, we can define a probability measure on market space. The probability of "drawing" a particular transaction will be simply the proportion of labour which that commodity being traded for represents relative to the total labour being traded:
\[ P(c_i) = \frac{\lambda(c_i)}{\lambda(\mathbb{C})} \]
At this point, the labour content of a transaction becomes a random variable, $\Lambda$, but a fundamental one in the sense that I am now choosing a specific perspective on the capitalist economy. By defining the probability measure on market space in terms of the labour content of commodities traded, I am now choosing to specifically concern myself with how the capitalist system interacts with the distribution of labour in producing the commodities being traded. The model is similar to the model for firm space: particles in firm space are individual, equally weighted "capitals", while particles in market space are individual, equally weighted "units of labour-power". We are now discussing a system in which the central concern is the dynamics of markets and capitalism \textit{relative} to the distribution of labour. The degree to which this choice is arbitrary will be discussed shortly. \par 
Before that discussion however, we should lay out the whole model relative to this labour basis. The random variable $\Pi$ will denote the actual price of a commodity in drawn in market space weighted according to labour proportion, measured in units of $\auw$. If $c_i$  indexes my purchase of a coffee from earlier, then $\Pi$ is not equal to $2$ dollars, but rather $.125 \auw$. I'll also write $\Pi = \pi(c_i) = 0.125$, i.e. use lower case $\pi$'s to denote particular values that the random variable $\Pi$ takes on. \par 
Finally, we denote the central random variable of interest pertaining to market space and labour-power space, the \textbf{specific price} $\Psi$ of a commodity. This is defined
\[ \Psi = \frac{\Pi}{\Lambda} \]
Thus the specific price of a commodity is the arrived at price of the commodity as a number of units of labour power which could be purchased at that same price (on average) versus the actual number of units of labour power which are actually needed in order to create the commodity being traded. \par 
This is an extremely provocative quantity. Suppose I pay for a commodity $c_i$, and that $\Psi < 1$. What does this mean? It means that, relative to the labour which went into creating this commodity, I struck a bargain; the amount of money I spent on $c_i$ would generally not be sufficient to the amount of labour I would need to purchase in order to produce the commodity myself. Conversely, if $\Psi > 1$, then from the standpoint of labour, I've been ripped off; I paid enough that the person who sold $c_i$, provided that they have the other necessary materials, could buy enough labour power to reproduce $c_i$ \textit{plus} some extra surplus. \par 
This means that our model has been specifically built in order to evaluate capitalist profit from the standpoint of a particular type of profit - profit of labour time. $\Psi > 1$ might mean that the merchant has profited in terms of labour, but they might have been ripped off in terms of, say, oil. To fully understand this framing, let's go through what would change to reframe our model in terms of oil rather than labour. \par 
First, instead of labour power space, we would instead define \textit{oil space}, 
\[ \mathcal{O} = \{o_1,o_2,...,o_{n_O} \]
in which we've specified a unit of oil, say perhaps gallons, and declared $\mathcal{O}$ to be an indexing of all gallons of oil sold during a fixed interval of time $T$. We can then declare a unit of measurement for price in terms of gallons of oil. Suppose the total price paid for oil is $S$, then we can define the average unit price by
\[ 1 \aup = \frac{S}{n_O} \]
We can then declare the oil price random variable $G$ to be the price of a particular oil unit measured in $\aup$, from which it can be shown that $E(G) = 1$, just as before for wages. We can then relate this denomination back to market space and define prices to be measured in units of $\aup$, that is, measuring in dollars, but not so much \textit{in} dollars, so much as in the number of gallons of oil that a certain dollar amount could purchase, on average. We can also measure the oil content of any commodities in the same sense that we measured the labour content, and denote it, say $O$, and then define specific price as the variable which is price (in gallons of oil purchasable by a given price) relative to the average cost of a gallon of oil. From this, we have built up, seemingly, the exact same model, except now we are concerned with the dynamics of capitalism relative to oil, as opposed to labour. There is a problem, however. \par 
The problem is what I've already mentioned: oil is produced as a commodity, whereas labour power is not. When we were discussing market space in relation to labour power, oil was presumably a commodity like any other, whose purchases and sales were indexed in market space. Thus oil was not \textit{absent} from the system, it just wasn't singled out as peculiar. The question we must ask is this: can the same be said of labour power in our new oilcentric system? Remember that the central assumption to this model is the definite existence of a meaningful oil-content of any commodity bought or sold. Well, what is the oil content of a unit of labour-power? Such a number must be assumed in order to render labour power sales in market space, but as we noted before, labour power is \textit{not} produced systematically in the same sense that other commodities are. People live wildly different lives from each other, there is no "mostly standardized method of existence" in the same sense that there is a "mostly standardized method of producing pencil lead". Such a number can be assumed (and Marx \textit{did} assume just such a thing), and in some sense it \textit{does} exist, in an average sense. But it requires it's own extremely dubious simplification of reality, a complete equivocation between the domestic economy of people, and the capitalist economy of commodity production. \textit{The only way to incorporate labour power into an oilcentric model of capitalism is to cover up and ignore the peculiarity of labour power.} \par 
This is not to say that an oil-centric model of capitalism can't be useful in it's own right. It has things to say, valuable things even. But the only way to evaluate capitalism in it's interaction with the humans labouring and benefiting from the system is to distill labour out of it. In order to understand capitalism \textit{relative to the human society from which it spawned}, is to single labour-power out as a special and fundamental substance which the dynamics of capitalism play out \textit{in relation to}. \par 
Thus, I will proceed to make claims about relations between prices and labour contents, and claims about capitalism's dynamics relative to labour, and those claims should be subject to the scrutiny of empirical scientific testing and verification. But the "labour theory of value" will not be \textit{proven} scientifically, nor should it be. It is not a scientific question, but a philosophical one. This model itself has a hidden universal quantifier behind every claim it makes, and this is something that I want to pay special and careful attention to as I continue laying the model out. \par 
With that said, let's return to market space. I defined the capital of a firm earlier, but at that point I did so in terms of dollar value. Now, I am considering a system "relativized" to labour, and as such I want to measure capital in units of $\auw$. I want to next turn to the rate of profit. Within the period of time $T$, a firm $f_i$ will produce some physical output of commodities $\mathbb{P}_i$, employing a total physical capital $\mathbb{K}_i$, which can be divided into wages and materials $\mathbb{V}_i$ and $\mathbb{I}_i$ respectively. The firm sells it's produce at a price $\pi(\mathbb{P}_i)$, constituting the revenue, and, as anyone who takes a business class will tell you, it's profit is the difference between revenue and cost, which is of course $\pi(\mathbb{I}_i) + \pi(\mathbb{V}_i)$. Now, for the purposes of rigour, it must be emphasized that these values of $\pi$ are not necessarily the outputs of the same function $\pi$ defined on market space. Remember that the random variable $\Pi$, and it's specific cases $\pi$, pertain specifically to commodities bought and sold during the fixed interval $T$, while the some of the commodities produced will likely be sold at a future interval, and some of the materials and labour power purchased were purchased prior to the interval beginning. Nonetheless, from the perspective of the firm, the profit rate is obviously the ratio of profits relative to the capital invested:
\[ r_i = \frac{\pi(\mathbb{P}_i) - \pi(\mathbb{I}_i) - \pi(\mathbb{V}_i)}{\pi(\mathbb{K}_i)} \]   
It should immediately be noted that $\pi(\mathbb{K}_i)$ is merely a fancy new way to express what I defined in the beginning, the capital of the $i^{th}$ firm $K(f_i)$. Or is it? The ambiguity here points to an ambiguity which was never resolved when I defined this random variable initially, which is that while the capital goods used in a period of production are definite, the numerical price of these goods is slippery. The simplest thing to do, and what we implicitly did back then, was to simply define $K(f_i)$ to be the number of dollars (expressed in $\auw$) spent by the firm whenever they bought the constituent goods. This is the only reasonable thing to do, and what I will do for the prices of the other bundles in this equation $\pi(\mathbb{P}_i), \pi(\mathbb{I}_i)$, and $\pi(\mathbb{V}_i)$. In any case, we've now defined the rate of profit $R$ of the $i^{th}$ firm, as a random variable on firm space. We've also defined the constant and variable capital $I$ and $V$ as random variables, via $I(f_i) = \pi(\mathbb{I}_i)$ and $V(f_i) = \pi(\mathbb{V}_i)$, as well as the price of produced goods $Pr(f_i) = \pi(\mathbb{P}_i)$. We also have the identity that 
\[ K = I+V \]
And we can now reexpress the rate of profit as a random variable defined in terms of these:
\[ R = \frac{Pr-I-V}{K} \]
Before moving on, note that while the numerical value of $K(f_i)$ will necessarily change when we go from measuring in dollars to measuring in $\auw$, the profit rate will remain the same regardless of the commodity we are relativizing our model to. The reason for this is that while the capital of a firm has a physical unit associated with it (namely either dollars $\dollars$ or average unit wages $\auw$), these units cancel out in the fraction defining the rate of profit, rendering the rate of profit unitless. Moreover the conversion from $\dollars$ to $\auw$ would equate to multiplying each term by the unitless factor $\frac{n_L}{S}$, all of which cancel out immediately just as the units do. Thus \textit{profit rates remain the same regardless of the commodity chosen to act as numeraire.} 
Suppose that for a period $T$, beginning at time $t_0$, a firm has the profit rate $R(f_i) = r_i$, and similarly a constant, variable, and working capital $j_i,v_i$, and $k_i$ respectively. Suppose the period $T$ is relatively small. In this case, the technical details of production, habits of consumption, and state of the class struggle (i.e. levels of wages) are not likely to change much. As a result, the prices in the equation for rate of profit are not likely to change much, and the rate of profit can reasonably be expected to stay approximately the same in the \textit{next} period, that is, from time $t$ to time $t+T$. Assuming that the profit of the previous period is reinvested as capital in the next period, we have that \begin{align}
	K(f_i,t_0+T) &= K(f_i,t_0)+Pr(f_i,t_0)-I(f_i,t_0)-V(f_i,t_0) \\
			&= K(f_i,t_0) + K(f_i,t_0)R(f_i,t_0) \\
			&= K(f_i,t_0)(1+R(f_i,t_0))
\end{align}
It's plain to see from this that assuming a decision to reinvest profits results in the rate of profit becoming an exponential growth factor for capital, but let's dig a bit deeper. The total change in capital then becomes:
\begin{align}
	\Delta K = K(f_i,t_0+T)-K(f_i,t_0) &= K(f_i,t_0)(1+R(f_i,t_0))-K(f_i,t_0) \\
	&= K(f_i,t_0)((1+R(f_i,t_0))-1) \\
	&= K(f_i,t_0)R(f_i,t_0)
\end{align}
If we fix $T=1$, i.e. we view the period $T$ as our fundamental unit of time (perhaps weeks, or years), then this change in $K$ can be seen as average slope of the capital as a function of time. For small $h$ then, we can approximate the capital as a function of time as a straight line, i.e. we assume that for general $t$ in a small neighborhood of $t_0$, 
\[ K(f_i,t) \approx m(t-t_0)+b \]
where $m = \Delta K = \frac{K(f_i,t_0+T)-K(f_i,t_0)}{T}$ and $b = K(f_i,t_0)$, the capital at time $t_0$. By substituting $K(f_i,t_0)R(f_i,t_0)$ for $\Delta K$, we have that for $t$ nearby $t_0$.
\begin{align}
 	K(f_i,t) &\approx \Delta K(t-t_0)+K(f_i,t_0) \\
 			&= K(f_i,t_0)R(f_i,t_0)(t-t_0) + K(f_i,t_0)  
\end{align}
Thus for a small $h>0$, $t_0+h$ will be very near $t_0$, and we can write
\begin{align}
	K(f_i,t+h) &\approx K(f_i,t_0)R(f_i,t_0)(t_0+h-t_0) + K(f_i,t_0) \\
	&\implies K(f_i,t_0+h)-K(f_i,t_0) \approx K(f_i,t_0)R(f_i,t_0)h \\
	&\implies \frac{K(f_i,t_0+h)-K(f_i,t_0)}{h} \approx K(f_i,t_0)R(f_i,t_0)
\end{align}
But the left-hand side clearly approaches the derivative of $K(f_i)$ with respect to $t$ as $h$ approaches $0$, while the right-hand side is constant. We've thus isolated the relationship between the capital of a firm $f_i$ and the rate of profit of that firm. If we assume that the capitalist firms are constantly reinvesting their profits into production after each period $T$, then we obtain a discrete process which, when "smoothed out", yields curves for $K(f_i,t)$ and $R(f_i,t)$ as functions of time $t$ which are deterministically related by the differential equation
\[ \frac{dK(f_i,t)}{dt} = R(f_i,t)K(f_i,t) \]
If we assume the rate of profit to be a constant with time, simply $R(f_i)$, then the solution to this as an initial value problem is 
\[ K(f_i,t) = K(f_i,t_0)e^{R(f_i)t} \] 
The "rate of profit" then, as capitalists and economists tend to think about it, is not as simple as simply the rate of change of the capital over time. It is in fact a constant of proportionality to an assumed exponential growth in capital over time. Such an exponential relationship would remain true in principle if the capitalist only reinvested a certain portion of their profits after each period. \par 
Note that this exponential relationship between the rate of profit and capital is not definite, in two senses. In one sense, it is contingent on the assumption that a capitalist will reinvest a portion of profits to expanding production. In another sense, our model is not assumed to be a smooth or continuous process, but is instead a stochastic process which updates after specified intervals of time. In other words, this discussion is only for the sake of intuition on how the rate of profit and capital are going to interact within a normally functioning capitalist system. \par 
Before moving on to some of the other random variables we wish to define on firm space, let's consider a basic fact about the expected value of the rate of profit, one which will in some sense vindicate our choice of defining proportions of capital as the probability measure. Typically, the expected value of a random variable $X$ on a probability space is defined by summing or integrating along the range of the random variable $X$. That is, if $Range(X) = \{x_1,x_2,...,x_m\}$, and the probability space is $\Omega = \{1,2,...,n\}$, then
\begin{align}
	E(X) = \sum_{i=1}^m x_i P(X = x_i) &= \sum_{i=1}^m x_i \sum_{j: X(j)=x_i} P(j) \\
	&= \sum_{i=1}^m \sum_{j: X(j)=x_i} x_iP(j) \\
	&=  \sum_{i=1}^m \sum_{j: X(j)=x_i} X(j)P(j) \\
	&= \sum_{i=1}^n X(i)P(i)
\end{align} 
Thus, in the case of a finite sample space, summing over the range of $X$ the numbers obtained from taking each element and multiplying by the probabilities of realizing that element is the same as simply summing directly over the sample space the numbers obtained from the random variable output times the probability of that particular outcome. Using this identity for expectation, we can obtain a striking expression of the expected rate of profit:
\begin{align}
	E(R) &= \sum_{i=1}^{n_f} R(i)P(i) \\
		&= \sum_{i=1}^{n_f} \frac{Pr(i)-I(i)-V(i)}{K(i)}\frac{K(i)}{\sum_{j=1}^{n_f}K(j)} \\
		&= \frac{\sum_{i=1}^{n_f} Pr(i)-I(i)-V(i)}{\sum_{j=1}^{n_f} K(j)} \\
		&= \frac{\sum_{i=1}^{n_f} Pr(i) - \sum_{i=1}^{n_f} I(i) - \sum_{i=1}^{n_f} V(i)}{\sum_{i=1}^{n_f} K(i)} \\
		&= \frac{\pi(\mathbb{P}) - \pi(\mathbb{I}) - \pi(\mathbb{V})}{\pi(\mathbb{K})}
\end{align}
Thus the expected rate of profit is simply the global rate of profit: the total profit across the entire economy divided by the total capital, all expressed in terms of the observed actual prices. This number we will refer to as the \textbf{general rate of profit}, and denote $r_g$ 
\[ r_g = \frac{\pi(\mathbb{P}) - \pi(\mathbb{I}) - \pi(\mathbb{V})}{\pi(\mathbb{K})} = E(R) \]
Next we would like to inspect more closely the division of the total capital $K$ into it's two components $I$ and $V$. First note that our division is one relative to labour. Just like I noted earlier, this choice of labour costs specifically to single out here is technically arbitrary. The choice could have just as easily been to define $V$ to be capital costs spent on oil, rather than labour power. I'll return to this point later. For now, let's inspect how $V$ changes in relation to $K$ and $R$. Define the random variable $Z$ by
\[ Z = \frac{V}{K} \]
That is to say, $Z$ is the proportion of wages relative to the total capital. Towards an understanding of $R$, I supposed earlier that all profits were reinvested as capital. When I did that, I paid no attention to the \textit{composition} of those profits. Perhaps the capitalist uses their profits to invest in new machinery, which requires less labour. In that case, even if all profits return as capital, the proportion $Z$ could very well be smaller than it was before. Let's suppose that the capitalist simply reinvests in the same composition as before, that is to say, that
\[ Z(f_i,t_0+T) = Z(f_i,t_0) \]
i.e.
\[ \frac{V(f_i,t_0+T)}{K(f_i,t_0+T)} = \frac{V(f_i,t_0)}{K(f_i,t_0)} \]
But immediately from this we obtain
\[ V(f_i,t_0+T) = Z(f_i,t_0)K(f_i,t_0+T) \]
Thus, subtracting the original wages $V(f_i,t_0)$ and applying the fact that $V = ZK$ by definition, we get 
\begin{align}
	V(f_i,t_0+T) - V(f_i,t_0) = \Delta V &= Z(f_i,t_0)K(f_i,t_0+T) - V(f_i,t_0) \\
	&= Z(f_i,t_0)K(f_i,t_0+T) - Z(f_i,t_0)K(f_i,t_0) \\
	&= Z(f_i,t_0)\Delta K \\
	&= Z(f_i,t_0)K(f_i,t_0)R(f_i,t_0)
\end{align}
From here the argument proceeds identically to $R$. Considering $T$ as our unit time, and assuming $0<h<T$ and that $T$ is sufficiently small that the curves $V(f_i,t)$, $K(f_i,t)$ and so forth can be approximated as straight lines (and rendered as continuous growth), we have that $\Delta V$ functions as the slope of this line, and obtain 
\[ V(f_i,t) \approx Z(f_i,t_0)K(f_i,t_0)R(f_i,t_0)(t-t_0)+V(f_i,t_0) \]
Plugging in $t_0+h$ for $t$ and subtracting $V(f_i,t_0)$ from both sides and dividing by the resulting $h$ thus gives
\[ \frac{V(f_i,t_0+h) - V(f_i,t_0)}{h} \approx Z(f_i,t_0)K(f_i,t_0)R(f_i,t_0) \]
Taking the limit as $h \to 0$ thus yields the approximate law
\[ \frac{dV(f_i,t)}{dt} = Z(f_i,t)R(f_i,t)K(f_i,t) \]
The truly striking equation however is given when we realize that $R(f_i,t)K(f_i,t)$ is the derivative of $K(f_i,t)$ by a subset of the same assumptions we made to get here:
\[ \frac{dV(f_i,t)}{dt} = Z(f_i,t)\frac{dK(f_i,t)}{dt} \] 
Thus we see that the "natural" behavior of $V$ and $K$, under the assumption that the proportion of wages to constant capital remains unchanging, is for the variable capital to "mimic" the total capital, up to $Z$ a constant of proportionality. \par
Next let's consider the expected value of $Z$, similarly to how we did for $R$:
\begin{align}
	E(Z) &= \sum_{i=1}^{n_f} Z(f_i)P(f_i) \\
		&= \sum_{i=1}^{n_f} \frac{V(f_i)}{K(f_i)} \frac{K(f_i)}{\sum_{j=1}^{n_f} K(f_j)} \\
		&= \frac{\sum_{i=1}^{n_f} V(f_i)}{\sum_{j=1}^{n_f} K(f_j)} \\
		&= \frac{\pi(\mathbb{V})}{\pi(\mathbb{K})} 
\end{align}
Thus the expected value of $Z$ is the global ratio of total wages to total capital in circulation. \par 
I want to stop here and digress a bit to connect these variables to what Marx was doing in capital. Marx's central variables were $S$, the surplus value, $V$, the variable capital, and $C$, the constant capital. In order to make sure all of our objects of interest have unique symbols representing them, I'm going to denote these $S_M$, $C_M$, and $V_M$. These numbers were all values, that is to say, they were conceptually speaking amounts of time. Recall that Marx operates in an idealized world in which there is a definite "average labour content of labour power", which he called the means of subsistence. This number could be determined globally or locally, and relative to any interval of time. Globally, this would be, for a period $T$, approximately equal to the total labour content of all of the commodities produced, i.e. $V_M^g = \lambda(\mathbb{V})$. Such a number certainly has meaning, even in our own model. This number for us would represent something close to the total work that the workers must do each day in order to reproduce their way of life for another period $T$. One can only say something close, not exactly the same, since of course a good portion of the work that must be done was already done at an earlier period, or is work which, while necessary, can be put off to a later date without having any effect on people's material lives. Nonetheless, as a number which fluctuates from period to period, one can easily imagine an average or "ideal" amount. This number obviously \textit{exists} in some sense. What is the problem, then? The problem is that this number, divided by the number of laborers, while representing an average amount time one has to work in order to "pull their weight", but this number likely differs in the extreme from what any particular worker actually consumes for themselves. This is especially true if we are speaking about a global capitalist economy, since the consumer consumption habits of a North American are wildly different from those of a sweatshop worker in Bangladesh. Here, such an average certainly exists, but exists in a "middle" which likely corresponds to an extremely small population even remotely closely. \par 
At the level of a firm, the number $\lambda(\mathbb{V}_i)$ would represent something similar. It would be the number of units of labour power the workers \textit{specifically at that firm} must perform to replicate \textit{their} way of life. Our $V$, in contrast, while similar as a quantity to Marx's refers to the actual prices paid for these commodities - $V_M = \pi(\mathbb{V}_i)$. Similar for $C$: to connect Marx's model to ours would be to say $C_M = \lambda(\mathbb{I}_i)$, while in contrast we have the variable $I = \pi(\mathbb{I}_i)$. Finally, let's consider Marx's surplus value $S_M$. Globally, it is the total labour value pocketed by the capitalist class which is in excess of the labour required for the workers to reproduce their way of life. It's tempting to just conclude that this labour content is embodied in the value $\lambda(\mathbb{P})$, that is, the total labour content of the bundle of all commodities produced during the period $T$. But this ignores the fact that much of this bundle is the product of both living \textit{and} dead labour. We cannot forget that this collection $\mathbb{P}$ is the product of living labour being added onto the raw materials $\mathbb{I}$, which were by definition produced prior to this period. (If a firm produces a unit of these raw materials, then that unit would no longer be part of the raw materials bundle. Instead the product made from it would appear in the bundle $\mathbb{P}$. To put the raw material in $\mathbb{P}$ \textit{and} $\mathbb{I}$ would make the identity claimed following this parenthesis invalid. So this is really a clarification that should have been provided earlier.) Thus the total labour time produced during the period $T$ is $\lambda(\mathbb{P})-\lambda(\mathbb{I})$. From this, the connection to our model would be that 
\[ S_M^g = \lambda(\mathbb{P})- \lambda(\mathbb{I}) - \lambda(\mathbb{V}) \]
The same argument would be true at the level of the firm: 
\[S_M(f_i) = \lambda(\mathbb{P}_i)- \lambda(\mathbb{I}_i) - \lambda(\mathbb{V}_i) \]
From these variables, Marx defined three peculiar ratios. These he didn't give consistent letters to, so I'll take my own liberties. First there was the rate of profit:
\[ R_M \equiv \frac{S}{C+V} = \frac{\lambda(\mathbb{P})-\lambda(\mathbb{V})}{\lambda(\mathbb{K})} \]
Or, at the level of the individual firm,
\[ R_M(f_i) \equiv \frac{S}{C+V} = \frac{\lambda(\mathbb{P}_i)-\lambda(\mathbb{I}_i) - \lambda(\mathbb{V}_i)}{\lambda(\mathbb{K})} \]
This looks remarkably similar to our definition of the rate of profit. Marx would claim that they are not just similar, but identical. The reason for this is that Marx made the following claim, the central column of his labour theory of value:
\begin{claim}[Marx's Labour Theory of Value]
	For a capitalist economy in which supply and demand are in constant equillibrium, there exists a constant $\psi_0$ such that for all commodities $C$, 
	\[ \frac{\bar{\pi}(C)}{\lambda(C)} = \psi_0 \]
where $\bar{\pi(C)}$ denotes the \textbf{ideal price} of $C$, that is, the nonrandom price which $C$ would take on, assuming that supply and demand were constantly equal throughout the capitalist economy.
\end{claim}
In other words, Marx claimed that priced would be identical with labour time up to some constant of proportionality converting units of dollars to units of time units, if supply and demand were in equilibrium. The coherence of this claim hinges on the existence of ideal prices, in the sense described above. We make no such claim, but for the sake of comparison let us assume for a moment that these prices did exist, and that our economy observed whatever conditions necessary that prices were ideal and deterministic, i.e. let us assume that $\bar{\pi}(C) = \pi(C)$. Then under the assumption of Marx's claim, his rate of profit is indeed identical to ours, since we can use this identity to substitute $\pi$ for $\lambda$: 
\[ R_M(f_i) \equiv \frac{\lambda(\mathbb{P}_i)-\lambda(\mathbb{I}_i) - \lambda(\mathbb{V}_i)}{\lambda(\mathbb{K})} = \frac{\pi(\mathbb{P}_i)\psi_0-\lambda(\mathbb{I}_i)\psi_0 - \lambda(\mathbb{V}_i)\psi_0}{\lambda(\mathbb{K})\psi_0} = \frac{\pi(\mathbb{P}_i)-\pi(\mathbb{I}_i) - \pi(\mathbb{V}_i)}{\pi(\mathbb{K})} = R(f_i) \]
Now, we don't have such a claim, but we do have a random variable $\Psi = \frac{\Pi}{\Lambda}$. In a little bit, we will obtain a result about $E(\Psi)$ which will allow us to show that, under a few assumptions, 
\[ \frac{\lambda(\mathbb{P}_i)-\lambda(\mathbb{I}_i) - \lambda(\mathbb{V}_i)}{\lambda(\mathbb{K})} \approx E(R) \]
[Need to show]
Marx also defined the \textbf{organic composition of capital} to be the ratio of constant capital to variable capital measured in labour time. Globally, this would mean
\[ O = \frac{C}{V} \equiv \frac{\lambda(\mathbb{I})}{\lambda(\mathbb{V})} \]
To recall Marx in some precision here, recall that Marx really defined three versions of "the composition of capital" back to back to back, late in Volume I. The \textbf{technical composition of capital} is what Marx defined as the actual material goods ratio of materials per worker. This is not a number that can be expressed purely numerically. If one widget machine and four magnets is enough to supply three workers for an hour of work, then the technical composition here is "one widget machine and four magnets per three worker hours". In other words, the technical composition of capital is the actual organizational distribution of capital goods per worker per unit of labour time. Next, Marx defined the \textbf{value composition of capital} as the numerical labour content ratio of the technical composition. That is to say, take the bundle of commodities to accommodate some number of units of labour-power, take the labour contents of both, and express that as a numerical ratio. (Remember, Marx is assuming that labour-power has a definite labour-content just like every other commodity.) Finally Marx defined the organic composition of capital as "the value composition of capital, in so far as it is determined by its technical composition and mirrors the changes of the latter." From this, I am interpreting that he is simply saying to take the value composition obtained from the technical composition, and render it as a function of a now changing with technology and innovation over time. Since our entire model is assumed dynamic, I feel comfortable referring to our number above as the organic composition, and not the technical or the value composition. It is the whole package. We don't have a variable yet which is directly analogous to $O$, but we do have the random variable 
\[ Z = \frac{V}{K} \equiv \frac{V}{C+V} \]
\[ \implies \frac{1}{Z} \equiv \frac{C}{V}+1 = O+1 \]
Thus our variable $Z$, inverted, while not exactly being directly analogous to $O$, is nearly identical to it, to the point where they might as well be the same. Finally, and most importantly, Marx defined the famous \textbf{rate of exploitation}, otherwise known as the \textbf{rate of surplus value}, to be
\[ E_M = \frac{S}{V} \]
Consider this value in relation to a ratio of the two random variables we just finished defining and discussing in detail: $\frac{R}{Z}$. Note that
\[ \frac{R}{Z} = R\frac{1}{Z} \equiv \frac{S}{C+V}\frac{C+V}{V} = \frac{S}{V} = E_M \]
It appears then that the ratio of the rate of profit to the rate of wage-bill will be crucial to examine in order to examine the relationship between this probabilistic model and Marx's own. We'll call this random variable $X$, and call it \textbf{rate of capital versus labour}:
\[ X = \frac{R}{Z} \]
\\
Next, we turn back to specific price $\Psi$. What we want to look at is large aggregate collections of commodity transactions from market space. Let $\mathbb{B}$ be such an aggregate, chosen uniformly at random. (Note by the notation, we are associating transactions in market space with the commodities being traded, a slight abusse in notation which we will do repeatedly.) Consider the random variable $E(\Psi|\mathbb{B})$, that is, the conditional expectation of $\Psi$ given we are looking at the particular subset of market space which is $\mathbb{B}$. Then
\begin{align}
	E(\Psi|\mathbb{B}) &= \sum_{i: c_i \in \mathbb{B}}\Psi(i)P(i|\mathbb{B}) \\
						&= \sum_{i: c_i \in \mathbb{B}} \Psi(i)\frac{P(i)}{P(\mathbb{B})} \\
						&=\frac{\sum_{i: c_i \in \mathbb{B}} \Psi(i)P(i)}{\sum_{i: c_i \in \mathbb{B}}P(i)} \\
						&= \frac{\sum_{i: c_i \in \mathbb{B}} \frac{\Pi(i)}{\Lambda(i)}\frac{\Lambda(i)}{\sum_{j = 1}^{n_m}\Lambda(i)}}{\sum_{i: c_i \in \mathbb{B}}\frac{\Lambda(i)}{\sum_{j = 1}^{n_m}\Lambda(i)}} \\
						&= \frac{\sum_{i: c_i \in \mathbb{B}} \Pi(i)}{\sum_{i: c_i \in \mathbb{B}} \Lambda(i)}
\end{align}
What this means is that if we were to for whatever reason consider a large random bundle of commodities, then the specific price we would expect to see is the ratio of the total price to the total labour content of that bundle. Note that the definition of the probability measure is crucial to arrive at this. On the one hand, in the algebra here, the details of our probability measure on market space are crucial: the probability of a transaction must be measured by it's weight in labour content for this result to hold. On the other hand, however, the transactions drawn \textit{were completely independent of labour content}, in the sense that the bundle $\mathbb{B}$ is not a drawing of transactions weighted by labour content. It is simply a bundle chosen completely at random. The same result would hold if we were relativizing our model to some other central commodity other than labour power, such as oil. It is a property of the specific price, by it's definition, interacting with the distribution of the central commodity, by it's own definition. \par 
The significance of this result is the following: Suppose we were to draw a random sample of specific prices from market space, $\Psi_1,\Psi_2,...,\Psi_n$, where each $\Psi_i$ is independent and identically distributed. Then by the law of large numbers, the sample mean converges in probability to $E(\Psi) = \mu_{\Psi}$
\[ \bar{\Psi}_n = \frac{1}{n}\sum_{i =1}^n \Psi_i \overset{p}{\to} \mu_{\Psi} \]
That is to say, the probability that the sample mean differs from $\mu_{\Psi}$ by any fixed distance at all goes to $0$ as $n \to \infty$. The more larger our sample is, the closer we can expect the average will be to the actual mean of the distribution. Now, drawing a random sample of specific prices is very similar to drawing a random bundle $\mathbb{B}$, and considering the associated specific prices of commodities in that set. However there are two slight differences. The first, less important difference is the fact that in the latter case, we are implicitly sampling without replacement, whereas in the first case we could conceivably sample the same transaction twice. This difference is minor in the sense that presumably market space is extraordinarily large, and the chance of sampling the same commodity twice is thus quite small. However, it could become more of an issue once considering the second difference, namely the fact that in the random sample case, we are presumably sampling transactions weighted by their labour content. If $50$ percent of all of the total labour content of society goes into a single one of these transactions, then that transaction has a very high chance of being sampled twice, which could produce big differences. \par 
Regardless, if we allow our sample $\mathbb{B}$ to grow larger and larger, it will obviously be the case that $E(\Psi|\mathbb{B})$ converges in probability to $E(\Psi)$. The reason for this is that as $\mathbb{B}$ approaches a larger and larger unbiased portion of the entire space $\mathcal{M}$, $P(i|\mathbb{B})$ must converge closer and closer to simply $P(i)$, so that equation (21) becomes identical to $E(\Psi)$. Thus we have the following crucial result:
\begin{fact}
	For a random and unbiased bundle of transactions $\mathbb{B}$, sufficiently large,
	\[ \frac{\sum_{i: c_i \in \mathbb{B}} \Pi(i)}{\sum_{i: c_i \in \mathbb{B}} \Lambda(i)} \approx E(\Psi) \]
This approximation approaches equality as $\mathbb{B}$ approaches the entire space $\mathcal{M}$ in number of elements. The practicality of this fact, i.e. the ability to use this approximation confidently, given a large and reasonably unbiased bundle of commodities, is contingent on 
\begin{center}
	\textbf{Assumption}: No single transaction holds a particularly large share of the total distribution of labour content.  
\end{center}
\end{fact}
This result gives us a statistical version of Marx's key assumption in Capital Volume 1, recall Marx claimed the existence of a $\psi_0$ such that for any commodity $C$, 
	\[ \frac{\bar{\pi}(C)}{\lambda(C)} = \psi_0 \]
Now, our model has no notion of ideal prices, nor can it predict anything similar to this for a particular commodity $C$. However, consider a large and reasonably unbiased \textit{collection} of commodities $\mathbb{B}$. Assume all commodities in the bundle are sold at some point, and let $\pi(\mathbb{B})$ be the total price of these commodities. Also let $\lambda(\mathbb{B})$ denote the total labour content. Now, our model assumes a fixed period of time $T$, such that $\Pi$ and $\Lambda$ only apply to the commodities bought and sold during this period, long enough to be significant but short enough that labor contents of commodities can be assumed fixed and unchanging. There's no guaruntee that everything in the bundle was produced during this period or bought/sold during this period. However, we can at least reasonably \textbf{assume that the products of the bundle were produced, bought and sold during a period of time \textit{nearby} the "current" period, and assuming nothing massive has changed in the economy in the timespan including these extra periods}, it can then be further assumed that all of the commodities in $\mathbb{B}$, whenever they were bought and sold, have similar price distributions to the current market space, and that consumption patterns more or less repeat themselves. Define the set $\mathbb{B}'$ to be the concrete subset of market space consisting of all commodities in $\mathbb{B}$ sold during the present period, along with, for each commodity in $\mathbb{B}$ not yet in $\mathbb{B}'$, the average of all specific prices of transactions for an identical commodity, if at least one exists (if no similar transaction occurs, simply ignore it). Under our assumptions then  
\[ \pi(\mathbb{B}) \approx \Pi(\mathbb{B}') \] 
and
\[ \lambda(\mathbb{B}) \approx \Lambda(\mathbb{B}') \]
But 
\[ \frac{\Pi(\mathbb{B}'}{\Lambda(\mathbb{B}')} \approx E(\Psi) \]
and so we have our analog to Marx:
\begin{align}
	\frac{\pi(\mathbb{B})}{\lambda(\mathbb{B})} \approx E(\Psi)
\end{align}
Compared to Marx's claim, we have replaced any mention of ideal prices with the actual, observed prices, value with our operationalized notion of labour content, and $\psi_0$ with the new constant $E(\Psi)$. Note the mathematical definiteness of this. We made a few assumptions to get here - what were they?
\begin{itemize}
	\item[(1)] No single transaction during a period $T$ embodies a particularly high percentage of the total labour done during that period.
	\item[(2)] For any unbiased and large collection of commodities $\mathbb{B}$, they were all produced, bought and sold during a period nearby the current period (i.e. no significant portion of commodities in the bundle were produced in, say, the 1980s, if the current year is 2000)
	\item[(3)] The distributions of $\Lambda$ and $\Pi$ change relatively slowly over time. 
\end{itemize}
These are all extremely innocent assumptions to make about our own capitalist economy, and indeed any capitalist economy. Assuming this, the "labour theory of value", in the sense described by equation (26), is not even something that requires empirical evidence to confirm. It is simply a logical tautology, and it would hold if we relativized our model to any other commodity measure satisfying assumptions $(1)$ through $(3)$, whether it be oil, watches, or designer clothing. \par 
Now, let's connect this with the rate of profit. Recall that the general rate of profit for a capitalist society is 
\[ r_g = \frac{\pi(\mathbb{P}) - \pi(\mathbb{I}) - \pi(\mathbb{V})}{\pi(\mathbb{K})} = E(R) \]
Where $\mathbb{P}$, $\mathbb{I}$, $\mathbb{V}$, and $\mathbb{K}$ are the total bundles of produced goods, raw materials, real wages, and capital respectively, of the entire economy. Now, each of these is obviously an \textit{enormous} collection of commodities. That each of them is \textit{unbiased} sufficiently to apply our "labour theory of value" approximation to it is not obvious, and warrants a discussion. F and J make an excellent and very convincing argument in the affirmative (though the argument would not necessarily apply in Marx's own time), which I will paraphrase here later if I have the energy. Assuming this, we can approximate $\pi(\mathbb{P})$ with $E(\Psi)\lambda(\mathbb{P})$, and so forth, and replace each term in $r_g$ with it's labour counterpart. Immediately the $E(\Psi)$'s all cancel out, and we obtain 
\[ E(R) = r_g \approx \frac{\lambda(\mathbb{P}) - \lambda(\mathbb{I}) - \lambda(\mathbb{V})}{\lambda(\mathbb{K})} \] 
Thus, the general rate of profit, which is also the mean rate of profit of society, can be assumed to approximately equal the rate of profit, can be expressed in terms of labour times \textit{interchangeably} with actual money prices.    
However, this entire argument, it must be again and again re-emphasized, would hold for \textit{ANY} commodity satisfying assumptions $(1)$ through $(3)$ above. 
\section{Exploitation, Price, and Profit}
I've now laid out the foundational plumbing for the model: I've defined the three spaces, the probability measures on those spaces, the variables (both random and physical) central to these spaces, and documented a set of minimal assumptions which, once made, link them together. Of special significance of course was the linking of specific price $\Psi$ with the rate of profit $R$. Next I would like to turn to what F and J have to say about the actual shapes and parameters of these distributions, and along the way arrive at three different definitions of something which would be analogous to "the general rate of exploitation". \par 
The discussion will be centered around the specific price, $\Psi = \frac{\Pi}{\Lambda}$. Consider the $\auw$ price of the $i^{th}$ transaction $\Pi(i)$. Now, it's tautological to decompose this price into three components, through two stages of decomposition. First, there is the decomposition into cost of production plus profit. Then, the price of production can itself be broken into wages and not wages. Thus we can write for any commodity $C$ which is sold for a price $\pi(C)$, that
\[ \pi(C) = \pi' + v' + s' \]
where $s'$ is the profit (which could be negative), $v'$ is the wages paid (which could be obtained concretely in a variety of wages; there is a discussion which must be had of how to distribute the the total revenue of a firm throughout the products produced and sold by that firm during the period $T$, but for now we accept that such a thing can be done in an objective manner), and $\pi'$ is the amount paid on the materials and production - what Marx would call the constant capital. \par 
If we consider $\pi'$ in relation to $\pi$, it becomes extremely tempting to believe that in general $\pi'$ must be strictly less than $\pi$. This cannot be said in general, but it is extremely likely, approaching certainty. $v'$ is of course non-negative, and can only be $0$ if there exists a commodity which requires no living labour to construct from it's constituent parts, something which can be imagined for land, but very little else. $s' < 0$ would mean the price is lower than the cost of production, which if happening widely would imply a negative profit rate for a significant number of firms. In order for it to not be the case that $\pi' < \pi$, it would need to be the case that the firm sells their commodity at a loss, one which \textit{exceeds} the wages used in it's production. Farjoun and Machover assume for descriptive purposes that this is the case for a commodity in order to demonstrate a deeper identity which can be derived from such an assumption. $\pi'$ itself can be split into pieces, namely the different purchases that the producing firm made from other firms which constitute $\pi'$. Call these $\pi^{''}_1,\pi^{''}_2,...,\pi^{''}_{m}$. For any one of these, say $\pi^{''}_i$, we can apply the same argument again, decomposing into wages $v^{''}_i$, profits $s^{''}_i$, and raw materials $\pi^{'''}_i$, and this third term can itself be broken down into a finite set of subcomponents, and so on. Now, the assumption that$\pi'$ is always strictly less than $\pi$ for \textit{ANY} commodity is a very strong one, and F and J seem concerned with making it seemingly in the hopes of arguing that the sequence of all such $\pi$ terms must go to zero, since it is at that point an essentially monotonically decreasing function. However, the monotonicity here remains tricky even here, since what we have is not actually a sequence, but a directed set, and not a sequence of prices but rather in topological terms a \textit{net}. Thus, they have to make the rather extreme assumption that there is exactly one firm producing all of the raw materials for a given commodity. They do all of this in the hopes of writing $\pi$ itself as a complex sum of wages and surpluses, eventually eliminating entirely the raw materials term. But even with all of their assumptions which guarantee that the sequence of $\pi$ terms approaches $0$, this is of course insufficient to show that the sum of all such terms converges. \par 
This is all a big mess, which Farjoun and Machover plainly admit. What I propose is something simpler. Define, for a given commodity $c_i$ in market space, a set of $2n_f$ many random variables $V_1^c, V_2^c,...,V_{n_f}^c,S_1^c,S_2^c,...,S_{n_f}^c$, where $V_j^c(c_i)$ to be the total money wages paid by the $i^{th}$ firm, directly or indirectly, in the production of $c_i$, and $S_j(c_i)$ is the money profit made by the $i^{th}$ firm in the sale of the product of those wages. These numbers are extremely complex: when we consider the contribution of the $j^{th}$ firm to the commodity $c_i$, we must remember to include \textit{every} contribution they made to \textit{every} firm which \textit{eventually} contributed themselves to $c_i$. For instance, a firm producing steel might sell steel directly to a firm producing cars, but they might also sell steel to another firm producing tires which it eventually sells to the same firm producing cars. In this case we must add the wages paid for \textit{both} the steel directly used, and the wages paid for the steel used indirectly in the production of the tires. I think it is clear that these numbers exist in an objective sense, despite being nearly impossible to calculate perfectly, and despite being very complex can be very effectively approximated. \par 
So, with all of that said, we have, for any commodity in firm space, a slew of new random variables, two for every element of firm space. We can more importantly define, for any commodity $c_i$ the \textit{sum} of the $V$'s and the \textit{sum} of the $S$'s:
\[ V^c(c_i) = \sum_{j=1}^{n_f} V_j^c(c_i) \]
\[ S^c(c_i) = \sum_{j=1}^{n_f} S_j^c(c_i) \]
Where the $c$ superscripts serve to distinguish these random variables to the $V$ and $S$ random variables previously described over firm space. The claim that F and J are concerned with making is that the following is true in general for all commodities:
\begin{align}
	\Pi(c_i) &= \sum_{j=1}^{n_f} V^c_j(c_i) + S^c_j(c_j) \\
			&= \sum_{j=1}^{n_f} V^c(c_i)+S^c(c_j)
\end{align}
Or, more simply, in general, 
\begin{align}
	\Pi = V^c + S^c
\end{align}
To state this as an identity considering the mess of a conversation earlier might seem odious, so let's think about it. Take the steel sold to a firm in production of a car as an example. This steel has a variety of components associated with it, and we can trace it backwards to smaller and smaller contributions. Suppose in tracing back the steel, we eventually land at a firm which chops trees and produces lumber, that this firm pays nothing for use of the trees, and that they for simplicity also produce their own equipment used to chop the trees. That is to say, this is a firm which pays nothing for the raw materials of their own production. Of course, there is a markup at this point; the firm profits, meaning that they markup the price from what wages were used to produce the lumber, and these profits can easily be seen as a markup based on monopoly ownership of the lumber and materials. But if we were to relativize the theory to some other commodity, say oil, then this same decomposition would be \textit{impossible}, since assuming it ever terminates in the sense that it does for the lumber firm would imply that the firm owns the labour power used, which would be slavery. It appears to me that F and J's mathematical attempts to justify this identity made things more confusing than they needed to be, and that what we are saying here is not only valid, but perhaps an essential piece of evidence towards the logical \textit{necessity} of viewing a capitalist economy in terms of labour power. \par 
With all of that said, we can now turn to specific price. Note the identity we now have for it:
\[ \Psi = \frac{V^c}{\Lambda} + \frac{S^c}{\Lambda} \]
This by itself says something significant, if we note that
\begin{align}
	 V^c + S^c &= \sum_{i=1}^{n_f} V^c_i + \sum_{i=1}^{n_f} S^c_i \\
	 		   &= \sum_{i=1}^{n_f} (V^c_i + S^c_i)
\end{align}
And so 
\[ \Psi = \frac{V^c_1+S^c_1}{\Lambda} + \frac{V^c_2+S^c_2}{\Lambda} + \ldots + \frac{V^c_{n_f}+S^c_{n_f}}{\Lambda} \]
So $\Psi$ is the sum of an extremely large number of at the very least similarly distributed random variables which are weakly correlated and low in variance, which by the central limit theorem heavily insinuates that the overall sum has a is normally distributed. They argue that due to the integrated nature of modern capitalism, hundreds if not thousands of these terms are necessarily nonzero, and, since "even the biggest firms are quite small relative to the economy as a whole", most of these terms are small relative to each other. In any case, this is a claim which requires empirical testing to confirm, but considering how loose the conditions are in order for some version of CLT to apply, it seems fairly apparent in lieu of this decomposition that the distribution is typically mostly normal. \par 
More than this even, this new expression for $\Psi$ allows a closer inspection of the expected value. Note
\[ E(\Psi) = E\left(\frac{V^c}{\Lambda}\right)+E\left(\frac{S^c}{\Lambda}\right) \]
Considering the first of these two terms, we see that
\begin{align}
	E\left(\frac{V^c}{\Lambda}\right) &= \sum_{i=1}^{n_m} \frac{V^c(c_i)}{\Lambda(c_i)}P(c_i) \\
		&= \sum_{i=1}^{n_m} \frac{V^c(c_i)}{\Lambda(c_i)}\frac{\Lambda(c_i)}{\sum_{j=1}^{n_m}\Lambda(c_j)} \\
		&= \frac{\sum_{i=1}^{n_m}V^c(c_i)}{\sum_{j=1}^{n_m}\Lambda(c_j)} \\
\end{align}
The numerator here is the total amount of money paid in wages for the production of all of the commodities sold in the period $T$. Now, again we must be careful to remember that the commodities bought/sold during a period of time is not the same as the commodities produced during that same period. Thus the number here in the numerator here is \textit{not} simply the total amount paid in wages, which we denoted $S$ earlier when defining the $\auw$ unit and the wages random variable $W$. However, it will generally be quite close to $S$, given a few assumptions. In fact, these assumptions are very similar to two of the assumptions  made earlier. In order to safely assume that $S$ is a good approximation of the numerator here, we must have that:
\begin{itemize}
	\item[(1)] The vast majority of commodities bought and sold during the period $T$ are commodities which are either produced during the current period or during one of the nearby periods
	\item[(2)] The real wages paid (i.e. the quantity $S$) changes relatively slowly between periods. 
\end{itemize}
Turning to the denominator, what we have is the total labour content of the entire collection of commodities bought/sold in the period $T$, i.e. $\lambda(\mathbb{C})$ (where recall $\mathbb{C}$ is the physical bundle of all commodities in market space). Again, we are faced with the question of whether or not this amount of labour time is the same as the total labour done \textit{during} the period $T$, which is of course $n_l$. Assumption (1) above is enough to assume this to be the case, though with likely a bit less accuracy. What we are left with then is that, under assumptions (1) and (2), 
\[ E\left(\frac{V^c}{\Lambda}\right) \approx \frac{S}{n_l} = 1\auw = E(W) \]
So this term is approximately $1$, with proximity to one dependent on the extent to which capitalism perfectly "reproduces itself" between periods. Depending on the choice of $T$, it appears to me that this approximation could be quite a bit aways from $1$, but it is nonetheless an interesting observation. \par 
Next, we turn to the second term $E\left(\frac{S^c}{\Lambda}\right)$. An identical sequence of steps gives us that
\[ E\left(\frac{S^c}{\Lambda}\right) = \frac{\sum_{i=1}^{n_m} S^c(c_i)}{\sum_{j=1}^{n_m} \Lambda(c_j)} \]
This term is similar enough to the other one that they can be related to each other. Define 
\[ e^* = \frac{\sum_{i=1}^{n_m} S^c(c_i)}{\sum_{i=1}^{n_m}V^c(c_i)} \]
Then 
\begin{align}
	E\left(\frac{S^c}{\Lambda} \right) &= \frac{\sum_{i=1}^{n_m} S^c(c_i)}{\sum_{i=1}^{n_m} \Lambda(c_i)}\frac{\sum_{i=1}^{n_m} V^c(c_i)}{\sum_{i=1}^{n_m} V^c(c_i)} \\
	&= \frac{\sum_{i=1}^{n_m} S^c(c_i)}{\sum_{i=1}^{n_m} V^c(c_i)}\frac{\sum_{i=1}^{n_m} V^c(c_i)}{\sum_{i=1}^{n_m} \Lambda(c_i)} \\
	&= e^*E\left( \frac{V^c}{\Lambda} \right)
\end{align} 
And so we have, with only the assumptions implicit to the model itself and the claim that price can be broken up into wages and markup, the identity
\begin{align}
	E(\Psi) =  E\left(\frac{V^c}{\Lambda}\right)+E\left(\frac{S^c}{\Lambda}\right) &= E \left( \frac{V^c}{\Lambda} \right) + e^* E \left( \frac{V^c}{\Lambda} \right) \\
	&= (1+e^*)E\left(\frac{V^c}{\Lambda}\right)
\end{align} 
And if we include our minimal equilibrium assumptions about a capitalist economy which result in $E\left(\frac{V^c}{\Lambda}\right) \approx E(W) = 1$, then we have
\[ E(\Psi) \approx 1+e^* \]
Remember that $E(\Psi)$ is this model's analog of $\psi_0$, Marx's constant of proportionality for the claim that prices are proportional to labour times. Thus $1+e^*$ here denotes \textit{a general markup on commodity process as one hundred plus $e^*$ percent of the labour time embodied in the commodity}. What is $e^*$, exactly? I will call this the \textbf{market rate of exploitation}, do distinguish it from the two similar numbers I'll define shortly, and highlight that it is a number defined purely within market space, rather than firm or labour-power space. What this number is, at face value, is the ratio of total markup of all commodities produced in a period of time $T$ versus total wages paid in that production. Purely from inspection of the number then, despite what I've decided to call it, this doesn't seem to highlight with any great that profits from the sale of commodities actually come from the expropriation of surplus labour from the worker. After all, there's no telling where the markup in prices came from, or why these markups were accepted by consumers. \par 
However, we can define a similar number from our other two spaces. First, consider labour-power space. The total labour done in the period $T$ is of course $n_l$. We also have the bundle of commodities $\mathbb{V}$ which constitute the real wages paid to these workers, i.e. the bundle of goods consumed by workers during this period. Then the difference $n_f - \lambda(\mathbb{V})$ \textit{directly} has an exploitative character: it is exactly how many units of labour power had no material benefit for the workers, labour which they offered up as tribute to the capitalist economy. We can then define a much more provocative \textbf{global rate of exploitation}
\[ e_g = \frac{n_l - \lambda(\mathbb{V})}{\lambda(\mathbb{V})} \]
A few caveats are in order before messing around too much with this number. First, we should note that there is no guaruntee that workers aren't purchasing their goods exclusively with the wages they earned during the period $T$. This muddies the initial claim that the difference in the numerator is \textit{exactly} the amount of labour offered up as tribute during the period $T$. However, the vast majority of workers in a modern economy spend money they earn during a pay period \textit{during} that period, and save relatively little of it. Thus, the claim, while not being exact, is \textit{far} from outright invalid, and that's without invoking any kind of confusing periodicity argument, which could be done if I felt like giving myself a headache. \par 
Now, turning directly to the matter at hand, the bundle $\mathbb{V}$ is absolutely massive, extremely varied, and very much unbiased. Workers in Marx's day may not have consumed much in the way of industrial equipment in the sense of steel, oil, electricity and so forth, but now they \textit{do}; with few exceptions they consume a bit of everything, and the exceptions that exist only exist at a "top level" (for example, workers don't spend their wages purchasing military drones, but they very likely \textit{do} spend their wages purchasing products made from the same materials which were used in order to create those drones). Thus, we can apply our statistical version of Marx's labour theory of value:
\[ \frac{\pi(\mathbb{V})}{\lambda(\mathbb{V})} \approx E(\Psi) \approx 1+e^* \]
What is $\pi(\mathbb{V})$? It's the wages paid to workers in actual money units. Particularly, units of $\auw$. However, from the definition of this unit, this total wage amount is simply itself $n_l$ (recall that if $S$ is the total wages in dollars paid to workers, then $1\auw = \frac{S}{n_l}$, and so $S$ dollars in $\auw$ is $S \dollars = S \times \frac{n_l}{n_l} \dollars = n_l \times \frac{S}{n_l} \dollars = n_l \auw$.) So $\pi(\mathbb{V}) \approx n_l \auw$ (approximate because, again, workers will save some of their money to spend at a later period, and buy some commodities using money saved from a previous period). Thus,
\begin{align}
	e_g &= \frac{n_l}{\lambda(\mathbb{V})} - 1 \approx \frac{\pi(\mathbb{V})}{\lambda(\mathbb{V})} - 1  \\
		&\approx (1+e^*)-1 = e^* \\
		\implies e^* \approx e_g
\end{align}
Suddenly then, the seemingly innocent markup of prices due to $e^*$ takes on a much more exploitative significance - profits suddenly seem to suddenly depend entirely on the exploitation of labour power. However, we're getting a bit ahead of ourselves. It's worth re-emphasizing here that the number $e^*$ \textit{is uniquely definable in a model of the sort we are working with relativized to labour as the central commodity.} The decomposition of price into wages and surplus would not be obviously valid (to me at least), relativized to oil, unless we were to view capitalist firms as interchangeable with individuals, or at least that every firm is a worker co-op, because otherwise we would have to assume that firms own the labour-power of their production, which would be slavery. \par 
We've linked profits from prices to exploitation of labour, but what about the rate of profit of a firm? Let's turn back now to firm space, and see what we can discern. Here, we have the rate of profit $R$ of a firm, the capital of the firm $K$, the portion of capital in the form of wages $V$, and the proportion of wages to total capital $Z = \frac{V}{K}$. We saw that with the basic assumption that capitalist firms reinvest a portion of profits into production that the capital of a firm grew exponentially with $R$ directly related to the base of that exponential function, and we also saw that assuming a constant "technical composition of capital" that the behavior of $Z$ with time mimicked the behavior of $R$. (We in fact saw that the reciprocal $\frac{1}{Z}$ was nearly identical to what Marx referred to as the organic composition of capital.) Finally, we defined, but did not discuss much, the rate of capital versus labour $X = \frac{R}{Z}$, which we saw was analogous to Marx's famous formula for the rate of exploitation $\frac{S}{V}$. This variable $X$ is obviously the one we need to look at very closely now. In particular, let's look at the number 
\begin{align}
	e_0 &= \frac{E(R)}{E(Z)} \\
	&= \frac{\pi(\mathbb{P}) - \pi(\mathbb{I}) - \pi(\mathbb{V})}{\pi(\mathbb{V})}
\end{align}  
That is, $e_0$, clearly related though not exactly equal to $E\left( \frac{R}{Z} \right) = E(X)$, is precisely the ratio of total profits to total wages in the economy. This is perhaps even more explicitly related to exploitation than $e_g$: the bigger it is, the bigger the tributary products of worker's labour power is leaving the worker's hands and not contributing directly to their well being (outside of government intervention, at least). \par 
Also more direct is the immediately discernable relationship between $e^*$ and $e_0$. It appears at first glance that they are identical: $e^*$ is the total profit of the economy computed in terms of overall markups in prices of commodity sales, and $e_0$ is the total profit of the economy computed in terms of aggregating the overall profits of each firm. How these differ of course is in the same ways that anyone reading this is probably used to at this point: commodities bought and sold don't identically correlate with those produced in a fixed period $T$, and profit rates and labour contents change between periods. Thus, in order to assume that $e_0 \approx e^*$, we must assume:
\begin{itemize}
	\item[(1)] The commodities bought and sold during the period $T$ are mostly the same as those produced in the period $T$, or at least produced in a preceding period close to the present one.
	\item[(2)] The distribution of $X$ changes relatively slowly over time, i.e. between periods (more directly, this is equivalent to saying that $R$ and $Z$ change relatively slowly with time).
\end{itemize}
Thus we have three numbers measuring exploitation of labour in a capitalist economy, all roughly equal with one another with fairly minimal and loose conditions of equilibrium/normality in the economy. These numbers link the three "viewpoints" of the capitalist economy, that of the market, that of the firm, and that of labour, together into a coherent model which seems to not so much as imply a labour determined system but \textit{command} it, \textit{necessitate} it, and that was despite my attempt to remain as unbiased towards labour as possible, at every possible moment noting the extent to which the choice of labour as the central relative commodity is arbitrary. \par 
\section{Estimating the Shape of the Distributions}
Before finishing this section, we'll note some of the more tenuous arguments made by Farjoun and Machover, ones which would require empirical verification. Firstly, they conjecture that both $R$ and $Z$ are best fit to a gamma distribution. Like any argument for such a thing, this is done by arguing that the probability of both $R$ and $Z$ being less than $0$ is negligibly small. F and J make two adequate arguments for this. First, they argue that while there are in a capitalist economy always a high number of loss making firms, that these firms are overwhelmingly operating at very low levels of capital, and since our probability measure weighs firms according to the proportion of capital operating through them, this will massively skew the distribution to the positive half of the number line. The second argument they make is that even among the loss making firms, these firms are typically making a loss only after payment on the interest of capital that has been borrowed, and F and J explicitly decide to measure the capital of a firm \textit{prior} to these payments. Both of these arguments apply equally well to both $R$ and $Z$. \par 
Thus they assume that $R \sim Gamma(\alpha,\beta)$, and $Z \sim Gamma(\alpha',\beta)$. They assume without much argument that these two variables have the same scale parameter $\beta$, which I understand since in a certain sense $Z$ emulates $R$ as a constant proportion of $R$, but this would require empirical evidence and more of an actual logical argument to really justify. \par 
They then note that the value $e_0$ has, according to the empirical evidence they have seen, barely changes at all for long periods of time, and that the variable random variable $X$ has a nearly degenerate (i.e. nearly $0$ variance) distribution, also not changing with time, and also based on their empirical evidence. Furthermore they mention that this observed unchanging value of $e_0$ is $1$. It appears that more recent evidence has called this into question, however, and that it is actually closer to $\frac{1}{2}$. They then go on to argue that the initial two empirical observations necessitate the third, in a clean logical argument which I'll replicate here:
\begin{fact}
	Suppose $R \sim Gamma(\alpha,\beta)$, $Z \sim Gamma(\alpha',\beta)$, and $V\left( \frac{R}{Z} \right) = 0$. Then $R = Z$, and $\frac{E(R)}{E(Z)} = 1$.  
\end{fact} 
\begin{proof}
	If $V\left( \frac{R}{Z} \right) = 0$, then $\frac{R}{Z}$ is in fact a constant, call is $a$. Then $\frac{R}{Z} = a \implies R = aZ$. Thus it follows that $E(R) = aE(Z)$ and $V(R) = a^2V(Z)$. Writing these equations in terms of the parameters $\alpha, \alpha', \beta$, we have that
	\[ \frac{\alpha}{\beta} = a\frac{\alpha'}{\beta} \implies \frac{\alpha}{\alpha'} = a \]
	\[ \frac{\alpha}{\beta^2} = a^2 \frac{\alpha'}{\beta^2} \implies \frac{\alpha}{\alpha'} = a^2 \]
	From which it can be plainly seen that
	\[ a = a^2 \]
The shape parameter of a Gamma random variable must be greater than $0$, meaning that $a$ must also be greater than $0$. The only positive number $a$ such that $a = a^2$ is $1$. Thus $a = 1$, from which it follows that $\alpha = \alpha'$. 
\end{proof}
If it is truly the case that more recent empirical evidence shows this $a$ to not be $1$, this fact is not useless: it would in fact contrapositively imply that at least one of the other empirically justified claims must \textit{also} be false. Thus this fact implies that either the assumption that the scale parameters are the same is false, or the distribution of $X$ is not actually degenerate, or both, is what we are left to conclude. \par 
F and J were excited to have reached a definite estimate of $e_0$, since $e_0 \approx e^*$, meaning that it fixes the parameters of the normally distributed $\Psi$. That $\Psi$ is actually normal without much of any assumptions (aside from the decomposition of price into wages and profits) is fairly clear, and recall that with some extra minimal assumptions we arrived at the estimate that $E(\Psi) \approx 1+e^*$. With their estimate of $e_0 = 1$, they then conclude that $E(\Psi) = 2$. They then, with one more assumption about the model, proceed to derive an estimate for the standard deviation. \par 
The assumption is that the probability of $\Psi < 1$ is extremely small. This, to me, is one of the more innocent of the assumptions we've made so far. If $\Psi$ is consistently less than $1$, then that would mean that commodities are consistently being bought and sold at a loss in terms of the centralizing commodity. In the case of our particular model that central commodity is labour, so concretely it would mean that people are consistently buying commodities for lower prices than it would cost them to simply hire the manpower to produce the commodity themselves, provided they have the other raw materials and machinery. To highlight how absurd this would be, suppose this situation is true for oil, rather than labour. E.g. I buy plastic consistently for less than the cost of the oil required for refinement into plastic. If this were the case, then the firms producing oil would be consistently operating at a negative rate of profit, and very quickly nobody would be making oil anymore. If we assume this is true for labour, then that means that labour-power is consistently sold at a price lower than it's cost of production, which would mean that the workers selling their labour are consistently being paid with too little money to simply continue existing, and the working population would quickly die out from starvation, exposure, disease, etcetera. (If one insists on an argument which makes no mention of an ideal labour content of labour power, then we could instead simply say that the average workers themselves would quickly adjust their daily consumption to whatever lower quality of life that their wages actually allow. At the end of the day though, we would be back in the situation in which $\Psi$ is generally nonnegative.)  \par 
 This observation, independently of any arguments we might be interested in making about standard deviations, pokes at the very heart of the tautological nature of the model and the "anything" theory of value. \textbf{The labour theory of value, in statistical aggregate, is true, because the "anything" theory of value is true, because if it \textit{weren't} true, then the capitalist system would either have to quickly adjust itself in order to \textit{make} the theory true, or collapse into chaos.} \par 
 In any case, they then supply a rough estimate that the probability of $\Psi < 1$ is less than $\frac{1}{1000}$. This constrains the normal distribution to one such that the standard deviation is at most $\frac{e_0}{3}$, or in their particular case $\frac{1}{3}$. Thus they conclude that the specific price is normally distributed, with mean approximately $2$, and standard deviation approximately $\frac{1}{3}$. I find this reasoning entertaining but not nearly as important as they seem to think it is. The major results of their model, it seems to me, don't hinge on any particular shape to any of these distributions. 
\section{Dynamics Over Time}
\subsection{Law of Decreasing Labor Content}
	Assume for simplicity that the capital of a firm $f_i$ is unchanging between a particular pair of periods $T_1,T_2$. Suppose that the firm found a way to reduce it's unit costs (that is, reduce the dollar cost of producing a unit of it's output), whereas all competing firms (that is, firms producing the same product) continue to produce at the conventional standard costs. Then there is no reason to assume that the price of this product once sold will differ from what it was before; that is to say, it is extremely likely that either
\begin{itemize}
	\item[(1)] The firm produces the usual amount of product at the reduced unit costs. In this case $\Pr(f_i)$ will very likely remain the same as in the previous period, while $I(f_i)-V(f_i)$ will shrink (that's simply the hypothesis) and equivalently $K(f_i)$ will also be able to shrink, so that the profit rate $R(f_i)$ increases
	\item[(2)] The firm produces invests the same or more capital in production at the reduced unit costs, i.e. $K(f_i)$ either remains the same or increases, and equivalently $I(f_i)-V(f_i)$ remains the same or increases. $Pr(f_i)$ however will increase to remain higher than the increase in $K(f_i)$, and expressing the rate of profit by
	\[ R = \frac{Pr}{K}-1 \]
makes it cleat that the rate of profit increases. 
\end{itemize} 
A third option would be that the firm intentionally sells their product at a lower price, in which case their rate of profit can remain the same, while nonetheless squeezing their competitors and pushing them out of the market. In any case, it is clear that in a capitalist economy, capitalist firms will seek to reduce the unit costs of production. There are four ways in which this can be pursued:
\begin{itemize}
	\item[(1)] Reduce the workers' rates of pay
	\item[(2)] Beat down the prices it pays to suppliers of non-labour inputs
	\item[(3)] Reduce the direct labour time spent per unit of output, either by speeding up the labour process or by "rationalizing" it.
	\item[(4)] Replace existing non-labour inputs by others which it can buy more cheaply. 
\end{itemize} 
What I'm concerned with is the way in which this incentive structure of a capitalist economy will over time reduce the labour content of an arbitrary commodity. Towards showing this, items (1) and (2) have no influence. Firms will pursue these options, without a doubt, but will find themselves hindered by direct resistance of the workers and other supplying firms, forcing the firm in question to turn to (3) and (4) more often than these two. \par 
The effect of strategy (3) is direct - if this is accomplished, then the labour content of the commodity in question will fall, and competing firms will find themselves forced to adopt similar organizational schemes, resulting in an overall fall of the labour content of the commodity in general. \par 
This leaves us at option (4), which is by far the most commonly pursued method of reducing costs. Unlike (3), there is no direct reason to assume that this will result in a reduction in labour contents, even as it is employed repeatedly across the economy. This is what we wish to show. Suppose that a firm uses up a particular bundle of commodities $\mathbb{B}_1$ of non-labour inputs to produce one unit of output. This is replaced by a new bundle $\mathbb{B}_2$ which is a completely different set of materials, but with the property that $\pi(\mathbb{B}_1) > \pi(\mathbb{B}_2)$. For simplicity we'll write $\pi_1$ and $\pi_2$ in place of $\mathbb{B}_1$ and $\mathbb{B}_2$. Define the \textbf{cheapening factor} $c$ by
	\[ c = \frac{\pi_1}{\pi_2} \]
We will also define the \textbf{labour content ratio}, abbreviated lcr, as 
	\[ h = \frac{\lambda_1}{\lambda_2} \]
where of course $\lambda_1 = \Lambda(\mathbb{B}_1)$ and $\lambda_2 = \Lambda(\mathbb{B}_2)$. We know that $c > 1$, and we wish to inspect what $h$ will be. Of course, the ratios $\psi_1 = \frac{\pi_1}{\lambda_1}$ and $\psi_2 = \frac{\pi_2}{\lambda_2}$ are the specific prices of these commodity bundles. Writing $h$ in terms of the specific prices gives us a relation between $h$ and $c$:
\begin{align}
	h = \frac{\lambda_1}{\lambda_2} \frac{c}{c} = c \frac{\lambda_1}{\lambda_2} \frac{\pi_2}{\pi_1} = c \frac{\lambda_1}{\pi_1} \frac{\pi_2}{\lambda_2} = c \frac{1}{\psi_1} \psi_2 = c \frac{\psi_2}{\psi_1}
\end{align}
Towards replacing this particular instance with a general one involving random variables, note that from the perspective of the firm, the specific prices of commodities are invisible. They only see prices in dollars, and so the resulting change in specific prices might as well be considered random. Suppose the firm $f_i$ uses the bundle $\mathbb{I}$ consists of the commodities $I_1,I_2,...,I_M$ (re-indexed from market space). Each of these has a specific price $\Psi(I_1) = \psi_1, \Psi(I_2) = \psi_2$,...,$\Psi(I_M) = \psi_M$, as well as a labour content $\lambda_1,...,\lambda_M$. The capital $M$ is to indicate that $M$, the number of commodities purchased as non-labour inputs, is itself a random variable on firm space. This is going to make things complicated, however, so we will hold $m$ as non-random by fixing it to be the maximum number of inputs bought by \textit{any} firm during the period in question, and where a particular firm has only bought $j < m$ many commodities, we simply let $\lambda_{j+1}= \lambda_{j+2} = \ldots... = \lambda_m = 0$.  What we are interested in when it comes to specific price is the ratio of total price to total labour content. The total labour content is obviously $\sum \lambda_i$. Meanwhile, the total price can be written $\sum \lambda_i \psi_i$, since this product is precisely $\pi_i$. Thus we define the \textbf{aggregate specific price} of the $i^{th}$ firm to be
 \[ \bar{\Psi}(f_i) = \frac{\sum_{i=1}^m \lambda_i \psi_i}{\sum_{i=1}^M \lambda_i} \]
The numerator in particular here is itself a random variable representing the total price of these inputs, which we will call the \textbf{aggregate price}, and denote $\bar{\Pi}$. The denominator meanwhile is of course the \textbf{aggregate labour content}, denoted $\bar{\Lambda}$. Thus we can see that another way to express aggregate specific price is with the identity 
\[\bar{\Psi} = \frac{\bar{\Pi}}{\bar{\Lambda}} \]
In the following analysis, we will actually be considering random variables over the product of two firm spaces for two different periods $T_1$ and $T_2$. In each period we have commodity bundles $\mathbb{I}_1$ and $\mathbb{I}_2$. For each of these, we have two aggregate prices $\bar{\Pi}_1$ and $\bar{\Pi}_2$, and two aggregate specific prices $\bar{\Psi}_1$, $\bar{\Psi}_2$. From these, and from the expression derived earlier more informally relating the lcr to the cheapening factor by way of specific prices, we can define the cheapening factor $C$ and the lcr as random variables:
\[ C = \frac{\bar{\Pi}_1}{\bar{\Pi}_2} \]
\[ H = C \frac{\bar{\Psi}_1}{\bar{\Psi}_2} = \frac{\bar{\Lambda}_1}{\bar{\Lambda}_2} \]
Where the equality in the expressions of $H$ follows in exactly the same way as before, via the fact that aggregate specific price is the ratio of aggregate price to aggregate labour content. \par 
Consider instead of $H$, the log of $H$:
\[ \log(H) = \log(C) + \log(\bar{\Psi}_1) - \log(\bar{\Psi}_2) \]
Note of course that $H > 1$ if and only if $\log(H) > 0$. Taking the expected value of both sides and applying linearity we obtain
\[ E(\log(H)) = E(\log(C)) + E(\log(\bar{\Psi}_1) - E\log(\bar{\Psi}_2)) \]
Now consider $\Psi_1$ and $\Psi_2$ more closely. It might not be the case that the market transactions corresponding to the bundles $\mathbb{B}_1$ and $\mathbb{B}_2$ occurred during the same period $T$. Nonetheless, we can assume, as we've been assuming, that they at least occur during a nearby period, one in which the methods of production and subsequently the price structure of the economy are relatively unchanging. Thus, ideally we have that $\bar{\Psi}_1$ and $\bar{\Psi}_2$ are identically distributed, so that $E(\log(\bar{\Psi}_1)) - E\log(\bar{\Psi}_2)) = 0$, or more realistically, they are \textit{almost} identically distributed, in which case the difference is very close to $0$. Thus
\[ E(\log(H)) \approx E(\log(C)) \]
with equality if market transactions for the two bundles occurred during the same period $T$. But since $C > 1$ with probability $1$, it follows that $E(\log(C)) > 0$, from which $E(\log(C)) > 0$, from which $E(\log(H))$ is almost certainly also greater than $0$. \par 
Now consider a sequence of $n$ changes over time, given by the commodity bundles $\mathbb{B}_1,\mathbb{B}_2,...,\mathbb{B}_n,\mathbb{B}_{n+1}$. For each transition $\mathbb{B}_i \to \mathbb{B}_{i+1}$, there is a corresponding lcr $H_i$, and by the above discussion it follows that we can reasonably assume that $E(\log(H_i)) > 0$ for each $i$. Define 
\[ H^*_n = \frac{\bar{\Lambda}_1}{\bar{\Lambda}_{n+1}} \]
I.e. we are interested in the \textit{overall} change in labour content of the non-labour inputs during the entire history of the firm. Note that 
\[ H^*_n = \frac{\bar{\Lambda}_1}{\bar{\Lambda}_{n+1}} = \frac{\bar{\Lambda}_1}{\bar{\Lambda}_2}\frac{\bar{\Lambda}_2}{\bar{\Lambda}_3}\ldots \frac{\bar{\Lambda}_n}{\bar{\Lambda}_{n+1}} = \prod_{i=1}^n \frac{\bar{\Lambda}_i}{\bar{\Lambda}_{i+1}} = \prod_{i=1}^n H_i \]
Thus by properties of logarithms
\[ \log(H^*_n) = \sum_{i=1}^n \log(H_i) \]
and subsequently by linearity of expectation,
\[ E(\log(H^*_n)) = \sum_{i=1}^n E(\log(H_i)) \]
From this it becomes clear that 
\[ \lim_{n \to \infty}P(\log(H^*_n) > 0) = 1 \]
The only assumption required to conclude this from our expression is to have it be the case that the expectations of the $\log(H_i)$ approach $0$ at a very high speed, which seems absurd. F and J justify this scenario by comparison to why the cumulative gains of a casino are nearly guaranteed to be positive if the expected gains of each of the games in the casino are positive (i.e. if each game is at least a little bit unfair). This is intuitively trivial: if one expects to lose money on any game they play in the casino, however little and whatever the distributions actually are, then one would expect that the more games one plays, the likelier it is that the luck of the player eventually runs out, and they end up in the red. That's exactly what is happening here. They claim the result follows from the law of large numbers, which it does if one assumes that the distributions of the $H_i$'s are all identical and independent, but neither of this assumptions should be necessary. It would have been nice if F and J provided a set of minimal conditions for this to be the case, but alas, they don't, and I don't have the time to come up with one, for a claim that is so obviously true. \par 
To summarize, what we have shown is that, under the sole assumption that all changes to the non-labour inputs to production of a firm are cost-saving, repeated use of strategy (4) will guarantee an eventual reduction in the labour content of the bundles of commodities being swapped. Across the entire economy then, as firms in all sectors update and innovate, that the labour content of all commodities must decrease over time, with probability approaching one as time approaches infinity. \par 
The next question to ask is how quickly this process occurs. First, we can note with no further assumptions that an even marginally high probability of the $\log(H)$ terms in general being positive will have an exponential effect on the growth rate of $H^*_n$ with $n$, since the dependence of $H*_n$ to the individual $H$'s is multiplicative. With nearly no assumptions at all, we can show it to be reasonable to assume that $P(\log(H) > 0)$ is at least $\frac{1}{2}$. To see this, note that since $C$ is always assumed greater than $1$, $\log(C)$ is always greater than $0$, and so
\begin{align}
	P(\log(H) > 0) &= P(\log(C)+\log(\bar{\Psi}_1) - \log(\bar{\Psi}_2) > 0) \\
				   &= P(\log(C) + \log(\bar{\Psi}_1) > \log(\bar{\Psi}_2)) \\
				   &\geq P(\log(\bar{\Psi}_1) > \log(\bar{\Psi}_2)) \\
				   &\approx \frac{1}{2}
\end{align} 
Where the final approximation follows from symmetry by the fact that $\Psi_1$ and $\Psi_2$ are approximately identically distributed. This fact alone is enough to conclude that the reduction in labour content over time will not be sluggish. However, we can proceed further still than this, by conducting an analysis of the standard deviation of our $\log(H)$ random variables. The speed at which labour content falls depends, among other things, on how concentrated around the mean this distribution is. We know that $E(\log(H))>0$. If the standard deviation is small, then $\log(H)$ will have a very high probability of being very close to the mean, and thus a very high probability of being positive. So far, we haven't made any assumptions at all about the actual distributions of any of these random variables, aside from that $P(C > 0) = 1$ whenever the two periods determining $C$ involve a change in the bundle of non-labour inputs. Nor have we made any assumptions of the relations to each other. In order to make analyze the speed at which this might be expected to occur, we will now highlight some reasonable assumptions which greatly influence this. The first such assumption is that $C$ is generally independent of $\bar{Psi}_1$ and $\bar{\Psi}_2$. This one is fairly innocent. Recall the capitalist is strictly concerned with reducing prices, and are not directly looking at specific prices at all. Thus the specific prices should from the psychological motivations of the capitalists generally be considered independent. Our second assumption is that the correlation random variables $\log(\bar{\Psi}_1)$ and $\log(\bar{\Psi}_2)$, is non-negative. This is because when switching from one commodity bundle to another, it will rarely if ever be the case that there is \textit{nothing} in common between the two bundles. Usually, almost always, there will be commodities in common, and this would create a positive correlation. If $\log(\bar{\Psi}_1)$ and $\log(\bar{\Psi}_2)$ are nonnegatively correlated, then of course $\log(bar{\Psi}_1)$ and $\log(\bar{\Psi}_2)$ are non-positively correlated, and so  
\begin{align}
	Var(\log(H)) &= Var(\log(C)+\log(\bar{\Psi}_1)-\log(\bar{\Psi}_2)) \\
			&= Var(\log(C))+Var(\log(\bar{\Psi}_1)) + Var(\log(\bar{\Psi}_2)) - 2Cor(\log(\bar{\Psi}_1),-\log(\bar{\Psi}_2)\\
			&\leq Var(\log(C))+Var(\log(\bar{\Psi}_1)) + Var(\log(\bar{\Psi}_2))
\end{align}
Since $\bar{\Psi}_1$ and $\bar{\Psi}_2$ are approximately identically distributed, the two variance terms are the same. Let's consider what this is exactly, and to do this we must look back at our formula for what the aggregate specific price is in the first place. Namely, it is a weighted sum of $m$ specific price random variables, which can be reasonably assumed independent of each other, and also reasonably assumed normal. $m$ itself is likely quite large, typically in the hundreds if not thousands. By both properties of sums of normal random variables then, or by the central limit theorem, we can say with, if anything, more confidence than we did before about the ordinary specific price distribution, that this thing is normal, with mean $E(\Psi)$ and standard deviation $\frac{SD(\Psi)}{\bar{m}}$, where $bar{m}$ is the average number of non-zero terms (i.e. $E(M)$, and still quite large). It was estimated earlier that $SD(\Psi)$ was less than a third, and so this standard deviation, proportional to how diverse the labor inputs are to capitalist firms, is very, \textit{very} small. \par 
The upshot is that it is very likely that $Var(\log(\bar{\Psi}_1)) + Var(\log(\bar{\Psi}_2))$ is quite small, so that \textit{the distribution of $\log(H)$ is just a bit narrower than that of $\log(C)$, likely extremely similar}. Now recall that we observed earlier that $E(\log(H)) \approx E(\log(C))$, and also have that since $P(C > 1) = 1$, that $P(\log(C) > 0) = 1$, and that $E(\log(C)) > 0$. From all of this it follows that $P(\log(H) > 0) \approx 1$. A \textit{single} change in non-labour inputs is extremely likely to drop the labour content! 
\subsection{Accumulation and the Rate of Profit}
Farjoun and Machover make the important point that effective modeling of accumulation cannot be measured by simply measuring the total raw amounts of goods over time, since commodity types are constantly being abandoned in favor of new commodity types. The only way to effectively track this is by measuring the contents of some chosen universal, in our case of course this being labor content. \par 
Recall the general rate of profit, which we saw to also be the expected rate of profit, was given by
\[  r_g = \frac{\pi(\mathbb{P}) - \pi(\mathbb{I}) - \pi(\mathbb{V})}{\pi(\mathbb{K})} \approx \approx \frac{\lambda(\mathbb{P}) - \lambda(\mathbb{I}) - \lambda(\mathbb{V})}{\lambda(\mathbb{K})}  \]
Recall also that 
\[ \lambda(\mathbb{P}) \approx \lambda(\mathbb{C}) \approx n_l \]
Simplifying the notation by letting $\lambda(\mathbb{V}) = \nu$ and $\lambda(\mathbb{K}) = \kappa$, we have
\[  r_g \approx \frac{n_l - \nu}{\kappa} \]
Here $n_l - \kappa$ is clearly very closely analogous to Marx's $S$, the total surplus value generated by the economy, or said differently, the total labor which is alienated from the workers themselves. Similarly $\kappa$ is analogous to $C+V$, in Marx's model, and so we can see that $r_g$ itself is analogous in Marx's model to $\frac{S}{C+V}$, which is exactly what he defined as the rate of profit. From this we can see that the general rate of profit is directly proportional to the "surplus value generated" $S$ and inversely proportional to the total capital. \par 
An important observation can be made here. If the economy is "closed", that is, then the sole source of growth is reinvestment of the surplus produced. If the entire surplus were reinvested, it would add $n_l+\nu$ units of labor content per period $T$. Thus we have, in continuous terms,
\[ \frac{d\kappa}{dt} < n_l-\nu \]
where the strict inequality follows from two different observations. First, it is impossible to truly reinvest all surplus into production, since if this were done the capitalist class would quickly starve to death. Second and more importantly, it is important to clarify something which should already have been clarified: the labor content of a commodity during a particular period is to be measured in terms of the \textit{current} technical coefficients of that period. If a commodity is produced in period $T_1$, and used as capital during $T_2$, and it happens that the labor content of this commodity type has shrunk due to the law of decreasing labor content between these periods, then the labor content of the commodity will have shrunk. This drop means that while the increase in $\kappa$ over time is bounded by $n_l-\nu$, there is also a reverse force pulling this value down all of the time. \par 
Farjoun and Machover view $n_l-\nu$ as the source of increasing capital flows, and the law of decreasing labor content as a sink. They identify two more sinks: destruction of capital stock due to war, and destruction of capital stock during times of economic crisis. \par 
With these observations in hand, let's review Marx's own explanation of the tendency of the rate of profit to fall, but within the vocabulary developed here. Marx defined his rate of exploitation as $e = \frac{S}{V}$, his rate of profit as $r = \frac{S}{C+V}$, and his organic composition of capital as $o =\frac{C}{V}$. He argued that the organic composition would tend to rise over time: due to labor content of commodities tending to decrease over time, subsistence wages would be allowed to fall, producing a drop in $V$, and that this reduction in real wages would be accompanied by an increase in the reliance on heavy machinery, producing a rise in constant capital $C$. A mathematical way to present the consequences of this would be to first re-express the rate of profit in terms of the rate of exploitation and the organic composition:
\[ r = \frac{S}{C+V}\times \frac{\frac{1}{V}}{\frac{1}{V}} = \frac{e}{1+o} \]
From here it is plain to see that if $o$ is to continually increase, this immediately produces a decrease in the rate of profit, unless accompanied by a simultaneous increase of exploitation. Thus, the capitalist class will continually operate under the incentive of increasing exploitation, since this will be the only way to stop their profit rates from falling. \par 
Now, let's put this in our own terms. In place of his rate of exploitation, we have our global rate $e_g = \frac{n_l-\nu}{\nu}$. Our own general rate of profit can be expressed in terms of this rate:
\[ r_g \approx \frac{n_l-\nu}{\kappa} = \frac{e_g\nu}{\kappa} = \frac{\nu}{\kappa}e_g \]
This equation is in full agreement with Marx's model, since $\frac{\nu}{\kappa}$ is analogous to $\frac{V}{C+V}$, which when inverted becomes $1+\frac{C}{V}$, which is exactly what we find in the denominator or $r$. Thus this equation is exactly analogous to Marx's in the way we would expect - we are simply operationalizing it in our own terms. Towards continuing this process, let us define 
\[ q_g = \frac{\kappa}{\nu} \equiv 1+o \]
The difficulty of this argument is that in order to actually view this as a tendency for the rate of profit to fall, we must produce an argument for why the rate of exploitation $e_g$ will generally fail to rise fast enough to counteract the rise of $q_g$. Farjoun and Machover go in a different direction from this, and their results are quite interesting. First, they solve for $\nu$ in the equation for $e_g$:
\begin{align}
	e_g &= \frac{n_l-\nu}{\nu} = \frac{n_l}{\nu}-1  \\
	&\implies e_g+1 = \frac{n_l}{\nu} \implies \nu = \frac{n_l}{1+e_g} 
\end{align}
Then 
\begin{align}
	r_g = \frac{\nu}{\kappa}e_g = \frac{n_l}{(1+e_g)\kappa}e_g = \left(\frac{n_l}{\kappa}\right) \left(\frac{e_g}{1+e_g} \right)
\end{align}
This is a fascinating equation, completely consistent with Marx's model but obscured by the fact that within that model, there wasn't a simple variable corresponding to the total labor done by society $n_l$. In our model, $n_l$ is the number of units of labor power employed during the period $T$, while $\kappa$ is the labor content of the total capital employed by society during the period $T$. Now, as a function of $e_g$, $\frac{e_g}{1+e_g}$ is strictly increasing, and thus the rate of profit here is still proportional to $e_g$. But this proportionality is no longer direct, and in fact, reveals the inadequacy of the rate of exploitation as a correcting factor. This function of $e_g$ is in fact \textit{bounded}, converging to $1$ as $e_g \to \infty$. Thus, if the first term $\frac{n_l}{\kappa}$ can and tends to decrease without bound (a claim we are about to consider), an accompanying increase of the global rate of exploitation $e_g$ can only temporarily prevent the global rate of profit from decreasing over time; it is powerless to stop it in the long term. \par 
Let's compare the term $q_g' = \frac{n_l}{\kappa}$ with the term $q_g$. What we currently have is two equations for the general rate of profit:
\[ r_g \approx e_g\frac{1}{q_g} \approx \left(\frac{e_g}{1+e_g}\right)\frac{1}{q_g'} \]
Clearly the two terms $q_g$ and $q_g'$ are similar, and in fact the general rate of profit is inversely related to both. For a fixed period $T$, $q_g = \frac{\kappa}{\nu}$ is the ratio of the total labor content of all capital used in production to the total labor content of real wages. $q_g'$ is the ratio of the total capital used in production to \textit{the total labor employed during the period} - or, more simply, the ratio of total capital employed per worker, measured in labor time. Marx seems to conflate these two terms, and assume that $q_g$ increasing necessarily implies an increase in $q_g'$. For these terms to be the same, the workers would have to be receiving \textit{all} of their labor back in the form of wages! So they are absolutely different quantities, but should be directly relatable to each other in terms of the global rate of exploitation $e_g$, since, as we have just noted, equality would imply that $e_g = 0$. \par 
Since $q_g = \frac{\kappa}{\nu}$, and $\nu = \frac{n_l}{1+e_g}$, substituting gives $q_g = \frac{\kappa(1+e_g)}{n_l}$. Thus
\[ q_g = (1+e_g)q_g' \]
This is totally consistent with what we just observed: $q_g$ is at all times some percentage bigger than $q_g$, the difference completely attributable to the amount of labor being alienated from workers to the capitalist class. Nonetheless we see that in some sense, Marx was justified in conflating his organic composition $q_g$ with this quantity - they are directly proportional. An increasing organic composition implies an increasing of the quantity $\frac{\kappa}{n_l}$, provided that the rate of exploitation stays constant. However, if the rate of exploitation were to simultaneously increase, this conflation would be in error. This is why the two equations for the general rate of profit are non-contradictory. In a sense, $q_g'$ is the technical ratio of capital per laborer. It is the organic composition as Marx discussed it, completely divorced from whatever exploitation is happening. \par
Nonetheless, Farjoun and Machover connect back to their earlier empirical evidence, namely the fact that from what they observe, $e_g$ is constant across developed capitalist economies and unchanging with time. Thus, in the context of our own world, it would seem that $q_g$ and $q_g'$ \textit{are in fact directly proportional}. Recall now the random variable defined over firm space, $Z = \frac{R}{K}$ - the proportion of wages relative to total capital of a firm. We saw then that $E(Z) = \frac{\pi(\mathbb{V})}{\pi(\mathbb{K})}$. Recall however that for large aggregates of commodities $\mathbb{B}$, $\pi(\mathbb{B}) \approx E(\Psi)\lambda(\mathbb{B})$. Since $\mathbb{V}$ and $\mathbb{K}$ are indeed enormous aggregates of commodities, we then have that
\[ E(Z) \approx \frac{\lambda(\mathbb{V})}{\lambda(\mathbb{K})} = \frac{\nu}{\kappa} = q_g \]
We mentioned earlier that the random variable $\frac{1}{Z}$ was analogous to Marx's organic composition of capital (plus $1$), and we now have, in a way, justified that claim, in the sense that $\frac{1}{E(Z)}$ is exactly (or rather, approximately) that. If we let $Q = \frac{1}{Z}$, we can then see it as the random variable measuring the organic composition of a firm. Note however that $Q$ is measured in price categories, not labor categories, and with the inversion of the $Z$ this produces an artifact in the sense that it is not the case that $E(Q) \neq \frac{1}{E(Z)}$. Since $Z$ was conjectured quite reasonably to have a gamma distribution, it follows that $Q$ has an \textit{inverse gamma distribution}. If the parameters of the gamma distribution are $\alpha$ and $\beta$, then these are also the parameters of the inverse gamma, though their role in the pdf and subsequently the mean and variance are different. In particular, the mean of an inverse gamma is $\frac{\beta}{\alpha-1}$, which is precisely $1$ over the \textit{mode} of the distribution of $Z$. However, since the expected value of a gamma distribution is $\frac{\alpha}{\beta}$, we have that
\[ q_g \approx \frac{1}{E(Z)} = \frac{\beta}{\alpha} \]
and that
\[E(Q) = \frac{\beta}{\alpha-1} \]
It is clear that despite $E(Q)$ not technically equaling $q_g$, it will still be very similar, at all times just a little big larger. (Which makes perfect sense, since relating the capital to labor in terms of prices produces distortion attributable to differences in markup due to exploitation, which due to exploitation deflates the wages relative to the labor actually done. Thus the denominator of the former will always be a bit smaller than what it should be, an artifact produced by exploitation.)  
\par In any case, the equation $r_g = \frac{1}{q_g'}\left(\frac{e_g}{1+e_g}\right)$ makes for a much cleaner argument in terms of what Marx was trying to show, and is consistent with Marx's thinking himself (although this is slightly attributable to a mistake). Since he conflated $q_g'$ with $q_g$, the argument he gave that $q_g$ grows without bound applies by his own reasoning to $q_g'$ as well, and as we have seen this is not entirely without merit. It makes for a cleaner overall argument that the profit rate will actually \textit{fall} over time, since it makes clear that $e_g$ cannot effectively compensate. However, Farjoun and Machover set up this argument in some sense just to knock it down. As is well known the empirical evidence for a falling rate of profit is very mixed and unclear, and Farjoun and Machover in fact argue that not only does the rate of profit not necessarily have a tendency to fall over time, but that it is in fact bounded. \par 
They first point out that $e_g$, as approximated to the more easily measurable $e_0$ (which recall is simply the ratio of profits to wages in a country), is more or less constant in developed countries, the same in all of them, and barely changes at all over time. Since it barely changes, it cannot be seen to influence the rate of profit much at all; $r_g$ depends entirely on either $q_g$ or $q_g'$. They note then that since there has been no observable downward trend over the last several generations, it could not be the case that $q_g$ or $q_g'$ have changed significantly. I personally find this argument not at all convincing, since if one is measuring profits in a particular country, they are likely only observing the wages of workers in that country, meaning that wages paid to employees of the firm that are working in \textit{other} countries are being ignored. Thus they seem to be missing the inflationary effect of, simply put, imperialism. \par 
Nonetheless, they go on to make an independent argument that, in lieu of simply the rate of profit being a non-uniform random variable, that it will not only not fall over a long period of time but in fact will be bounded. Their argument is worth understanding. It hinges on the simple observation that \textit{if the rate of profit is a random variable, then the health of the economy as a whole depends not on the average rate of profit, but on the distribution as a whole.} To illustrate this, they point to a table of different gamma distributions with different means but each with standard deviations which are half of that. For example, if $X \sim Gamma(4,20)$, then $E(X) = .2$, and $SD(X) = .1$, and if $Y \sim Gamma(4,40)$, then $E(Y) = .1$ and $SD(X) = .05$, so these represents a simple case of an economy which goes from a global rate of profit of $.2$ to a global rate of $.1$, a ten percent drop in profitability. Since a ten percent profit rate is still plenty good, an economist viewing the rate of profit strictly by its average wouldn't see this drop as particularly catastrophic, but the fact is that when at $.2$, $1.69$ percent of firms are operating at below a $5$ percent profit rate, but at $.1$, that number jumps to $14.3$ percent. That's more than $10$ percent of firms the economy suddenly on the brink of being unable to cover their debts. When the average rate of profit drops to $.05$ percent, the number below a $5$ percent profit rate jumps to $56.7$ percent. Their point is that the economy will be perceived to be in crisis \textit{well before} the average rate of profit is at a point where most economists will acknowledge that a crisis has arrived. The probabilistic approach also reveals how even in times of deep crisis, many firms in the economy will remain doing quite well for themselves. Farjoun and Machover, after making this observation, illustrate some of the reactive mechanisms which will begin occurring well before a crisis of the sort that an economist would start their thoughts at.\par 
By their earlier empirical observations, they are assuming that regardless of whatever else is going on, $e_g$ remains constant, and focus on inspecting $r_g$ through the variable $\frac{k}{n_l}$, which is inversely proportional to the rate of profit. First, as the rate of profit falls, less profitable firms go out of business, and their capital is written off. This reduces $k$. The crisis also stimulates competition, incentivizing well off firms to quickly adapt the innovations developed between the last crisis and the current one, increasing productivity of labor and thus the ratio of capital to labor. Both of these factors cause a decrease of $\frac{k}{n_l}$, and subsequently an increase of $r_g$. \par 
Conversely when the boom times have been sufficiently initiated, profits are flowing more rapidly than usual, and there is less incentive to innovate. Capitalists are reinvesting their capital in increasing amounts, but they are doing so in a way that merely scales production up without changing the method at all. This invokes the law of falling marginal productivity. For example, in the business of coal mining, less rich mines are brought into operation which require more labor to produce the same amount of coal. Thus the labor content of each ton of coal tends to rise. This happens across the entire economy, causing $\frac{k}{n_l}$ to increase, beginning the process of $r_g$ reversing course and eventually beginning to fall again. \par 
Farjoun and Machover clearly feel that, in lieu of their observation that the capitalist economy will be perceived by its inhabitants as in crisis well before the profit rate actually falls to a dramatic degree, this extremely standard description of the industrial cycle, and in particular the mechanisms which bounce $r_g$ back up again, trigger earlier than one would expect. This only serves to reinforce the more heavy lifting parts of their argument which allow them to conclude that the rate of profit is \textit{bounded}. Here is how the conclusion is reached: $e_g$ is unchanging, and thus $r_g$ depends entirely on $\frac{k}{n_l}$, which oscillates as described above by the opposing tendencies of innovation and quantitative expansion. Thus $r_g$ is bounded if and only if $\frac{k}{n_l}$ is bounded. Their argument for this was made earlier: empirical evidence suggesting that the rate of profit has not fallen, along with empirical evidence suggesting that $e_g$ does not change much at all, leads one to conclude that $q_g'$ remains unchanging as well. Thus the rate of profit is bounded. Casting aside the strange circularity of this argument, which effectively says that "the rate of profit is theoretically bounded because we have observed that it is bounded", I don't find it convincing at all, because the idea that $\frac{k}{n_l}$ would remain unchanging for the duration of capitalism's history appears as absurd to me. Not only that, I would wager any amount of money that if one were to account for the child laborers employed in the DRC mining minerals for the production of silicone chips manufactured in Taiwan, then the $e_g$ of Taiwan would be clearly observed to have fallen significantly after that exporting of capital. 
\section{The Equilibrium Model}
\subsection{The Consumption Matrix, Labor, and Value}
	Have $a_{ij}$ being the amount of commodity $i$ required to produce a single unit of commodity $j$. \textbf{THIS IS THE SAME CONVENTION MORISHIMA USES}. Value: $\vec{\lambda} = \begin{pmatrix} \lambda_1 \\ \lambda_2 \\ \vdots \\ \lambda_n \end{pmatrix}$ is the values of a unit amounts of each commodity. Clearly 
	\[ \lambda_i = a_{1i}\lambda_1 + a_{2i}\lambda_2 + \ldots + a_{mi}\lambda_m + l_i \]
Where $\vec{l} = \begin{pmatrix} l_1 \\ l_2 \\ \vdots \\ l_m \end{pmatrix}$ is the vector of living labor values for a unit of each commodity. Thus if 
\[ A = \begin{pmatrix} a_{11} & a_{12} & \ldots & a_{1m} \\
						a_{21} & a_{22} & \ldots & a_{2m} \\
						\vdots \\
						a_{m1} & a_{m2} & \ldots & a_{mm}  \end{pmatrix}
						= \begin{pmatrix} \vec{a}_1 & \vec{a}_2 & \ldots & \vec{a}_m \end{pmatrix} \]
Then 
\[ \vec{\lambda} = A^T\vec{\lambda} + \vec{l} \]
Again, \textbf{this is the same $A$ as the one that Morishima uses}. Clearly $\lambda$ represents the labor embodied in a unit of each commodity. But is the labor embodied in a unit of some commodity the same as the total gross labor required to create a single unit of some commodity? It should be. Let $\vec{x}_1 = \begin{pmatrix} x^1_1 \\ x^1_2 \\ x^1_m \end{pmatrix}$ be the gross bundle of commodity amounts required to create a single unit of commodity $1$. How to find this? To make a single unit of this first commodity, we must use raw materials, given by the $a$ coefficients of the matrix $A$. But these raw materials themselves must be assembled, and so forth, propagating backward forever. We must satisfy
\[ \begin{cases}
	& x_1^1 = a_{11}x_1^1 + a_{12}x_2^1 + \ldots + a_{1m}x_m^1 + 1 \\
	& x_2^1 = a_{21}x_1^1 + a_{22}x_2^1 + \ldots + a_{2m}x_m^1 + 0 \\
	& \vdots \\
	& x_m^1 = a_{m1}x_1^1 + a_{m2}x_2^1 + \ldots + a_{mm}x_m^1 + 0  
\end{cases} \]
Note the difference in subscripts on the $a$'s compared to the system of equations for $\lambda$. With $\lambda$, we accumulated. With $x$, we are taking away and making sure we still have enough. $\lambda_1$ is the labor from production of raw material $1$ required to make commodity $1$ plus the labor from the production of raw material $2$ required to make commodity $2$ plus $\ldots$. $x_1^1$, in contrast, we must use $a_{11}x_1^1$ of commodity $1$ in the production of raw material $1$, then use $a_{12}x_2^1$ of commodity $1$ in the production of raw material $2$, etcetera, and then after taking all of that away, still have a full unit left over. In vector form, these equations can be expressed
\[\vec{x}_1 = A\vec{x}_1 + \vec{e}_1 \]
Where $\vec{e}_1$ is the first standard basis vector in $\mathbb{R}^m$. If we let 
 \[ X = \begin{pmatrix} \vec{x}_1 & \vec{x}_2 & \ldots \vec{x}_m \end{pmatrix} \]
Then all of these can be expressed simultaneously in
\[ X = AX + I \]
More generally, if we have some desired net output $\vec{d}$, then the required gross bundle $\vec{x}$ is found by solving the equation
\[ \vec{x} = A\vec{x} + \vec{d} \]
Focusing for the moment on $X$ however, note that if $\vec{x}_i$ is the total gross commodity bundle required to produce a net single unit of commodity $i$, then $\vec{l} \cdot \vec{x}_1$ is the gross labor involved in the production of materials for it. Overall we may write
\[ X^T\vec{l} = \begin{pmatrix} \vec{x}_1^T \\ \vec{x}_2^T \\ \vdots \\ \vec{x}_m^T  \end{pmatrix}\begin{pmatrix} l_1 \\ l_2 \\ \vdots \\ l_m \end{pmatrix}
 = \begin{pmatrix} \vec{x}_1 \cdot \vec{l} \\ \vec{x}_2 \cdot \vec{l} \\ \vdots \\ \vec{x}_m \cdot \vec{l} \end{pmatrix} := \begin{pmatrix} \mu_1 \\ \mu_2 \\ \vdots \\ \mu_m \end{pmatrix} = \vec{\mu} \]
Thus we have two very similar vectors, $\vec{\mu}$ and $\vec{\lambda}$. $\vec{\mu}$, the vector of gross labor times required to produce a net unit of each commodity, and $\vec{\lambda}$, the vector of total labor "embodied" in a unit of each commodity. These ought to be the same, and indeed they are. To show this, we have three equations to work with
\begin{itemize}
	\item[(1)] $\vec{\lambda} = A^T\vec{\lambda}+\vec{l}$
	\item[(2)] $\vec{\mu} = X^T\vec{l}$
	\item[(3)] $X = AX + I$
\end{itemize}
Begin by multiplying both sides of the first equation by $X^T$ and regrouping:
\begin{align*}
	& X^T\vec{\lambda} = X^TA^T\vec{\lambda} + X^T\vec{l} \\
	&\implies X^T\vec{\lambda} - X^TA^T\vec{\lambda} = X^T\vec{l} \\
	&\implies (X^T-X^TA^T)\vec{\lambda} = X^T\vec{l}
\end{align*}
However, by the third equation, we have that $X-AX = I$, and so of course $X^T - X^TA^T = I$ as well, replacing the left-hand side with simply $\vec{\lambda}$. On the right hand side, we just have $X^T\vec{l}$, but by equation $1$ this is precisely $\vec{\mu}$. Thus we have that $\vec{\lambda} = \vec{\mu}$. \par 
Practically speaking, what this shows is that the value of a net product is equal to the total labor of the gross product used to create the net. To see this, let $\vec{d}$ be some net bundle of commodities. As we've seen, the gross bundle $\vec{x}$ satisfies 
\begin{align*} 
	& \vec{x} = A\vec{x} + \vec{d} \\
	&\implies \vec{x} = (I-A)^{-1}\vec{d}
\end{align*}
(We will momentarily discuss the assumptions that will be made of $A$, and these will guarantee, among other things, the existence of $(I-A)^{-1}$.) Likewise in general we have by equation $(1)$ above that $\vec{\lambda} = (I-A^T)^{-1}\vec{l}$. The of the net product is then $\vec{l} \cdot \vec{x}$. Substituting $\vec{x}$ for it's solution and simplifying gives:
\begin{align*}
	\vec{l} \cdot \vec{x} &= \vec{l}^T(I-A)^{-1}\vec{d} = (((I-A)^{-1})^T\vec{l})^T\vec{d}\\
		&= ((I-A^T)^{-1}\vec{l})^T\vec{d} = \vec{\lambda}^T \vec{d} \\
		&= \vec{\lambda} \cdot \vec{d}
\end{align*}
So in general we have that if $\vec{d}$ is a net product, and $\vec{x}$ is the gross product used for the creation of the net, then 
\begin{align}
	\vec{\lambda}\cdot \vec{d} = \vec{l} \cdot \vec{x}
\end{align}
Before continuing further, we should pause and figure out the assumptions that must be made of the matrix $A$ in order to have a meaningful labor theory of value. First, it is obviously the case that $a_{ij} \geq 0$ for all $i,j \leq m$. If this weren't the case, then the second law of thermodynamics would be broken. Second, we assume that $\vec{l}$ has entries which are strictly nonnegative. Having them be strictly positive is obviously stronger, but perhaps not as realistic, and the choice of which to go with will weaken or strengthen the conditions required of $A$. We will consider both cases. However these conditions of nonnegativity are not sufficient to ensure the nonnegativity of the unit value vector $\vec{\lambda}$. For instance, suppose that 
\[ A = \begin{pmatrix} .5 & .6 \\ .4 & .7 \end{pmatrix} \hspace{2cm} \vec{l} = \begin{pmatrix} .1 \\ .1 \end{pmatrix} \]
Then it can be seen that we end up with the solution $\vec{\lambda} = \begin{pmatrix} -1 \\ -1 \end{pmatrix}$. What went wrong here? It's difficult to see currently, but the issue will turn out to be that one of these commodity types is not technically feasible to produce; i.e. there is no gross amount of a certain commodity which will be sufficient to produce a single net unit of that commodity. It turns out then that the conditions for $\vec{\lambda}$ to be nonnegative are completely harmless - since they happen only when an economy is completely unfeasible in the first place. \par 
 Thus we begin by considering the conditions not for $\vec{\lambda}$ to be positive or even nonnegative, but rather the conditions required for a viable economy. To have such a thing it must be the possible for the economy to produce any positive net output from a nonnegative gross input. That is to say, if $\vec{d}$ is a given demand bundle, it needs to be the case that there exists a gross bundle $\vec{x}$ with positive entries such that 
\[ \vec{x} = A\vec{x} + \vec{d} \]
If $A$ were a matrix such that this was claim was true, then any fixed $\vec{d}$ with all entries greater than zero, we would have a gross bundle $\vec{x}^0$ which satisfies the above, and in that case it would certainly have to be true that 
\[ \vec{x}^0 > A\vec{x}^0 \]
where $>$ here means that every entry of the left-hand vector is greater than the corresponding entry of the right-hand vector. The existence of such an $\vec{x}^0$, without mention of any net demand vector $\vec{d}$, is clearly then a necessary condition for a viable economy. We will call such matrices \textit{productive}.
\begin{definition}
	A matrix $A$ is productive if there exists a $\vec{x}^0$ with strictly positive entries such that all entries of $\vec{x}^0$ are greater than those of $A\vec{x}^0$ 
\end{definition}
Surprisingly, that $A$ be productive is not just a necessary condition, but also sufficient, as the following characterization theorem:
\begin{theorem}
	Suppose $A$ is an $m\times m$ matrix with nonzero entries. Then $A$ is productive if and only if $I-A$ is invertible, and $(I-A)^{-1}$ has only nonnegative entries.
\end{theorem}
\begin{proof}
	We begin with (i) $\implies$ (ii). First, suppose that $A$ is a productive matrix with nonnegative entries, and let $\vec{x}$ be the strictly positive vector witnessing it's productivity, that is, the entries of $\vec{x}$ are greater than those of $A\vec{x}$. Suppose by way of contradiction that $I-A$ is not invertible. Then $dim(Nul(I-A)) > 0$, so there must exist a $\vec{y} \neq \vec{0}$ such that $(I-A)\vec{y} = \vec{0}$. But then $\vec{y} = A\vec{y}$, i.e. $\vec{y}$ is an eigenvector of $A$ with associated eigenvalue $\lambda = 1$. (This isn't really pertinent to the proof, it's just interesting that $I-A$ being invertible is equivalent to saying that $1$ is not an eigenvalue of $A$.) We may assume without loss of generality that at least one of the coordinates for $\vec{y}$ is positive (if that is not the case, simply take $-\vec{y}$). Thus the value $\sup\{\frac{y_i}{x_i}\} = c > 0$. Suppose $k$ is the coordinate which produces this supremum, so $\frac{y_k}{x_k} = c$. Note that
	\[ c(\vec{x} - A\vec{x})_k = c(x_k-\sum_{i=1}^ma_{ki}x_i) = cx_k - \sum_{i=1}^ma_{ki}cx_i \]
Also we have that
	\[cx_k = y_k = (A\vec{y})_k = \sum_{i=1}^m a_{ki}y_i  \]
Thus 
\begin{align}
	c(\vec{x} - A\vec{x})_k = \sum_{i=1}^m a_{ki}(y_i - cx_i) \label{prodCont}
\end{align} 
Now we know that the left-hand side of \ref{prodCont} is strictly greater than $0$, by virtue of $\vec{x}$ being positive in all entries, $\vec{x} > A\vec{x}$ and $c > 0$. However, now look at the right-hand side. If $i=k$, then $y_i-cx_i = y_i-y_i = 0$, so that term of the sum is $0$. Otherwise, $i \neq k$. Suppose this is the case and that $y_i - cx_i > 0$. Then 
\[  y_i > cx_i = \frac{y_k}{x_k}x_i \implies \frac{y_i}{x_i} > \frac{y_k}{x_k} \]
a contradiction, since $\frac{y_k}{x_k}$ is supposed to be the biggest of these ratios. Thus $y_i - cx_i \leq 0$ for all $i = 1,...,m$, and thus given that all entries of $A$ are nonnegative, the right-hand side must be $\leq 0$. Thus \ref{prodCont} cannot possibly be the case, and we have our contradiction. $I-A$ must thus be invertible. \par 
Now consider the entries of $(I-A)^{-1}$. We wish to show that these are nonnegative. Suppose it has a negative entry. Then of course this would imply the existence of a strictly positive $\vec{z}$ such that $\vec{y}=(I-A)^{-1}\vec{z}$ has a negative entry as well. This ensures that $\sup\{-\frac{y_i}{x_i}\} = c$ is $> 0$ (with $\vec{x}$ still the same as in the previous part of the proof - the vector witnessing $A$'s positivity). Suppose $k$ is the index witnessing this supremum. Now 
\[ c(\vec{x} - A\vec{x})_k = c(x_k - \sum_{i=1}^m a_{ki}x_i) = -y_k - \sum_{i=1}^m a_{ki} cx_i \] 
Now since $\vec{y} = (I-A)^{-1}\vec{z}$, we have that $\vec{z} = (I-A)\vec{y} = \vec{y} - A\vec{y}$, and so 
\[ z_k = y_k - \sum_{i=1}^m a_{ki} y_i \]
Adding these leaves us with
\begin{align}
	c(\vec{x} - A\vec{x})_k+z_k = -\sum_{k=1}^m a_{ki}(y_i+cx_i) \label{prodCont2}
\end{align}
Now considering the right-hand side of \ref{prodCont2} first, fix an $i$ and look at the term $y_i+cx_i$. $c$ and $x_i$ are both greater than $0$, and by definition $-\frac{y_i}{x_i} \leq c \implies -y_i \leq cx_i \implies y_i \geq -cx_i$, so then $y_i+cx_i \geq 0$. This, along with entries of $A$ being nonnegative ensures that the sum here is nonnegative, and thus the right-hand side is necessarily nonpositive. But then we have
\[ c(\vec{x} - A\vec{x})_k+z_k \leq 0 \implies z_k \leq -c(\vec{x} - A\vec{x})_k \]
But $c > 0$ and $(\vec{x} - A\vec{x})_k > 0$, so $-c(\vec{x} - A\vec{x})_k < 0$, meaning that $z_k < 0$. But this is a contradiction since $\vec{z}$ was supposed to be strictly positive. Thus $(I-A)^{-1}$ cannot have negative entries. \par 
Finally, it is near trivial to see that if $(I-A)^{-1}$ exists and has nonnegative entries, then $A$ is productive. Simply let $\vec{y}$ be any strictly positive vector in $\mathbb{R}^m$, and consider $\vec{x}^0 = (I-A)^{-1}\vec{y}$. This obviously must be nonnegative. But then 
\[ \vec{x}^0 - A\vec{x}^0 = (I-A)\vec{x}^0 = \vec{y} \]
which was strictly positive by definition. Thus $\vec{x}^0 > A\vec{x}^0$, and we have found our vector witnessing that $A$ is productive. 
\end{proof}
\begin{theorem}
	 If $A$ is productive, then the series $I+A+A^2+A^4 + \ldots$ converges in each entry, and in fact converges to $(I-A)^{-1}$ 
\end{theorem}
\begin{proof}
	Suppose that $A$ is productive, and let $\vec{x}$ be the vector such that $\vec{x} > A\vec{x}$. Then there exists a constant $c \in (0,1)$ such that $c\vec{x} > A\vec{x} \geq \vec{0}$. Multiplying all sides of this by $A$ we obtain $cA\vec{x} > A^2\vec{x} \geq \vec{0}$. But then 
\[ c(c\vec{x}) > c(A\vec{x}) > A^2\vec{x} \geq \vec{0} \]
And so we obtain
\[ c^2\vec{x} > A^2\vec{x} \]
Multiplying both sides by $A$ again and comparing with $c^3\vec{x}$ yields similarly that $c^3\vec{x} > A^3\vec{x}$, and so forth. Thus we have that for all $n \in \mathbb{N}$, that 
\[ c^n \vec{x} > A^n\vec{x} \geq \vec{0} \] 
But of course since $0 < c < 1$, $c^n \to 0$ as $n \to \infty$, and so the limit of $c^n\vec{x}$ is $\vec{0}$. It follows then that $A^n\vec{x}$ must go to $\vec{0}$ as $n \to \infty$. From this we have that $A^n$ converges in each entry. Moreover for any $i \in \{1,...,m\}$, we must have
\[ \lim_{n\to\infty} a^n_{ij}x_j = 0 \]
where $a^n_{ij}$ denotes the $i,j$ entry of $A^n$ (not necessarily the $n^{th}$ power of $A^n$). Since all of the entries of $x_j$ are positive, the only way for this equation to hold is for $a^n_{ij} \to 0$ for all $i,j$, as $n \to \infty$. \par 
To show that the series itself converges to $(I-A)^{-1}$, let $B_n = I+A+A^2+\ldots + A^n$. Multiplying this identity with $A$ gives that $AB_n = A+A^2+A^3 + \ldots + A^{n+1}$. Subtracting the second equation from the first gives
\begin{align*}
	& B_n - AB_n = I-A^{n+1} \\
	&\implies B_n(I-A) = I-A^{n+1} \overset{n}\to I
\end{align*}
Thus $\lim_{n\to\infty}B_n = (I-A)^{-1}$. 
\end{proof}
The implications of this are very revealing. For any vector of net demand $\vec{d}$, we know that $(I-A)^{-1}\vec{d}$ represents the bundle of gross commodities required to produce $\vec{d}$. Intuitively we would expect that this vector is $\vec{d}$ itself, plus the materials required to make $\vec{d}$, plus the materials required to make the materials required to make $\vec{d}$, plus the materials required to make \emph{those} materials, plus $\ldots$. This amounts to the chain 
\[ \vec{d} + A\vec{d} + A^2\vec{d} + A^3\vec{d} + \ldots \]
and we can now see that provided that $\vec{d}$ always has a solution, this is exactly what $(I-A)^{-1}$ represents.  \par 
With this theorem it follows that the property of being productive is sufficient for any net demand vector to have an associated nonnegative gross bundle, since given a $\vec{d}$, we have that $\vec{x} = (I-A)^{-1}\vec{d}$ not only exists and is unique, but in fact is also strictly nonnegative, since the entries of $(I-A)^{-1}$ and $\vec{d}$ are nonnegative. Of course, we have already seen that $A^T$ is just as valuable as $A$. The following theorem is thus very useful:
\begin{theorem}
	The transpose of a productive matrix is productive
\end{theorem}
\begin{proof}
	If $A$ is productive, then $(I-A)^{-1}$ exists and has nonnegative entries. A matrix is invertible if and only if it's transpose is invertible. Thus $(I-A)^T = I-A^T$ is invertible, and in fact we have $(I-A^T)^{-1} = ((I-A)^T)^{-1} = ((I-A)^{-1})^T$. Hence not only is $I-A^T$ invertible, but is is none other than the transpose of $(I-A)^{-1}$. Since this matrix has nonnegative entries, so must it's transpose, and so we now can be sure that $A^T$ is productive. 
\end{proof}
Another sufficient condition for productivity is the following:
\begin{theorem}
	If the row sums of a nonnegative matrix $A$ are all less than $1$ (row sum meaning the sum of the entries along a row), then the matrix $A$ is productive. Also if the column sums of $A$ are all less than $1$, then $A$ is productive.
\end{theorem} 
\begin{proof}
	Suppose that the row sums are all less than $1$. Let $\vec{x} = \begin{pmatrix} 1 \\ 1 \\ \vdots \\ 1 \end{pmatrix}$. Then $i^{th}$ entry of $A\vec{x}$ is precisely $i^{th}$ row sum. But since the row sums are less than $1$, we have that the entries of $A\vec{x}$ are all smaller than those of $\vec{x}$, and so $A$ is productive. \par 
	Now suppose that the column sums are all less than $1$. Then by the above it follows that $A^T$ is productive. Thus $(I-A^T)^{-1}$ exists and has strictly nonnegative entries. But $(I-A^T) = (I-A)^T$, and a matrix is invertible iff it's transpose is invertible, so $((I-A)^T)^T = I-A$ is invertible. It follows then that $(I-A^T)^{-1} = ((I-A)^T)^{-1} = ((I-A)^{-1})^T$. Thus the fact that $(I-A^T)^{-1}$ has nonnegative entries means that $((I-A)^{-1})^T$ also has nonnegative entries, and so of course this means that $(I-A)^{-1}$ itself has nonnegative entries. Thus $I-A$ is invertible and the inverse is nonnegative, which we know is equivalent to $A$ being productive. 
\end{proof}
 Now we show that a productive consumption matrix is sufficient for $\vec{\lambda}$ to be nonnegative. Suppose we have a productive consumption matrix $A$ and that $\vec{l}$ is nonnegative. Then the vector of values for a unit of each commodity type is given
 \[ \vec{\lambda} = A^T\vec{\lambda} + \vec{l} \]
 We know now that if $A$ is productive then so is $A^T$, and equivalently that $(I-A^T)^{-1}$ exists and has nonnegative entries. Thus $\vec{\lambda} = (I-A^T)^{-1}\vec{l}$ is unique, and nonnegative. However simply being nonnegative isn't good enough for our purposes. There is nothing here precluding that $\vec{\lambda} = \vec{0}$, even if $\vec{l}$ itself is nonzero! We will return to this in a moment. \par 
 Before that however, we should consider one more thing. We have that productivity of the matrix $A$ and nonnegative values for the living labor vector $\vec{l}$ are certainly two of the most minimal of conditions required to have a reasonable capitalist (or socialist, for that matter) economy, and these condition also serve as sufficient conditions for a nonnegative $\vec{\lambda}$. For a capitalist society to be viable however, we also need each industry to be \textit{profitable}. Let $\vec{p} = \begin{pmatrix} p_1 \\ p_1 \\ \vdots \\ p_m  \end{pmatrix}$ be the unit price vector, where $p_i$ denotes the price in "dollars" (whatever the currency denomination of our hypothetical economy) of a single unit of commodity $i$. Consider the price required to produce a unit of commodity $1$. The price of the materials will be
\[ p_1a_{11} + p_2a_{21} + \ldots + p_ma_{m1} \]
Which is of course the first coordinate of the vector $A^T\vec{p}$. Also required is living labor in the amount $l_1$. This labor is measured in units of time, say hours. Let us specify a new parameter for the model, the hourly wage rate $w$, which is the price in dollars for an hour of labor, which we will assume is the same for the entire economy. Then the additional cost required in assembling the materials is $wl_1$, the first coordinate of $w\vec{l}$. The vector denoting the cost of production for a unit of each commodity is thus
\[ A^T\vec{p} + w\vec{l} \]
Assuming supply and demand are in equilibrium and every product has a customer willing to buy at the standard price, we have then that in order for each commodity industry to profit from production, it must be the case that
\[ \vec{p} > A^T\vec{p} + w\vec{l} \label{profit} \]
Suppose this is the case. We will deal with the wage rate later. For now, we are only interested in the necessary conditions for a viable economy independent of anything besides the consumption matrix and living labor vector - the truly technical coefficients of the system. If \ref{profit} is true, then it must be the case that
\[ \vec{p} > A^T\vec{p} \]
Thus $\vec{p}$ must witness that $A^T$ is productive. To conclude the discussion of $A$, we have identified productivity as the necessary and sufficient condition both for labor values to be well defined and nonnegative, and for net demand bundles to always be producible by a corresponding gross bundle. It is striking that the same condition is equivalent to both of these. We've also identified productivity of $A$ as a necessary condition for a viable \textit{capitalist} economy, in which every industry can profit simultaneously. Necessary, but not sufficient, as we will see later. \par 
Unfortunately, one more condition must be added to the matrix $A$ to ensure a positive $\vec{\lambda}$, unless we are willing to restrict the scope of our to commodity types which require positive amounts of living labor. With the knowledge that a matrix is productive if and only if it's transpose is productive, we can now identify the problem in a more general sense; it is now the case that for a nonnegative $\vec{l}$, ensuring $\vec{\lambda}$ positive is equivalent to ensuring that the solution to $\vec{x} = A\vec{x} + \vec{d}$ is \textit{always} positive, as long as $\vec{d}$ is nonnegative. \par
In terms of having a minimal set of assumptions for our model, this condition is absurd, since it would imply that every industry requires positive input from every other industry in order to produce, and a car making factory probably doesn't need any toothbrushes to assemble a car. To express the \emph{minimal} condition for a positive $\vec{\lambda}$ is going to require us to divide up our economy into at least two departments - a capital goods industry and a wage goods industry, and apply a condition only to the former. Before this, however, we find the property generally for $A$. \par 
Partition the industries $(1,2,...,m)$ into two groups, $(g_1,g_2,...,g_k)$, and $(g_{k+1},g_{k+2},\ldots,g_m)$. If either of these groups is able to produce it's commodities without any input from the other group, then we call it an independent subgroup. The overall set of industries $(1,2,...,m)$ is said to be \textbf{irreducible}, or \textbf{indecomposable}, if there are no independent subgroups. This is the condition on industry that we are looking for, and we want to express it as a property of the matrix $A$. Suppose we arbitrarily partitioned $A$ into a block matrix representation
\[ A = \begin{pmatrix} A_{11} & A_{12} \\ A_{21} & A_{22} \end{pmatrix} \]
with $A_{11}$ and $A_{22}$ being square. Suppose industries are irreducible, and consider a desired strictly positive output vector $\vec{d} = \begin{pmatrix} \vec{d}_1 \\ \vec{d}_2 \end{pmatrix} > 0$.  If $\vec{x}_2$ is positive, then irreducibility of industries would require that the second subindustry requires input from the first to produce the net output of $\vec{x}_2$ - therefore we must have $A_{21}\vec{x}_1 \neq \vec{0}$, i.e. it cannot be the case that $A_{21}$ is entirely $0$. Furthermore it cannot be the case that this corner submatrix is $0$ however we choose to permute the industries. A rearrangement of industries amounts to a change of basis which switches out coordinate locations, which can be accomplished by conjugation by a permutation matrix $P$. This is how we arrive at the following definition:
\begin{definition}
	A matrix $A$ is \textbf{irreducible} if, for any permutation matrix of matching dimension $P$, it is never the case that
	\[ P^TAP = \begin{pmatrix} A'_{11} & A'_{12} \\ O & A'_{21} \end{pmatrix} \]
where $0$ denotes a $0$ matrix of some dimension, and $A'_{11}$ and $A_{22}$ are square. 
\end{definition}
Other equivalent definitions of irreducibility exist, including a more rigorous formulation of the claim made about industries. One of the more interesting ones involves a statement about the directed graph associated with $A$, $G(A)$. This is the graph which has $m$ nodes (where $A$ is $m\times m$, and there exists a edge from node $i$ to node $j$ iff $a_{ij} > 0$. We say that a directed graph is \emph{strongly connected} if for any two nodes, there exists a finite path (i.e. a sequence of edged) from $i$ to $j$.
\begin{theorem}
	The following are equivalent:
	\begin{itemize}
		\item[(i)] $A$ is irreducible.
		\item[(ii)] The graph $G(A)$ is strongly connected.
		\item[(iii)] For any specification of a nondiagonal entry of dimension compatible with $A$ $(i,j)$, there exists a power $n$ such that the $i,j$ entry of $A^n$ is positive.
		\item[(iv)] There is no subspace of $\mathbb{R}^m$ spanned by a proper subset of the standard basis vectors which is closed under multiplication by $A$.  
	\end{itemize} 
\end{theorem}
\begin{proof}
	To do. 
\end{proof}
\begin{fact}
	$A$ is irreducible if and only if $A^T$ is irreducible.
\end{fact}
\begin{proof}
	To do.
\end{proof}
The following theorem confirms that this property is what we are looking for:
\begin{theorem}
	Let $A$ be a nonnegative matrix. Then the following are equivalent:
	\begin{itemize}	
		\item[(i)] For any nonnegative nonzero vector $\vec{d}$, the equation $\vec{x} = A\vec{x}+\vec{d}$ has a positive solution.
		\item[(ii)] The matrix $(I-A)^{-1}$ exists and is strictly positive. 
		\item[(iii)] The matrix $A$ is productive and irreducible. 
	\end{itemize}
\end{theorem}
\begin{proof}
	For (i) $\implies$ (ii), assume (i), and let $\vec{d}$ be a strictly positive vector. Then by hypothesis there exists a strictly positive solution $\vec{x}$ such that $\vec{x} = A\vec{x} + \vec{d}$. But then of course this implies immediately that $\vec{x} > A\vec{x}$, demonstrating that $A$ is productive. It follows that $(I-A)^{-1}$ exists. We must show next that it is positive. Fix a column $i$, and fix $\vec{d} = \vec{e}_i$. The solution $\vec{x}_i$ for this is then given by $\vec{x}_i = (I-A)^{-1}\vec{e}_i$. But this is precisely the $i^{th}$ column of $(I-A)^{-1}$, and so by identity with our strictly positive $\vec{x}_i$ we have that the column is strictly positive. \par 
	(ii) $\implies$ (i) is nearly trivial. Suppose $(I-A)^{-1}$ is strictly positive and denote the entries $b_{ij}$. Consider $\vec{d}$ an arbitrary nonnegative and nonzero vector. Then $\vec{x} = A\vec{x} + \vec{d} \implies \vec{x} = (I-A)^{-1}\vec{d}$ is the unique solution for $\vec{x}$. Note that for a fixed $i$, we have
	\[ x_i = \sum_{j=1}^m b_{ij}d_j \]
But presuming that $\vec{d}$ is positive for at least one entry, this sum is positive. Thus $\vec{x}$ is strictly positive. \par 
For (ii) $\implies$ (iii), if $(I-A)^{-1}$ exists and is positive, then $A$ is automatically productive. Furthermore we know this means specifically that
\[ (I-A)^{-1} = \lim_{n \to \infty} \sum_{k=0}^n A^k \]
But if $(I-A)^{-1}$ is positive then from the above identity it eventually has to be the case that any entry of these partial sums must eventually become positive, meaning that for any $i,j$ there exists an $n$ such that $(A^n)_{ij} > 0$. But by our above lemma this means that $A$ is irreducible. \par 
Finally we show that (iii) $\implies$ (ii), completing a chain. Suppose that $A$ is productive and irreducible. Then productivity implies that $(I-A)^{-1} = I+A+A^2+\ldots$ exists and is nonnegative, and furthermore irreducibility implies that for any $i \neq j$, $(A^n)_{ij}$ is eventually positive, from which it follows that $((I-A)^{-1})_{ij}$ is positive. Since the diagonal entries are automatically positive from $I$ in the summand, it follows that $(I-A)^{-1}$ is positive in all entries. 
\end{proof}
From this it follows that $A$ being productive and irreducible is sufficient for $\vec{\lambda}$ to be strictly positive, even if $\vec{l}$ is $0$ for some industries. However this is a very strong condition on $A$ and should be weakened for a more realistic model. We can weaken the condition by splitting the economy into two subeconomies. Let the first $n$ many industries denote the \textbf{capital goods} industries, while the remaining $m-n$ industries will denote the \textbf{wage goods} industries (which can include luxury goods). Capital goods are those goods which are not consumed at all by workers of capitalists - they are only used for the production of goods, either capital or wage. Wage goods, meanwhile, are \emph{never} needed for the production of \emph{any} commodity. That is to say
\[ \forall i > n (a_{ij} = 0) \]
Thus the final $m-n$ rows of the matrix $A$ are assumed to be zero rows. 

 Wage goods are defined as such by having the property that they are strictly for consumption and not for production. This means that for any $j$, we will have $a_{ij} = 0$ for all $i>n$. We can then split our matrix $A$ into the block matrix
\[ A = \begin{pmatrix} A_1 & A_2 \\ O_1 & O_2 \end{pmatrix} \]
where
\[ A_1 = \begin{pmatrix} a_{11} & a_{12} & \ldots & a_{1n} \\
			a_{21} & a_{22} & \ldots & a_{2n} \\
			\vdots \\ a_{n1} & a_{n2} & \ldots & a_{nn} \end{pmatrix} \hspace{2cm} A_2 = \begin{pmatrix} a_{1(n+1)} & a_{1(n+2)} & \ldots & a_{1m} \\
			a_{2(n+1)} & a_{2(n+2)} & \ldots & a_{2m} \\
			\vdots \\ a_{n(n+1)} & a_{n(n+2)} & \ldots & a_{nm} \end{pmatrix} \]
with $O_1$ and $O_2$ being the $(m-n)\times n$ and $(m-n) \times (m-n)$ $0$ matrices respectively. This great divide also splits our value and living labor vectors into the block forms:
 \[ \vec{\lambda}_1 = \begin{pmatrix} \lambda_1 \\ \lambda_2 \\ \vdots \\ \lambda_n \end{pmatrix} \hspace{2cm} \vec{\lambda}_2  = \begin{pmatrix} \lambda_{n+1} \\ \lambda_{n+2} \\ \vdots \\ \lambda_m \end{pmatrix} \hspace{2cm} \vec{l}_1  = \begin{pmatrix} l_1 \\ l_2 \\ \vdots \\ l_n \end{pmatrix} \hspace{2cm} \vec{l}_2  = \begin{pmatrix} l_{n+1} \\ l_{n+2} \\ \vdots \\ l_m \end{pmatrix} \]
So that 
\[ \vec{\lambda} = \begin{pmatrix} \vec{\lambda}_1 \\ \vec{\lambda}_2 \end{pmatrix}  \hspace{2cm} \vec{l} = \begin{pmatrix} \vec{l}_1 \\ \vec{l}_2 \end{pmatrix} \]
The right side of the determining equation $\vec{\lambda} = A^T\vec{\lambda}+\vec{l}$ then disaggregates to
\begin{align*}
	\begin{pmatrix} A_1^T & O_1^T \\ A_2^T & O^T_2 \end{pmatrix} \begin{pmatrix} \vec{\lambda_1} \\ \vec{\lambda_2} \end{pmatrix} + \begin{pmatrix} \vec{l}_1 \\ \vec{l_2} \end{pmatrix} = \begin{pmatrix} A_1\vec{\lambda}_1 +\vec{l}_1 \\ A_2^T\vec{\lambda}_1 + \vec{l}_2 \end{pmatrix}
\end{align*} 
So that our single value determining equation becomes equivalent to the two equation system
\[ \vec{\lambda}_1 = A^T_1\vec{\lambda}_1+\vec{l}_1 \]
\[ \vec{\lambda}_2 = A^T_2\vec{\lambda}_1+\vec{l}_2 \]
The unit price vector also disaggregates:
\[ \vec{p} = \begin{pmatrix} \vec{p}_1 \\ \vec{p}_2 \end{pmatrix} \]
so that the condition for profit, that $\vec{p} > A^T\vec{p}+w\vec{l}$, also disaggregates. Just like before, the right hand side becomes
\[ \begin{pmatrix} A_1^T & O^T_1 \\ A_2^T & O^T_2  \end{pmatrix} \begin{pmatrix} \vec{p}_1 \\ \vec{p}_2 \end{pmatrix} + \begin{pmatrix} w\vec{l}_1 \\ w\vec{l}_2 \end{pmatrix} = \begin{pmatrix} A_1^T\vec{p}_1 + w\vec{l}_1 \\ A_2^T\vec{p}_1 + w\vec{l}_2 \end{pmatrix} \]
so that we are left equivalently with the two sets of conditions
\[  \vec{p}_1 > A_1^T\vec{p}_1 + w\vec{l}_1 \]
\[  \vec{p}_2 > A_2^T\vec{p}_1 + w\vec{l}_2 \]
Obviously we must keep the condition that both matrices $A_1$ and $A_2$ are non-negative, along with non-negativity for both $\vec{l}_1$ and $\vec{l}_2$. Assume for a moment though that \emph{only} $A_1$ is productive. Then $\vec{\lambda}_1$ can be solved for, and written uniquely as $\vec{\lambda}_1 = (I-A_1^T)^{-1}\vec{l}_1$, and this solution can be plugged into the second equation to derive $\vec{\lambda}_2$, giving us the unique solution
\[ \vec{\lambda}_1 = (I-A_1^T)^{-1}\vec{l}_1 \]
\[ \vec{\lambda}_2 = A_2^T(I-A_1^T)^{-1}\vec{l}_1 + \vec{l}_2 \]
(Verifying dimensionality, note that $(I-A_1^T)^{-1}\vec{l}_1$ is $n\times 1$, and $A_2^T$ is $(m-n) \times n$, so that the product is $(m-n)\times 1$, compatible with addition of $\vec{l}_2$.) Clearly productivity for $A_1$ gives us that $\vec{\lambda}_1$ is non-negative and unique, as well as a unique solution for $\vec{\lambda}_2$. It is also clear from this that $\vec{\lambda}_2$ is non-negative here as well. \emph{Thus, the assumption of productivity for $A$ only actually needs to be assumed for $A_1$.} We can immediately also see from our characterization theorem for irreducible and productive matrices that if just $A_1$ is productive and irreducible, then $\vec{\lambda}_1$ is strictly positive, but still not quite good enough to ensure $\vec{\lambda}_2$ is positive. Fix an $i$, and consider the solution for the $i^{th}$ entry of $\vec{\lambda}_2$:
\[ \lambda^2_i = \sum_{j=1}^n a^2_{ij}\lambda^1_j + l^2_i \]
Thus for this term to be ensured positive, we must have that \emph{either} $a^2_{ij} > 0$ for some $j$, \emph{or} that $l_i > 0$. This condition can be condensed in matrix form to the condition that $\begin{pmatrix} A^T_2 \\ \vec{l}_2^T \end{pmatrix}$ has no column of entirely $0$'s, or equivalently that $\begin{pmatrix} A_2 & \vec{l}_2 \end{pmatrix}$ has no row of entirely $0$'s. Thus the true minimal conditions which must be made for a meaningful labor theory of value is the following:
\begin{itemize}
	\item[(1)] $A_1$, $A_2$, $\vec{l}_1$, and $\vec{l}_2$ are non-negative.
	\item[(2)] $A_1$ is productive and irreducible
	\item[(3)] The matrix $\begin{pmatrix} A_2 & \vec{l}_2 \end{pmatrix}$ has no $0$ rows.
\end{itemize} 
Because Morishima wants to stick to these minimal conditions, his entire model is developed with this two department breakdown. This leads to a lot of unclear and overly complicated formulas however. Thus for the sake of having simpler formulas, we will for time being we will stick to a single sector economy with no distinction between wage and capital goods, and in this setting we will simply assume that the entire matrix $A$ is non-negative, productive, and irreducible. We will however also periodically return to this more granular formulation when it is appropriate.
\par Before finally moving on, there is a theorem about positive matrices, some of whose results can be extended to nonnegative irreducible matrices, which will be extremely important later on.
\begin{theorem}[Perron-Frobenius Theorem]
	The actual theorem is pretty long and has a lot of details that we don't care about. I'll only state the parts that matter to this theory. Let $A$ be either a positive matrix, or a nonnegative irreducible matrix. Then there exists a positive eigenvalue equal to the spectral radius of $A$ (that is, the maximum magnitude out of all of the eigenvalues of $A$), which has an associated eigenvector $\vec{x}$, all of whose entries are positive. Furthermore the eigenspace associated with this eigenvalue is one dimensional; Any other eigenvector of associated with this largest eigenvalue is a scalar multiple of $\vec{x}$. Finally, the \emph{only} other strictly positive eigenvectors of $A$ are the ones in this one-dimensional eigenspace. 
\end{theorem}
\subsection{Exploitation}
In order to define exploitation, the critical ingredient for Morishima is the means of subsistence. This is a bundle of goods $\vec{b} = \begin{pmatrix} b_1 \\ \vdots \\ b_m \end{pmatrix}$ representing the amount of each commodity that a worker needs to consume each day to reproduce their labor power for the next day. The assumption of such a thing is in some sense definite and rigorous, and in another sense nonsensical. The former, in that there \textit{does} exist an average consumption of each commodity across all workers. The latter, in the sense that the variance of this consumption will vary wildly across those workers. Nonetheless, it establishes a baseline for what workers can be paid - a \textit{real} minimum wage. Note that in the context of an economy divided into capital and wage goods sectors, we have 
\[ \begin{pmatrix} \vec{0} \\ \vec{b}_2 \end{pmatrix} \] \par
Since workers do not consume any capital goods by our definition of them. Suppose a worker is paid precisely enough money to purchase the bundle $\vec{b}$, and in exchange for this labor they work for $T$ hours a day. The bundle has a labor value, given by $\vec{\lambda} \cdot \vec{b}$. Let $\omega = \frac{1}{T}$. Then each hour, the worker will receive pay equal to this fraction of the means of subsistence, which has value $\omega \vec{\lambda}\cdot \vec{b}$. The difference $1-\omega\vec{\lambda}\cdot \vec{b}$ then, represents the amount of each hour that the worker offers as tribute to the capitalist without compensation. The rate of exploitation can then be defined
\[ e = \frac{1-\omega\vec{\lambda}\cdot \vec{b}}{\omega\vec{\lambda}\cdot \vec{b}} \]   
that is, the ratio of "unpaid to paid labor", for a worker laboring at the minimal standards of society. Note that despite this rate clearly being a global one, a descriptor of society as a whole, it is defined in terms of an individual. Moreover this individual likely does not even exist, as the average bundle consumed daily by a worker varies wildly from person to person, so much so that the average is likely to apply to nobody at all. The rate can be derived in two other ways. \par 
For the first such identity, let $\bar{N}$ be the total number of workers. Thus, each day, $\bar{N}\vec{b}$ is the total bundle of commodities that needs to be produced in a stable system. We can imagine each worker producing their respective portion of this number, as we did above, but just as easily we can imaging a portion of these workers, $N < \bar{N}$, that produces just this bundle, and the other workers producing the surplus. The $N$ "necessary" workers do a collective $TN$ hours of labor, while the number of total hours worked by everyone is $T\bar{N}$. $T\bar{N} - TN$ is thus the total surplus, and we can define another rate of exploitation as the ratio of this surplus labor to the total:
\[ e^* = \frac{T\bar{N} - TN}{T\bar{N}} = \frac{\bar{N} - N}{\bar{N}} \]
We can show that this is in fact an identity for $e$. To see this, note that the total value of the bundle $\bar{N}\vec{b}$ is $\vec{\lambda} \cdot \bar{N}\vec{b}$. If this is to be produced entirely by the $N$ necessary workers, then we have the equilibrium condition
\[ TN = \vec{\lambda} \cdot \bar{N}\vec{b} \]
Plugging this into our rate above yields
\begin{align*}
	 \frac{T\bar{N} - TN}{T\bar{N}} &=  \frac{T\bar{N} - \vec{\lambda} \cdot \bar{N}\vec{b}}{\vec{\lambda} \cdot \bar{N}\vec{b}} \\
	 &= \frac{T - \vec{\lambda} \cdot \vec{b}}{\vec{\lambda} \cdot \vec{b}} \\
	 &= \frac{1 - \omega \vec{\lambda} \cdot \vec{b}}{\omega\vec{\lambda}\cdot \vec{b}} = e
\end{align*}
Neither of these notions of the rate of exploitation directly relate to Marx's own classic formulation of the rate of exploitation which he defined as the ratio $\frac{S}{V}$, where $S$ is the total surplus value produced, and $V$ is the total value of variable capital. Do define this, we must bring in and derive the classic value quantities $S$, $C$, and $V$. \par 
Let $\vec{x}$ denote the total daily gross output of society. Since there are $\bar{N}$ workers, each working $T$ hours at the same quality of labor, we have that the total value produced in a day is $T\bar{N}$, establishing the condition
\begin{align}
	T\bar{N} = \vec{l}\cdot \vec{x}
\end{align}
(Recall since $\vec{x}$ is the gross output and not the net, $\vec{\lambda}\cdot \vec{x}$ would not be the total value - it would overshoot the actual value produced and be bigger.)
The surplus obviously needs to be the vector $\vec{x}$ minus whatever is necessary. Two bundles will be necessary for a system in equilibrium. On the one hand, the bundle $\bar{N}\vec{b}$, necessary for each worker to produce their subsistence. This bundle is in fact the variable capital in goods. Thus the value of this bundle is exactly what Marx denoted $V$:
\[ V = \vec{\lambda} \cdot \bar{N}\vec{b} \]  
Also necessary is the constant capital $C$. These are the materials necessary for producing the bundle $\vec{x}$, which is precisely $A\vec{x}$. Thus
\[ C = \vec{\lambda} \cdot A\vec{x} \]
From the standpoint of system which perfectly reproduces itself daily, we must assume that $A\vec{x}$ exists on day $0$. This is used up to produce $\vec{x}$, which must include a copy of $A\vec{x}$, to be used the next day. Thus the bundle of surplus goods is given $\vec{x} - (A\vec{x} + \bar{N}\vec{b})$, and the total surplus value is
\[ S = \vec{\lambda}\cdot [\vec{x} - (A\vec{x} + \bar{N}\vec{b})] \]
The rate of surplus value is then
\[ s' = \frac{S}{V} = \frac{\vec{\lambda}\cdot [\vec{y} - (A\vec{y} + \bar{N}\vec{b})]}{\vec{\lambda} \cdot \bar{N}\vec{b}} \] 
It is obviously critical that we show this to be equal to the rate of exploitation $e$. To do this we need to be a little sneaky. First, we come up with an equation for $1$ involving $e$ as we know it:
\begin{align*}
	& e = \frac{1-\omega\vec{\lambda}\cdot \vec{b}}{\omega\vec{\lambda}\cdot \vec{b}}\\
	&\implies (1+e)(\omega\vec{\lambda}\cdot \vec{b}) = 1
\end{align*}
We will slip this into the value determination equation
\begin{align*}
	\vec{\lambda} &= A^T\vec{\lambda} + \vec{l} \\
	&= A^T\vec{\lambda} + (1+e)(\omega\vec{\lambda}\cdot \vec{b})\vec{l} \\
	&= A^T\vec{\lambda} + \omega(\vec{\lambda}\cdot\vec{b})\vec{l} + e\omega(\vec{\lambda}\cdot\vec{b})\vec{l} \\
\end{align*}
This equation for $\vec{\lambda}$ will be important to look at on it's own in a moment, but for now what is important is that we have the identity
\[ e\omega(\vec{\lambda}\cdot\vec{b})\vec{l} = \vec{\lambda} - A^T\vec{\lambda} - \omega(\vec{\lambda}\cdot\vec{b})\vec{l} \]
Now, for the denominator of $s'$, since $T\bar{N} = \vec{l}\cdot \vec{x}$, we have that $\bar{N} = \omega (\vec{l} \cdot \vec{x})$, so that 
\[ V = \vec{\lambda} \cdot \bar{N}\vec{b} = \omega (\vec{l} \cdot \vec{x})(\vec{\lambda} \cdot \vec{b}) \]
For the numerator:
\begin{align*}
	S = \vec{\lambda}\cdot [\vec{x} - (A\vec{x} + \bar{N}\vec{b})] &= \vec{\lambda}^T\vec{x} - \vec{\lambda}^T A\vec{x} - \vec{\lambda}\cdot \bar{N} \vec{b} \\
	&= \vec{\lambda}^T\vec{x} - \vec{\lambda}^T A\vec{x} - \omega (\vec{\lambda} \cdot \vec{b})(\vec{l} \cdot \vec{x}) \\
	&= \vec{\lambda}^T\vec{x} - \vec{\lambda}^T A\vec{x} - \omega (\vec{\lambda} \cdot \vec{b})\vec{l}^T\vec{x} \\
	&= (\vec{\lambda}^T - \vec{\lambda}^T A - \omega (\vec{\lambda} \cdot \vec{b})\vec{l}^T)\vec{x} \\
	&= (\vec{\lambda} - A^T\vec{\lambda} - \omega(\vec{\lambda}\cdot \vec{b})\vec{l})^T\vec{x} \\
	&= (e\omega(\vec{\lambda}\cdot\vec{b})\vec{l})^T\vec{x} \\
	&= e[\omega(\vec{\lambda} \cdot \vec{b})(\vec{l}\cdot \vec{x})]
\end{align*}
Therefore we have
\begin{align*} 
s' = \frac{e[\omega(\vec{\lambda} \cdot \vec{b})(\vec{l}\cdot \vec{x})]}{\omega (\vec{l} \cdot \vec{x})(\vec{\lambda} \cdot \vec{b})} = e
\end{align*}
Consider the equation for $\vec{\lambda}$ derived during this process:
\[ \vec{\lambda} = A^T\vec{\lambda} + \omega(\vec{\lambda}\cdot\vec{b})\vec{l} + e\omega(\vec{\lambda}\cdot\vec{b})\vec{l} \]
Remember that the entries of $\vec{\lambda}$ refer to the "unit values" of each commodity. Originally, we have $\vec{\lambda} = A^T\vec{\lambda} + \vec{l}$, where $A^T\vec{\lambda}$ we say referred to the "dead labor" or constant capital required, and $\vec{l}$ was the living labor. What we've done with this equation then is split the living labor $\vec{l}$ into two components: the living labor which is "paid for", and the surplus labor. It is thus perfectly reasonable for a particular commodity industry $i$, to see $(A^T\vec{\lambda})_i = c_i$, $(\omega(\vec{\lambda}\cdot\vec{b})\vec{l})_i = v_i$, and $e\omega(\vec{\lambda}\cdot\vec{b})\vec{l})_i = s_i$, so that
\[ \lambda_i = c_i + v_i + s_i \]
This allows us to define individual rates of exploitation for each individual industry, via $e_i = \frac{s_i}{v_i}$. However for every industry $e_i = e$, since $v_i$ and $s_i$ is scalar multiples of each other. Thus the rate of exploitation in this economy is equalized across all industries. \par 
We saw that the global surplus was $S = e[\omega(\vec{\lambda} \cdot \vec{b})(\vec{l}\cdot \vec{x})]$. If we have a bundle of goods $\vec{y}$, it's value is $\vec{\lambda} \cdot \vec{y}$, which we can now see as 
\begin{align*}
	& (A^T\vec{\lambda} + \omega(\vec{\lambda}\cdot\vec{b})\vec{l} + e\omega(\vec{\lambda}\cdot\vec{b})\vec{l})^T\vec{y}  \\
	&= (A^T\vec{\lambda})\cdot\vec{y} + (\omega(\vec{\lambda}\cdot\vec{b})\vec{l})\cdot \vec{y} + (e\omega(\vec{\lambda}\cdot\vec{b})\vec{l})\cdot\vec{y} \\
	&= \vec{c} \cdot \vec{y} + \vec{v}\cdot \vec{y} + \vec{s}\cdot \vec{y}
\end{align*}
Thus breaking the labor value of $\vec{y}$ its constant, variable, and surplus capital components. Furthermore,if $\vec{y}$ is the total gross bundle of goods produced in a day, then we clearly have the identities
\[ V = \vec{v} \cdot \vec{y} \hspace{2cm} C = \vec{c}\cdot \vec{y} \hspace{2cm} S = \vec{s}\cdot \vec{y} \]  
Next we bring in prices, and consider profits. Recall $\vec{p} = \begin{pmatrix} p_1 \\ p_2 \\ \vdots \\ p_m \end{pmatrix}$ is the vector representing the price of a single unit of each commodity type, denominated in some dollar measure. Assume that the \textbf{wage rate} $w$ is the hourly price that the worker is paid. We are assuming that the worker is always able to purchase their subsistence basket. So the hourly wage rate is at least enough to purchase the basket $\omega \vec{b}$, and the price of this basket is $\vec{p} \cdot \omega \vec{b}$. Thus it must be the case that
\begin{align}
	w \geq \vec{p} \cdot \omega \vec{b}  \label{suffWages}
\end{align} 
 Recall also that in order for every capitalist industry to profit simultaneously, we must have 
 \[ \vec{p} > A^T\vec{p} + w\vec{l} \]
where $>$ here means that all of the coordinates of the left-hand side are greater than those of the right-hand side. We can now show that exploitation is a necessary condition for this to be the case. Before doing this we summarize the model at it's base, and our basic assumptions about it: \par 

\textbf{Summary of the model and it's base assumptions:} 
 Assume that an economy is in daily equilibrium with fixed and unchanging unit prices for all commodities, along with fixed and unchanging methods of production (thus fixing the $a_{ij}$ of the matrix $A$). Assume that the matrix $A$ is productive. Assume in this economy that a collection of $\bar{N}$ workers sell their labor for an hourly wage rate $w$, all labor for $T$ hours per day, and all produce labor of the same quality so that an hour of labor from one worker is identical to an hour of labor from any other worker, justifying the existence of standard fixed and unchanging living labor values for all commodities when paired with the assumption of unchanging methods of production. Call this vector $\vec{l}$, and assume that all entries of it are positive. Assume the existence of an average daily bundle of goods $\vec{b}$ meeting the minimum requirements of reproducing labor power for the average lifespan of a worker, and assume that the hourly wage rate is at least sufficient to purchasing this bundle (ie $w \geq \vec{p} \cdot \omega\vec{b}$). Note that no assumptions have been made yet about $\vec{p}$ or $w$, aside from $w$ needing to be sufficient to purchase means of subsistence. The question at hand is whether a vector of unit prices exists in which all industries can profit simultaneously 
\begin{theorem}[Fundamental Marxian Theorem]
	Given the above model, there exists a vector of prices $\vec{p}$ and a wage rate $w$ such that all industries can profit simultaneously, if and only if the rate of exploitation $e > 0$.
\end{theorem}
\begin{proof}
	For the forward direction, assume we have fixed a unit price vector $\vec{p}$ and a wage rate $w$ such that all industries can simultaneously profit. Substituting \ref{suffWages} for $w$ in this condition, we have
\begin{align*}
	\vec{p} > A^T\vec{p} + (\vec{p}\cdot \omega \vec{b})\vec{l} \label{profitability} 
\end{align*} 
Note that 
\[ (\vec{p} \cdot \omega \vec{b}) \vec{l} = \omega (\vec{b}^T \vec{p})\vec{l} = (\omega \vec{l} \vec{b}^T)\vec{p} \]
Thus the right-hand side of the inequality above can be seen as the result of a linear transformation on $\vec{p}$, specifically it equals $(A^T + \omega \vec{l} \vec{b}^T)\vec{p}$. One can immediately see then that the claim of all industries simultaneously profiting is equivalent to the claim that this linear transformation is productive. Since the transpose of a productive matrix is productive, we therefore have that $A+\omega\vec{b}\vec{l}^T$ must be productive, so there exists a vector $\vec{x}$ with strictly positive entries such that all entries of $\vec{x}$ are greater than all entries of $A\vec{x} + \omega\vec{b}\vec{l}^T\vec{x}$. Thus the dot product $\vec{\lambda}\cdot\vec{x}$ must be greater than $\vec{\lambda}\cdot (A\vec{x} + \omega\vec{b}\vec{l}^T\vec{x})$. Rewriting this second dot product:
\begin{align*}
	\vec{\lambda}\cdot (A\vec{x} + \omega\vec{b}\vec{l}^T\vec{x}) &= \vec{\lambda}\cdot A\vec{x} + \omega\vec{\lambda} \cdot (\vec{b}\vec{l}^T\vec{x}) \\
	&= (A^T\vec{\lambda})\cdot \vec{x} + \omega[\vec{\lambda}^T(\vec{b}\vec{l}^T)\vec{x}] \\
	&= (A^T\vec{\lambda})\cdot \vec{x} + \omega[(\vec{\lambda}^T\vec{b})(\vec{l}^T\vec{x})] \\
	&= (A^T\vec{\lambda})\cdot \vec{x} + (\omega(\vec{\lambda}\cdot\vec{b})\vec{l})\cdot\vec{x}
\end{align*}
As we said, $\vec{\lambda}\cdot\vec{x}$ is greater than this, and so
\begin{align*}
	& \vec{\lambda}\cdot\vec{x} - (A^T\vec{\lambda})\cdot \vec{x} - (\omega(\vec{\lambda}\cdot\vec{b})\vec{l})\cdot\vec{x} > 0 \\
	&\implies [\vec{\lambda} - A^T\vec{\lambda} - \omega(\vec{\lambda}\cdot\vec{b})\vec{l})]\cdot\vec{x} > 0
\end{align*} 
But recall that 
\begin{align*}
 \vec{s} = \vec{\lambda} - (\vec{c}+\vec{v}) \implies e\omega(\vec{\lambda}\cdot\vec{b})\vec{l} = \vec{\lambda} - A^T\vec{\lambda} - \omega(\vec{\lambda}\cdot\vec{b})\vec{l}
 \end{align*}
and this is precisely what is in the brackets above. Thus we have
\begin{align*}
	& (e\omega(\vec{\lambda}\cdot\vec{b})\vec{l})\cdot\vec{x} > 0 \\
	&\implies e[\omega(\vec{\lambda}\cdot\vec{b})(\vec{l}\cdot\vec{x})] > 0
\end{align*}
Since the entries of $\vec{x}$ and $\vec{l}$ are strictly positive, that dot product is positive, as is the dot product $\vec{\lambda} \cdot \vec{b}$, as was shown earlier by virtue of $A$ being productive. Thus the term in brackets is a number greater than $0$, and so it must follow that $e > 0$ as well. \par 
Conversely, suppose that $e > 0$. Again let us consider the equation for $\vec{\lambda}$:
\[ \vec{\lambda} = \vec{c} + \vec{v} + \vec{s} = A^T\vec{\lambda} + \omega(\vec{\lambda}\cdot\vec{b})\vec{l} + e\omega(\vec{\lambda}\cdot\vec{b})\vec{l} \]
Since $\vec{l}$ is assumed positive and $A$ assumed productive, $\omega(\vec{\lambda}\cdot\vec{b})\vec{l}$ is greater than $0$ in all entries. Since $e > 0$ by hypothesis, we can conclude that the third term $\vec{s}$ itself is positive in all entries, and so we have
\[ \vec{\lambda} > A^T\vec{\lambda} + \omega(\vec{\lambda}\cdot\vec{b})\vec{l} \]
Let $\psi > 0$, and define $\vec{p} := \psi\vec{\lambda}$, and $w := \vec{p}\cdot \omega\vec{b} = \psi\omega(\vec{\lambda}\cdot\vec{b})$. Then clearly our system is able to reproduce it's workforce daily (minimally), and in fact 
\begin{align*}
	A^T\vec{p} + w\vec{l} &= A^T\psi\vec{\lambda} + \psi\omega(\vec{\lambda}\cdot\vec{b})\vec{l} \\
	&= \psi(A^T\vec{\lambda} + \omega(\vec{\lambda}\cdot\vec{b})\vec{l}) \\
	&< \psi\vec{\lambda} = \vec{p}
\end{align*}
Thus, all industries are profiting simultaneously, by simply setting unit prices as proportional to unit values. 
\end{proof}
Note here that we have found a set of prices and wages that produce profits for all industries by setting profits as proportional to values, but we did \textit{not} show that doing so is the only way in which all industries can profit. Nonetheless, we have shown that exploitation is in some sense the source of profits, in that, under the assumptions above, exploitation is a necessary and sufficient condition for profits. \par 
\subsection{Profit}
Recall that the vector of unit prices of production is given $A^T\vec{p} + w\vec{l}$. Thus 
\[ \vec{p} - A^T\vec{p} - w\vec{l} \]
is the vector of unit profits for a unit sold of each commodity type. The rate of profit of the $i^{th}$ industry, $\pi_i$, is simply put the ratio of profit to cost price of a unit of the $i^{th}$ commodity type. Writing $A$ in terms of it's columns:
\[ A= \begin{pmatrix} \vec{a}_1 & \vec{a}_2 & \ldots & \vec{a}_m \end{pmatrix} \]
we can see that the $i^{th}$ entry of the cost price vector $A^T\vec{p}+w\vec{l}$ will be $\vec{a}_i \cdot \vec{p} + wl_i$. The rate of profit of the $i^{th}$ industry can then be written
\[ \pi_i = \frac{p_i - (\vec{a}_i \cdot \vec{p} + wl_i)}{\vec{a}_i \cdot \vec{p} + wl_i} \]
Assume for a while that, somehow, the rate of profit is equal across all industries, and we will denote this equilibrium rate of profit $\pi$. Replacing $\pi_i$ with $\pi$ and solving for it:
\begin{align*}
	  p_i &= \pi(\vec{a}_i \cdot \vec{p} + wl_i) + (\vec{a}_i \cdot \vec{p} + wl_i) \\
	 &= (1+\pi)(\vec{a}_i\cdot \vec{p} + wl_i)
\end{align*}
Thus we can see that
\[ \vec{p} = (1+\pi)(A^T\vec{p}+w\vec{l}) \]
Consider the vector $\vec{p}_w := \frac{1}{w}\vec{p}$. This would be analogous for the specific price random variable from Farjoun and Machover. It measures the price of a commodity not by it's price, but by the number of hours of labor which that commodity could purchase if exchanged at it's price. Thus it will be extremely important to compare this vector with the vector of unit values $\vec{\lambda}$. Suppose that all industries are simultaneously profiting, i.e. 
\[ \vec{p} > A^T\vec{p} + w\vec{l} \]
Dividing both sides by $w$ gives
\[ \vec{p}_w > A^T\vec{p}_w + \vec{l} \]
Solving for $\vec{p}_w$ then gives
\[ \vec{p}_w > (I-A^T)^{-1}\vec{l} \]
Where inequality persists since the entries of $I-A^T$ are nonnegative; if all entries of the left side are greater than all those on the right, then taking a linear combination of both sides in the same proportions of each coordinate will preserve inequality. But $\vec{\lambda} = A^T\vec{\lambda} + \vec{l}$, and thus the right hand side is $\vec{\lambda}$. We thus have that
\begin{align}
	\vec{p}_w > \vec{\lambda} \label{priceExceedsValue}
\end{align}
Thus in a profitable capitalist economy, prices measured in labor hours must exceed the actual number of hours required to create the commodities. This doesn't preclude the proportionality between prices and values, however. Returning to profit, the equation we had earlier relating prices to the rate of profit, with the price vector replaced by the specific price vector $\vec{p}_w$, is
\[ \vec{p}_w = (1+\pi)(A^T\vec{p}_w + \vec{l}) \]
Suppose workers are paid exactly subsistence wages, i.e. $w = \vec{p} \cdot \omega\vec{b}$. Then equivalently we have
\[ 1 = \vec{p}_w \cdot \omega \vec{b} \]
We will now show that in such a situation, the rate of profit is necessarily higher than the rate of exploitation. We begin by noting that since workers are only paid subsistence, we have $w = \vec{p}\cdot\vec{b}$. Substituting this for $w$ in the equation for the rate of profit gives
\begin{align*}
 \vec{p} &= (1+\pi)(A^T\vec{p} + (\vec{p}\cdot \omega\vec{b})\vec{l}) \\
 	&= (1+\pi)(A^T\vec{p} + \omega\vec{l}\vec{b}^T\vec{p}) \\
 	&= (1+\pi)(A^T + \omega\vec{l}\vec{b}^T)\vec{p}
\end{align*} 
The matrix $A^T + \omega\vec{l}\vec{b}^T$ might look familiar. Call it $M$, since it's obviously becoming important. We'll reflect on what this matrix actually is shortly. For now though, we would like to show the existence of a row vector $\vec{x}$ such that 
\[ \vec{x} = (1+\pi)\vec{x}M \]
To see this, note first that since $\vec{p} = (1+\pi)M\vec{p}$, it follows that $\vec{p}$ is a nonzero vector in the null space of $I-(1+\pi)M$. Thus this null space has a dimension greater than $0$, and it follows that the matrix is not invertible. Thus the transpose $I-(1+\pi)M^T$ is also not invertible, and thus must also have a null space of dimension greater than $0$. Let $\vec{y}$ be a vector in this null space. Then we have that $\vec{y} = (1+\pi)M^T\vec{y}$, so then if $\vec{x} = \vec{y}^T = (1+\pi)\vec{x}M$. We wish to show that this vector is nonnegative. Suppose it has a negative coordinate, and moreover suppose that any vector we choose which satisfies the equation also must have a negative coordinate. Without loss of generality we can assume that at least one of it's coordinates is positive, since if all nonzero coordinates are negative we can simply take $-\vec{x}$ in place of $\vec{x}$. Then we can define the nonnegative vector $\vec{x}^*$ by replacing any negative coordinates with $0$. Consider an arbitrary coordinate $x^*_i$, and let 
\[ M = \begin{pmatrix} w_{11} & w_{12} & \ldots & w_{1m} \\
		w_{21} & w_{22} & \ldots & w_{2m} \\
		\vdots \\ w_{m1} & w_{m2} & \ldots & w_{mm} \end{pmatrix} = \begin{pmatrix} \vec{w}_1 & \vec{w}_2 & \ldots & \vec{w}_m \end{pmatrix} \]
Then from these definitions it follows that 
\[ (\vec{x}^*M)_i = \vec{x}^* \cdot \vec{w}_i = \sum_{j=1}^m x^*_ia_{ji} \]
Now by definition of $\vec{x}*$ and since the entries of $M$ are nonnegative. If $i$ is such a coordinate that $x_i$ has been replaced with $0$, then obviously we have that the sum on the right is greater than or equal to $x^*_i = 0$. Suppose it is an untouched coordinate. Then we have that
\[ x_i = \sum_{j=1}^m x_ia_{ji} \leq \sum_{j=1}^m x_i^*a_{ji} \]
since, having replaced negative weights with $0$ weights, this linear combination cannot possibly be smaller. Thus we have that
\[ \vec{x}^* \leq (1+\pi)\vec{x}^*M \]
Now if all entries are equal, then we have a nonnegative $\vec{x}^*$ satisfying the equation which we assumed was impossible. Thus it must be the case that one of the entries has strict inequality. Thus, if multiplying by $\vec{p}$ (effectively a dot product), we have
\[ \vec{x}^*\vec{p} < (1+\pi)\vec{x}^*M \vec{p} \]
But we also have that $\vec{p} = (1+\pi)M\vec{p}$, and so 
\[ \vec{x}^*\vec{p} = (1+\pi)\vec{x}^*M\vec{p} \]
We have thus shown that $\vec{x}^*\vec{p}$ is both equal to and smaller than a particular number, a contradiction. It is thus the case that either we have constructed a nonnegative $\vec{x}^* \neq \vec{0}$ satisfying the desired condition, or we have that one must exist. Either way, we have what we came for. \par 
With all of that said, we can now assume the existence of a nonnegative row vector $\vec{y} \neq\vec{0}$ such that $\vec{x} = (1+\pi)\vec{x}M$. Let $\vec{y} = \vec{x}^T$. With this vector we can show that $\pi < e$, and demonstrate a condition in which an important relationship between the two rates holds. Turning to $\vec{\lambda}$, note
\begin{align*}
	\vec{\lambda} &= \vec{c} + \vec{v} + \vec{s} = A^T\vec{\lambda} + \omega(\vec{\lambda}\cdot\vec{b})\vec{l} + e\omega(\vec{\lambda}\cdot\vec{b})\vec{l} \\
	&= A^T\vec{\lambda} + (1+e)(\vec{\lambda}\cdot \omega \vec{b})\vec{l} \\
\end{align*} 
Thus 
\begin{align*}
	\vec{y} \cdot \vec{\lambda} = \vec{y}^TA^T\vec{\lambda} + \vec{y}^T(1+e)(\omega\vec{l}\vec{b}^T)\vec{\lambda}
\end{align*}
At the same time we have $\vec{y}^T = (1+\pi)\vec{x}^T(A^T + \omega\vec{l}\vec{b}^T)$. Multiplying by $\vec{\lambda}$ then gives
\begin{align*}
	\vec{y}\cdot \vec{\lambda} = (1+\pi)\vec{y}^T(A^T + \omega\vec{l}\vec{b}^T)\vec{\lambda}
\end{align*}
Thus we can equate the right hand side of both of these equations and simplify:
\begin{align*}
	& \vec{y}^TA^T\vec{\lambda} + \vec{y}^T(1+e)(\omega\vec{l}\vec{b}^T)\vec{\lambda} = (1+\pi)\vec{y}^T(A^T + \omega\vec{l}\vec{b}^T)\vec{\lambda} \\
	&\implies \cancel{\vec{y}^TA^T\vec{x}} + \cancel{\vec{y}^T\omega\vec{l}\vec{b}^T\vec{\lambda}} + e\vec{y}^T\omega\vec{l}\vec{b}^T\vec{\lambda} = \cancel{\vec{y}^TA^T\vec{\lambda}} + \cancel{\vec{y}^T\omega\vec{l}\vec{b}^T\vec{\lambda}} + \pi\vec{y}^TA^T\vec{\lambda} + \pi\vec{y}^T\omega\vec{l}\vec{b}^T\vec{\lambda} \\
	&\implies \pi(\vec{y}^TA^T\vec{\lambda}+\vec{y}^T\omega\vec{l}\vec{b}^T\vec{\lambda}) = e\vec{y}^T\omega\vec{l}\vec{b}^T\vec{\lambda} \\
	&\implies \pi = e\frac{\omega\vec{l}\vec{b}^T\vec{\lambda} \cdot \vec{y}}{(A^T\vec{\lambda} + \omega\vec{l}\vec{b}^T\vec{\lambda})\cdot\vec{y}} = e\frac{[(\lambda \cdot \omega\vec{b})\vec{l}]\cdot\vec{y}}{[A^T\vec{\lambda} + (\vec{\lambda}\cdot \omega\vec{b})\vec{l}]\cdot\vec{y}} = e\frac{\vec{v} \cdot\vec{y}}{(\vec{c} + \vec{v})\cdot \vec{y}}
\end{align*}
Since $\vec{l}$ and subsequently $\vec{\lambda}$ are assumed positive, $\vec{v}$ is strictly positive, as is $\vec{c}$. Thus clearly it must be that the numerator is smaller than the denominator of this final equation. Thus it follows that
\[ \pi < e \]
This relationship is true in general, given our standard set of assumptions with nothing extra added on. Suppose however that $\vec{y}$ is seen as the gross bundle of goods produced in society in a day. In this case, the numerator and denominator take on new meaning: $\vec{v} \cdot \vec{y}$ becomes $V$, the total value produced going to the workers in the form of wage goods, and $\vec{c} \cdot \vec{y}$ becomes $C$, the total value produced in raw materials. Since $e = \frac{S}{V}$, we thus have that in \textit{this specific case}, Marx's famous formula holds as valid:
\[ \pi = \frac{S}{C+V} \]
It goes without saying that $\vec{y}$ can be seen as a bundle of goods, since it is the right dimension and nonnegative. But what does it signify, and does it actually satisfy the necessary equilibrium conditions for our economy? More on this later. \par 
We've seen a lot of the matrix $M = A^T + \omega\vec{l}\vec{b}^T$. Clarifying the reason for this will help move things forward. Note that we already have the vectors $\vec{c} = A^T\vec{\lambda}$, and $\vec{v} = \omega(\vec{\lambda} \cdot \vec{b})\vec{l} = (\omega\vec{l}\vec{b}^T)\vec{\lambda}$. Thus $\vec{c} + \vec{v} = M\vec{\lambda}$. I.e. applying the matrix $M$ to the vector $\vec{\lambda}$ returns the "necessary" components of each unit value. Meanwhile consider the equivalent factoring of the unit price vector $\vec{p}$. For the $i^{th}$ commodity, we can consider the cost of raw materials and the cost of labor in assembling a single unit. For the former, $c^p_i = \sum_{j=1}^m p_ja_{ji}$. This is clearly the dot product of the $i^{th}$ column of $A$ with $\vec{p}$, and we therefore have 
\[ \vec{c}^p = A^T\vec{p} \] 
For the latter, $v^p_i = wl_i$, but for workers laboring at subsistence we have $w = \vec{p}\cdot (\omega\vec{b})$, and so $v^p_i = \vec{p}\cdot (\omega\vec{b})l_i = \omega l_i \vec{b}^T\vec{p}$. Thus 
\begin{align*}
	\vec{v}^p = \omega\begin{pmatrix} l_1 \\ l_2 \\ \vdots \\ l_m \end{pmatrix} \begin{pmatrix} b_1 & b_2 & \ldots & b_m \end{pmatrix}\vec{p} = (\omega\vec{l}\vec{b}^T)\vec{p}
\end{align*}
Thus once again, $M\vec{p}$ gives the components of each unit price which are "necessary", but this time instead of values it gives prices. Finally, consider the significance of $M^T$. Consider a bundle of goods $\vec{x}$. Then
\begin{align*}
	 M^T\vec{x} &= (A + \omega\vec{b}\vec{l}^T)\vec{x}
	 	&= A\vec{x} + \omega\vec{b}\vec{l}^T\vec{x} = A\vec{x} + \omega(\vec{l}\cdot \vec{x})\vec{b}
\end{align*} 
If this bundle $\vec{x}$ is the total gross product of a day, then recall we have $T\bar{N} = \vec{l}\cdot\vec{x} \implies  \bar{N} = \omega(\vec{l}\cdot\vec{x})$. But $\vec{v} = \omega(\vec{l}\cdot\vec{x})\vec{b} = (\omega\vec{l}\vec{b}^T)\vec{x}$, exactly that second term above. Thus 
\[  M^T\vec{x} = A\vec{x} + \bar{N}\vec{b} \]
Now $A\vec{x}$ is clearly the bundle of raw materials necessary to create the gross amount $\vec{x}$, while just as clearly $\bar{N}\vec{b}$ is the total bundle of necessary wage goods which are needed to reproduce the day's labor power. Thus, viewing $\vec{x}$ as the gross daily bundle of goods produced, $M^T\vec{x}$ returns the total bundle of necessary goods required to produce that gross output. It of course follows, and can be seen separately, that $(A\vec{x})\cdot\vec{\lambda}$ should then be the total value of the necessary materials, i.e. $C$. Indeed,
\[ (A\vec{x})\cdot\vec{\lambda} = \vec{x}^TA^T\vec{\lambda} = (A^T\vec{\lambda})\cdot\vec{x} = \vec{x} \cdot \vec{x} = C \]
And similarly for the second term. Putting the two together we then have the identity
\[ (M^T\vec{x})\cdot\vec{\lambda} = C+V \]
Another essential fact worth noting:
\begin{fact}
	If $A$ is irreducible, then $M$ is also irreducible. 
\end{fact}
\begin{proof}
	For simplicity we show that $M^T$ is irreducible on the assumption of $A$ being irreducible, for notational simplicity. Let $\omega\vec{b}\vec{l}^T = B$, so that $M^T = (1+\pi)A+(1+\pi)B$. Suppose that $M^T$ is not irreducible. Then there exists a permutation matrix $P$ such that
	\[ P^T(M^T)P = \begin{pmatrix} M_{11} & M_{12} \\ O & M_{21} \end{pmatrix} \]
For some block matrix of zeros $O$. Additionally, let
\[ (1+\pi)P^TAP = (1+\pi)\begin{pmatrix} A_{11} & A_{12} \\ A_{21} & A_{21} \end{pmatrix} \hspace{2cm} (1+\pi)P^TBP = (1+\pi)\begin{pmatrix} B_{11} & B_{12} \\ B_{21} & B_{21} \end{pmatrix} \]
Thus it must be the case that $(1+\pi)(A_{21} + B_{21}) = O$, the matrix of $0$'s. But both $A$ and $B$ are nonnegative as is $(1+\pi)$, so the entries of $A_{21}$ and $B_{21}$ must also be nonnegative, and so the equation can only be true if both $A_{21}$ and $B_{21}$ are themselves that zero matrix $O$. Thus if $M^T$ is not irreducible, then neither are $A$ nor $B$, a contradiction since we are assuming $A$ is irreducible. Thus $M^T$ must be irreducible as well as $M$. 
\end{proof}
From this and the Perron-Frobenius theorem the following corollary is worth noting:
\begin{corollary}
	The \emph{balanced growth} vector $\vec{y}$ satisfying $\vec{y} = (1+\pi)M^T\vec{y}$ is unique up to positive scalar multiples. The so called "balanced growth path" is one-dimensional.  Furthermore, the path itself is independent of the profit rate $\pi$. 
\end{corollary}
To conclude this section laying out the basic definitions and relationships, we already have the unit surplus vector $\vec{s} = \vec{\lambda} - (\vec{c}+\vec{v})$. Similar we can now define the vector of unit profits, which we will denote $\vec{\rho}^p = \vec{p} - (\vec{c}^p + \vec{v}^p)$. We add the superscript $p$ to this vector to indicate that it is going to be a function of the unit price vector, when constrained by the coming equilibrium conditions.
\section{The Transformation Problem}
 Marx's theory of prices of production found in volume three of capital. In this volume, Marx assumes a capitalist society in which the rate of profit has been equalized across all industries, despite the composition of capital of each industry being allowed to differ. These assumptions at their core create immediate difficulties for any labor theory of value: if two industries have differing compositions of capital, then their profit rates should differ. It is clear that if the labor theory of value is to be true in such an ideal society, that there must be some sort of process by which the values of commodities are \emph{transformed} into new numbers of (of the same numeraire, labor time) which are themselves proportional to prices by a constant of proportionality $\alpha$. Marx argued exactly this. He called these adjusted numbers the prices of production of a commodity, and made the following claims about them: 
\begin{itemize}
	\item[(1)] "The sum of the prices of production of all commodities produced in society - the totality of all branches of production - is equal to the sum of their values."
	\item[(2)] "It remains true, nevertheless, that the cost price of a commodity is always smaller than it's value."
	\item[(3)] "Surplus values and profit are identical from the standpoint of their mass."
	\item[(4)]
	\item[(5)]
\end{itemize}
The picture that Marx is painting here is that the transformation of values into prices of production is conservative in the sense that it is a mere shifting around of the mass of value produced. Some value from industries with lower composition of capital is shifted over to those industries with higher composition of capital, so that all capitalists can enjoy the same rate of profit despite some industries producing objectively more value than others. A sort of 'communism among the capitalist class', as David Harvey I think put it. This process, to Marx, was also important in that it served to mystify the true source of value, so that capitalists, entirely focused on their profit rates, would be blind to the true source of those profits. In other words, this transformation process was in some sense to Marx the technical process by which money fetishism takes root. \par
Thus in this section we assume an equalized profit rate across industries, $\pi$. As we have a vector of unit prices, $\vec{p}$, we also should have a vector of unit prices of production, which Marx defines as 
\[ \vec{q} := (1+\pi)(\vec{c} + \vec{v}) = (1+\pi)M\vec{\lambda} \]
These were the numbers that Marx used in his own calculations and demonstrations, but he nonetheless he did not claim that these were the \textit{actual} prices of production. Instead, Marx considered these to be merely first approximations to the prices of production. The idea is that these updated "values" create a "ripple effect" of sorts. If the "value" of commodity $i$ changes in such a way as to effect prices, then this will in turn cause the "values" of all other commodities in the economy to change which use that commodity as a direct or even indirect input. Thus Marx's $\vec{q}$ is simply the first step $\vec{q}_1$ of what he considered to be an iterative process defined by
\[  \vec{q}_0 = \vec{\lambda} \]
\[ \vec{q}_{n+1} = (1+\pi)M\vec{q}_n \]
\begin{claim}
	The iterative process above converges. Moreover it converges to a vector $\vec{q}$ satisfying
	\[ \vec{q} = (1+\pi)M\vec{q} \]
	We saw above in the proof of the existence of the balanced growth path $\vec{y}$ that such a vector \emph{exists}, (or at least it's column vector equivalent for $M^T$), but we did not see that such an iterative process above must converge (if it converges at all) to such a vector. 
\end{claim} 
One thing which we can immediately see about the above claim initiates the cascade of issues with this entire framing; which is that the steady state here must be unique. This is because $M$ is irreducible, and the steady state must be strictly positive. If $\vec{q}$ is a solution to the above equation, then it is a positive eigenvector of $M$ with corresponding eigenvalue $\frac{1}{1+\pi}$, which by Perron-Frobenius is unique up to scalar multiples. This observation about $M$ though, that it is irreducible, creates a concern in the following sense. \par 
Recall that if all industries are profiting simultaneously at a common equilibrium rate of profit $\pi$, then the price vector must satisfy the equation 
	\[ \vec{p} = (1+\pi)M\vec{p}  \]
Again, this amounts to the claim that $\vec{p}$ is a strictly positive eigenvector of $M$ with eigenvalue $\frac{1}{1+\pi}$. Again, by Perron Frobenius, this is unique in it's being a positive vector with a positive eigenvalue, up to scalar multiples. The issue presents itself when we realize that the \emph{the profit rate is completely and uniquely determined by the technical matrix} $M$. What determines $M$? The technical coefficients of production within $A$, the living labor values $\vec{l}$, the length of the working day $T$, and finally the means of subsistence vector $\vec{b}$. If these are fixed, then there is only a single profit rate which could possible be the equilibrium rate of profit. Consequently the prices of all commodities is completely determined by $M$ as well. This is very concerning, since it means that profits and prices in this situation have nothing directly to do with labor values. $\vec{\lambda}$ appears nowhere in these equations. This doesn't mean the situation is hopeless, but it does make it seem like values are superfluous and unnecessary to determine prices. \par 
The next question to ask then is: do prices and labor values at least coincide in this situation where all industries are profiting simultaneously? The answer is yes, but only in a very specific and unrealistic circumstance: 
\begin{theorem}
	Suppose that the rate of profit $\pi$ is equalized across all industries. Then profits are proportional to surplus values (i.e. there exists a positive $\alpha$ such that $\vec{s} = \alpha\vec{\rho}$) if and only if all industries have an equal value composition of capital. That is to say, $\frac{c_1}{v_1} = \frac{c_2}{v_2} = \ldots = \frac{c_m}{v_m}$. 
\end{theorem}
\begin{proof}
	Suppose first that $\vec{s} = \alpha\vec{\rho}$ for some $\alpha$. The clearly $\vec{s} - \alpha\vec{\rho} = \vec{0}$. And yet
	\begin{align*}
		\vec{s} - \alpha\vec{\rho} &= (\vec{\lambda} - M\vec{\lambda}) - \alpha(\vec{p} - M\vec{p}) \\
		&= (I-M)(\vec{\lambda} - \alpha\vec{p})
	\end{align*}
Recall that when all industries are profiting simultaneously, $\vec{p} > M\vec{p}$, and so $M$ is productive and therefore $I-M$ is invertible. Thus multiplying both sides by $(I-M)^{-1}$, we obtain that $\vec{\lambda} = \alpha\vec{p}$. \par 
From this it follows that $\vec{c} = A^T\vec{\lambda} = A^T\alpha\vec{p} = \alpha\vec{c}^p$, and similarly $\vec{v} = \alpha\vec{v}^p$. Thus for any $i$ we have
\[ \pi = \pi_i = \frac{\rho_i}{c_i^p + v_i^p} = \frac{s_i}{c_i+v_i} = e\frac{v_i}{c_i+v_i} \]
Thus by assuming the rate of profit to be equalized we have that all of these fractions are the same. Therefore 
\begin{align*}
	& \frac{v_1}{c_1+v_1} = \frac{v_2}{c_2+v_2} = \ldots = \frac{v_m}{c_m+v_m} \\
	&\implies \frac{v_1}{c_1}+1 = \frac{v_2}{c_2}+1 = \ldots = \frac{v_m}{c_m}+1 \\
	&\implies \frac{c_1}{v_1} = \frac{c_1}{v_1} = \ldots = \frac{c_1}{v_1}
\end{align*}
Conversely, suppose that all industries have an equal composition of capital. By the reverse of the logic immediately above then, we have
\[ e\frac{v_1}{c_1+v_1} = e\frac{v_2}{c_2+v_2} = \ldots = e\frac{v_m}{c_m+v_m} \]
Call this ratio $\pi_s$. Note that
\[ \lambda_i = s_i + c_i + v_i = (\frac{s_i}{c_i+v_i} + 1)(c_i+v_i) = (\pi_s+1)(c_i+v_i) \]
Thus we have the equations
\[ \vec{\lambda} = (1+\pi_s)(\vec{c} + \vec{v}) = (1+\pi_s)M\vec{\lambda} \]
\[ \vec{p} = (1+\pi)M\vec{p} \]
Now \textit{if} prices are proportional to values, and the equation for $\vec{\lambda}$ determined as it is here, then $\vec{p} = \alpha\vec{\lambda}$ implies that $\pi_s = \pi$, and we can see then that this $\vec{p}$ must be a solution to the above price equation. So prices proportional to values satisfy the necessary conditions here. By the uniqueness up to proportionality of such a solution, we conclude that any $\vec{p}$ satisfying the above must be proportional to $\vec{\lambda}$, since that will always be a solution. 
\end{proof}
Marx was prepared for this much. As we said, never pretended that prices should be directly proportional to values in the presence of an equilibrium rate of profit. What he suggested instead is that the five global claims about the economy described in the beginning of this section are true - the picture painted overall is that labor is still the sole source of prices, and that the prices differing are merely a redistribution of that value - a sort of communism among the capitalist class. \par 
What Morishima is interested in doing first and foremost is finding the minimal conditions under which Marx's self described first approximation $\vec{q}$ is actually the final approximation, so that his calculations hold precisely. From the above theorem we have the following corollary:
\begin{corollary}
	If all industries have identical value compositions of capital, then Marx's prices of production $\vec{q} = \vec{\lambda}$, and furthermore, since $\vec{q}$ is proportional to $\vec{p}$, it follows that prices in general are proportional to values. 
\end{corollary}
This is extremely unrealistic, however, and Marx is not interested in this extremely special case. Morishima identifies the necessary and sufficient conditions for Marx's first approximation to be exact:
\begin{theorem}
	Have $\vec{q}$ defined as above, that is, $\vec{q} = (1+\pi)M\vec{\lambda}$. Then $\vec{q} = (1+\pi)M\vec{q}$ iff 
	\[ \pi M(\vec{c}+\vec{v}) = M\vec{s} \]
\end{theorem}
\begin{proof}
	Note that
	\begin{align*}
		& M\vec{s} = \pi M(\vec{c}+\vec{v}) \\
		&\iff M(\vec{c}+\vec{v})+M\vec{s} = M(\vec{c}+\vec{v}) + \pi M(\vec{c}+\vec{v}) \\
		&\iff M(\vec{c}+\vec{v}+\vec{s}) = (1+\pi)M(\vec{c}+\vec{v}) \\
		&\iff M\vec{\lambda} = (1+\pi)M(M\vec{\lambda}) \\
		&\iff (1+\pi)M\vec{\lambda} = (1+\pi)M[(1+\pi)M\vec{\lambda}] \\
		&\iff \vec{q} = (1+\pi)M\vec{q}  
	\end{align*}
\end{proof}
This rather mystifying condition, $\pi M(\vec{c}+\vec{v}) = M\vec{s}$, deserves some consideration. Fundamentally, like the condition of equal compositions of capital, is a condition on the technical properties of the capitalist economy. ($A$ and $\vec{l}$). Note that since $\vec{s} = \vec{\lambda} - M\vec{\lambda}$, $M\vec{s} = M\vec{\lambda} - M(\vec{c}+\vec{v})$. Adding this second term to both sides of our condition above gives another equivalent way to state this condition:
\[ M\vec{\lambda} = (1+\pi)M(\vec{c}+\vec{v}) \]
The left hand side here is obviously $\vec{c}+\vec{v}$, the necessary component of the vector of unit values. The right hand side is the somewhat mystifying part. I feel like this version is what to stare at to make physical sense of the condition, but am having trouble wording and thinking about it. \par 
Morishima also notes that this condition can be written as the statement that 
\[ M(\pi(\vec{c}+\vec{v})-\vec{s}) = \vec{0} \label{linDepInd} \]
This can be true in two ways:
\begin{itemize}
	\item[(1)] $\pi(\vec{c}+\vec{v})-\vec{s} = \vec{0}$ \\
	\item[(2)] $\pi(\vec{c}+\vec{v})-\vec{s} \in Null(M)$ 
\end{itemize}
In the first case, we have that $\vec{s} = \pi(\vec{c}+\vec{v})$, and this means
\[ \vec{\lambda} = \vec{c}+\vec{v}+\vec{s} = (\vec{c}+\vec{v}) + \pi(\vec{c}+\vec{v}) = (1+\pi)(\vec{c}+\vec{v}) \]
i.e. the vector of surplus value is merely a scalar multiple of the constant and variable value, a proportion of it equal to the rate of profit. Moreover $\vec{s} = \pi(\vec{c}+\vec{v})$ implies that $s_i = \pi(c_i+v_i)$ for all $i$, i.e. $\frac{s_i}{c_i+v_i} = \pi$ for all $i$. Thus this first case implies that all industries have an equal composition of capital, or equivalently that profits are proportional to surplus values. Note also that in this first case, $\vec{c} = k\vec{v}$ for some $k$, and so $\ldots$ and so $det(M) = 0$ (this argument seems to require a two department model, so I need to come back here and verify if it's still true for a single industry. See page 77 footnote 5 of Morishima). Condition two also obviously implies that $det(M) = 0$. Thus a necessary condition for our own condition is that $M$ be singular. Necessary but obviously not sufficient, since of course $M$ can be singular without the particular vector $\pi(\vec{c}+\vec{v})-\vec{s}$ being $0$ or witnessing a nontrivial null space. Despite this, Morishima uses this as grounds to call \ref{linDepInd} that of \emph{linearly dependent industries} (referring to the fact that the columns of $M^T$ are the capital input feeding coefficients $a_{1i},a_{2i},\ldots,a_{mi}$ plus the labor input feeding coefficients $\omega l_i b_1,\omega l_i b_2,\ldots \omega l_i b_m$.) \par 
If we have \ref{linDepInd}, what exactly follows? By uniqueness up to proportionality of the solution $\vec{p} = (1+\pi)M\vec{p}$, it follows that whatever the unit price vector actually is, it must be proportional to $\vec{q}$, i.e. $\vec{p} = \alpha\vec{q}$. Take unit prices to be normalized such that $\vec{q} = \vec{p}$, i.e. $\alpha = 1$. Then the prices satisfy $\vec{p} = (1+\pi)(\vec{c}^p + \vec{v}^p)$, but $\vec{p} = \vec{q} = (1+\pi)(\vec{c}+\vec{v})$. These expressions can then be equated, the $(1+\pi)$ can be canceled out, and we have
\[ \vec{c}^p + \vec{v}^p = \vec{c}+\vec{v} \]
Thus under the condition of linear independent industries, unit costs of production are precisely the necessary components of the unit values. In other words, costs of production remain unchanged in the transformation of values into prices. This is true \emph{despite} prices no longer being proportional (necessarily) to values! Note that the above only implies that
\[ M\vec{\lambda} = M\vec{p} \]
This implies that $\vec{\lambda} = \vec{p}$ only if $M$ is invertible, but the condition of linearly independent industries, as we saw, strictly precludes this! \par 
Now let us consider Marx's claims about his prices of production one by one. Marx conflated prices and values in his own work. He did this because he assumed that profits could be proportional to surplus values without it being the case that industries have equal compositions of capital. As we have seen, that cannot be the case, at least in this situation of an equilibrium rate of profit (a situation I currently strongly disagree with in it's premise, but it's nonetheless what Marx himself wanted to do). This is why the following errors occur. \par  
With Marx's claim (1), he is claiming that the sum of the prices of production of the total bundle of goods is equal to the sum of the values of those goods. Let $\vec{x}$ be the total bundle of goods. Then the sum of the the prices of production of all goods of this bundle is $\vec{q}\cdot \vec{x}$, while the total value is of course $\vec{\lambda} \cdot \vec{x}$. We then have 
\begin{align*}
	\vec{q}\cdot\vec{x} &= (1+\pi)(\vec{c}+\vec{v})\cdot\vec{x} \\
		&= (1+\frac{\vec{s}\cdot\vec{y}}{(\vec{c}+\vec{v})\cdot\vec{y}})(\vec{c}+\vec{v})\cdot\vec{x} \\
		&= \vec{c}\cdot\vec{x} + \vec{v}\cdot\vec{x} + \frac{(\vec{c}+\vec{v})\cdot\vec{x}}{(\vec{c}+\vec{v})\cdot\vec{y}}\vec{s}
\end{align*} 
We can now see that Marx's claim is not always true, even under the condition of "linearly dependent industries". It requires a very particular production path, that of the equilibrium balanced growth path $\vec{y}$. 
\section{Discussion}
Morishima is obsessed with this idea of balanced growth paths, and makes a great deal of the fact that all of Marx's fundamental claims from volume 3 hold true in the case where industries are "linearly dependent" and production proceeds along the balanced growth path $\vec{y}$. This is well and good but doesn't address the fundamental issue here, which is that the claims fail if this hyper specific and furthermore impossible output bundle isn't what society is producing daily. If labor is the true source of value, then the claims should apply for any \textit{realistic} growth path chosen. What I mean by realistic is the following. \par 
The equilibrium rate of profit is real strictly in the sense that there is undeniably a bundle of goods which society could be producing which could be associated with this equilibrium rate. If society were to produce this and the process of profit equalization were to occur, then things could settle here. But why would society settle into this kind of production? My impression is that Marx's claims should be universally applicable but only to the production bundles which are reachable from some initial state in which prices are proportional to values. To be convinced that \textit{any} of this matters in the slightest, I would have to be shown a dynamic model which demonstrates under some set of ideal conditions and basic assumptions about consumer and capitalist behavior that an equilibrium profit rate were being approached in actuality. Marx assumed an equilibrium profit rate existed because so did his contemporaries, but I'm not convinced. \par 
Capitalists seek higher profits, this much can be assumed. For the sake of an ideal situation I'll even assume that production methods are unchanging - innovation doesn't exist. A capitalist which sees an industry with a higher profit rate than his own will certainly in this situation move capital over to that industry, until a glut in that industry is produced driving down prices and subsequently profitability of that industry. This leads capitalists in that industry to reinvest their capital into other industries, and so forth and so on. This produces a dynamically evolving sequence of profit rates for each firm. Every economist of the time saw this, but they also all seem to assume that
\begin{itemize}
	\item[(1)] The sequence of profit rate of each industry is Cauchy, i.e. they are actually converging to something over time. 
	\item[(2)] Not only are those profit rates all converging, but they are converging to the same value. 
\end{itemize}
Why on earth should either of these be true? If anything, the primitive labor theory of value, Adam Smith's ideal starting point wherein prices are directly proportional to labor values, if it were \textit{ever true} in any capacity at any historical moment, seems to heavily imply that none of these rates of profits should \emph{ever} converge to each other, since not only would they be starting from wildly different rates, but there is no reason to assume that those initial industrial natural rates wouldn't continue to exert pull as attractors themselves! \par 
There was never any good reason as far as I can tell for these economists to assume that an equilibrium rate of profit even exists as a number to discuss within their theories. Never mind the claims that "the equilibrium profit rate will never actually emerge in reality" (despite the claims being valid). Despite never actually being realized, an equilibrium rate of profit could still exert an influence on the economy in the sense of an attracting force. But if this rate can't even be a number theoretically, then it couldn't even be that. The fact that assuming that such a thing exists in the economy produces consequences radically differing from what one would expect doesn't to me implicate problems with a labor theory of value, but rather problems with that assumption, which unlike the labor theory of value I haven't seen anything close to a satisfactory justification. 
\section{Reproduction and the Reserve Army}
\subsection{Simple Reproduction}
For these it is essential to model the two departments separately. For each day let $\vec{x}_1$ and $\vec{x}_2$ be the real output bundles of the two departments. We assume that workers spend their entire income each day on wage goods. Suppose that for each unit of labor time the worker consumes $d_i$ amount of the $i^{th}$ wage good. Also let $f_i$ be the \emph{total} amount of commodity $i$ consumed by the capitalist class. 
\begin{align}
	\vec{d} = \begin{pmatrix} d_{m+1} \\ d_{m+2} \\ \vdots \\ d_n \end{pmatrix} \hspace{2cm} \vec{f} = \begin{pmatrix} f_{m+1} \\ f_{m+2} \\ \vdots \\ f_n \end{pmatrix}
\end{align}
represent unit consumption vectors. For simple reproduction, we have the condition $\vec{f}$ is consumed entirely by the capitalist class rather than any surplus being reinvested, and additionally we have that the bundle of raw materials $A_1\vec{x}_1 + A_2\vec{x}_2$ must be produced each day, so as to have the required materials tomorrow to make the same bundles tomorrow. These required finished goods are first of all $\vec{f}$, but also the wage goods, which amount to the total hours of labor worked multiplied by the unit bundle $\vec{d}$. The total consumption of wage goods in society then is 
\[ (\vec{l}_1\cdot \vec{x}_1 + \vec{l}_2 \cdot \vec{x}_2) \vec{d} \]
We thus have the system
\begin{align}
	& \vec{x}_1 = A_1 \vec{x}_1 + A_2 \vec{x}_2 \\
	& \vec{x}_2 = (\vec{l}_1\cdot \vec{x}_1 + \vec{l}_2 \cdot \vec{x}_2) \vec{d} + \vec{f} 
\end{align}
Given a unit price vector $\vec{p}_2$ for wage goods, we have a fixed hourly wage $w  =\vec{p}_2 \cdot \vec{d}$. Likewise the total price of the capitalist consumption bundle is given $\Pi = \vec{p}_2 \cdot \vec{f}$. At an assumed equilibrium rate of profit $\pi$, we have price vector equations
\begin{align}
	& \vec{p}_1 = (1+\pi)(A_1^T\vec{p}_1 + w\vec{l}_1) \\
	& \vec{p}_2 = (1+\pi)(A_2^T\vec{p}_1 + w\vec{l}_2) 
\end{align}
and the value determination equations
\begin{align}
	& \vec{\lambda}_1 = A_1^T\vec{\lambda}_1 + \vec{l}_1 = \vec{c}_1 + \vec{v}_1 + \vec{s}_1 \\
	& \vec{\lambda}_2 = A_2^T\vec{\lambda}_1 + \vec{l}_2 = \vec{c}_2 + \vec{v}_2 + \vec{s}_2
\end{align}
What we need to do is aggregate all capital goods industries together, and the same for the wage goods industries. The way this is done is by defining unit bundles $\vec{\delta}_1$ and $\vec{\delta}_2$ representing an abstract unit of output from the whole department, and we will require for obvious reasons that 
\begin{align}
	\vec{\lambda}_1 \cdot \vec{\delta}_1 = \vec{\lambda}_2 \cdot \vec{\delta}_2 = 1 
\end{align}
We can now reduce our $\vec{c}$, $\vec{v}$ and $\vec{s}$ vectors into simple scalars:
\[ c_I := \vec{c}_I \cdot \vec{\delta}_1 \hspace{1cm} v_I := \vec{v}_1 \cdot \vec{\delta}_1 \hspace{1cm} s_I := \vec{s}_1 \cdot \vec{\delta}_1 \]
\[ c_{II} := \vec{c}_2 \cdot \vec{\delta}_2 \hspace{1cm} v_{II} := \vec{v}_2 \cdot \vec{\delta}_2 \hspace{1cm} s_{II} := \vec{s}_1 \cdot \vec{\delta}_1 \]
This is technically overloading the notation since $c_1$ earlier was the first entry of $\vec{c}$, but Roman numeral subscripts are annoying and we probable won't need the old $c_1$ much, so I'm going to try running with this. Finally, let 
\[ \vec{y}_1 = \vec{\lambda}_1 \cdot \vec{x}_1 \]
\[ \vec{y}_2 = \vec{\lambda}_2 \cdot \vec{x}_2 \]
Collapsing the vector equations of reproduction down to an aggregated system in terms of value requires some assumptions about the industries. In particular, we assume that all industries in the capital goods department have the same composition of capital, as do all industries in the wage goods department, so that they can be aggregated. With that known we get common ratios 
\[ \frac{l_1}{\lambda_1} = \ldots \frac{l_m}{\lambda_m} \]
\[ \frac{c_1}{\lambda_1} = \ldots \frac{c_m}{\lambda_m} \]
\[ \frac{v_1}{\lambda_1} = \ldots \frac{v_m}{\lambda_m} \]
\[ \frac{s_1}{\lambda_1} = \ldots \frac{s_m}{\lambda_m} \]
as well as the same identities for department 2. For the second of these identities, denote the common ratios of $\frac{c_i}{\lambda_i} = \alpha_1$ for $i=1,\ldots,m$ and $\frac{c_i}{\lambda_i} = \alpha_2$ for $i=m+1,\ldots,n$. We then clearly have that $c_i=\alpha_1 \lambda_i$ for $i=1,\ldots,m$ and $c_i=\alpha_2 \lambda_i$ for $i=m+1,\ldots,n$, and so $\vec{c}_1 = \alpha_1 \vec{\lambda}_1$ and $\vec{c}_2 = \alpha_2 \vec{\lambda}_2$. Note then that for any vector $\vec{z}$ of the appropriate dimension, we have
\[ \frac{\vec{c}_i \cdot \vec{z}}{\vec{\lambda}_i \cdot \vec{z}} = \alpha_i \] 
for $i$ equal to either $1$ or $2$. In particular, we have
\[ \frac{\vec{c}_i \cdot \vec{x}_i}{\vec{\lambda}_i \cdot \vec{x}_i} = \alpha_i = \frac{\vec{c}_i \cdot \vec{\delta}_i}{\vec{\lambda}_i \cdot \vec{\delta}_i} = \vec{c}_i \cdot \vec{\delta}_i \]
Where the term on the right is either $c_I$ for $i=1$ and $c_{ii}$ for $i=2$. But the denominator of the left hand side is $y_i$. Multiplying this over for the two $i$s gives us the identity
\[ c_Iy_1 = \vec{c}_1 \cdot \vec{x}_1 \]
\[ c_{II} y_2 = \vec{c}_2 \cdot \vec{x}_2 \]
We are left with the aggregated equation for the first department:
\begin{equation}
	y_1 = c_I y_1 + c_{II}y_2
\end{equation}
Furthermore we can note that it turns out that $\alpha_1 = c_I$ and $\alpha_2 = c_{II}$, so that $c_I$ turns out to be the common ratio of the $c_i$'s to the overall $\lambda_i$'s. An identical argument works for $s$ and $v$ to show that $\frac{s_i}{\lambda_i}=s_I$ for $i=1,\ldots,m$, or $s_{II}$ for $i=m+1,\ldots,n$, and to show that $\frac{v_i}{\lambda_i}=v_I$ for $i=1,\ldots,m$ or $v_{II}$ for $i=1,\ldots,n$. \par 
Next note that $\vec{d} = \omega \vec{b}$. By dotting both sides of the department $2$ equation we get an equation for $y_2$. We get another equation for $y_2$ by dotting both sides of the second value determination equation by $\vec{x}_2$. Equating these gives
\[ \vec{\lambda}_2 \cdot (\vec{l}_1 \cdot \vec{x}_1 + \vec{l}_2 \cdot \vec{x}_2)\omega\vec{b} + \vec{\lambda}_2 \cdot \vec{f} = A_2^T \vec{\lambda}_1 \cdot \vec{x}_2 + \vec{l}_2 \cdot \vec{x}_2 \]
We can do the same thing with the department $1$ equation and the first value determination equation. In this case a term cancels on either side and we are left with 
\[ \vec{l}_1 \cdot \vec{x}_2 = \vec{\lambda}_1 \cdot A_2 \vec{x}_2 \]
If we solve the former of these equations for $\vec{l}_2 \cdot \vec{x}_2$, add the two equations together and cancel terms we get the following equation
\[ \vec{l}_1 \cdot \vec{x}_2 + \vec{l}_2 \cdot \vec{x}_2 = \vec{\lambda}_2 \cdot (\vec{l}_1 \cdot \vec{x}_1 + \vec{l}_2 \cdot \vec{x}_2)\omega \vec{b} + \vec{\lambda}_2 \cdot \vec{f} \]
Recall however that 
\[ e = \frac{1 - \vec{\lambda}_2 \cdot \omega \vec{b}}{\vec{\lambda}_2 \cdot \omega \vec{b}} \]
If we multiply top and bottom of this by $\vec{l}_1 \cdot \vec{x}_1 + \vec{l}_2 \cdot \vec{x}_2$, we get a numerator that only includes $\vec{\lambda}_2 \cdot \vec{f}$. Multiplying both sides by the denominator gives us the equation
\[ e\vec{\lambda}_2 \cdot (\vec{l}_1 \cdot \vec{x}_1 + \vec{l}_2 \cdot \vec{x}_2) \omega \vec{b} = \vec{\lambda}_2 \cdot \vec{f} \]
which says that the capitalist class consumes in value exactly what they extract in surplus value from the workers each day. Using this equation we can finally aggregate the second reproduction equation. We start by dotting the second unaggregated equation by $\vec{\lambda}_2$ to get $y_2$ on the left hand side. 
\begin{align*}
	y_2 &= \vec{\lambda}_2 \cdot (\vec{l}_1 \cdot \vec{x}_1 + \vec{l}_2 \cdot \vec{x}_2)\omega \vec{b} + \vec{\lambda}_2 \cdot \vec{f} \\
	&= \vec{\lambda}_2 \cdot (\vec{l}_1 \cdot \vec{x}_1 + \vec{l}_2 \cdot \vec{x}_2)\omega \vec{b} + e\vec{\lambda}_2 \cdot (\vec{l}_1 \cdot \vec{x}_1 + \vec{l}_2 \cdot \vec{x}_2) \omega \vec{b} \\
	&= [(\omega \vec{\lambda}_2 \cdot \vec{b})\vec{l}_1] \cdot \vec{x}_1 + [(\omega \vec{\lambda}_2 \cdot \vec{b}) \vec{l}_2] \cdot \vec{x}_2 + e[(\omega \vec{\lambda}_2 \cdot \vec{b})\vec{l}_1] \cdot \vec{x}_1 + e[(\omega \vec{\lambda}_2 \cdot \vec{b}) \vec{l}_2] \cdot \vec{x}_2 \\
	&= \vec{v}_1 \cdot \vec{x}_1 + \vec{v}_2 \cdot \vec{x}_2 + \vec{s}_1 \cdot \vec{x}_1 + \vec{s}_2 \cdot \vec{x}_2 \\
	&= v_I y_1 + v_{II}y_2 + s_I y_1 + s_{II} y_2
\end{align*}
At this point we're done aggregating, and so we will use integers instead of Roman numerals since we won't need to reference the first entry of the vector $\vec{c}$, for instance. (This previously was referred to as $c_1$, but now $c_1$ will instead take the place of $c_I$, and so forth. We thus have the following system of aggregated equations:
\begin{align}
	& y_1 = c_1 y_1 + c_2 y_2 \\
	& y_2 = v_1 y_1 + v_2 y_2 + s_1 y_1 + s_2 y_2
\end{align}
which describe what the distribution of value between the departments must be day to day in order for society to reproduce itself. Note also that since $c_i + v_i + s_i = \lambda_i$ for each $i$, and $c_I = \frac{c_i}{\lambda}_i$ etc, we have
\[ c_1 + v_1 + s_1 = \frac{c_i+v_i+s_i}{\lambda_i} = 1 \]
As well as
\[ c_2 + v_2 + s_2 = 1 \]
This gives us another system of equations that $y_1$ and $y_2$ must satisfy:
\begin{align}
	& y_1 = c_1y_1 + v_1 y_1 + s_1 y_1 \\
	& y_2 = c_2 y_2 + v_2 y_2 + s_2 y_2
\end{align} 
\subsection{Extended Reproduction}
All variables will be implicitly assumed functions of $t$, which is time measured in units of one industrial cycle. As before, $y_1(t)$ is the total capital investment in department $1$ (measured in value) at the end of period $t$, $y_2(t)$ the same for department $2$. Assume that the same rate of exploitation prevails in both industries, i.e. $\frac{s_1}{v_1} = \frac{s_2}{v_2} = e$. The prevailing technical coefficients for period $t$ are $c_i(t)$, $v_i(t)$, and $s_i(t)$, $i=1,2$, but we won't see these as changing with $t$ just yet. \par 
 Suppose that all capitalists reinvest the same percentage of their surplus product into industry, and that the same rate of profit prevails in both industries (ie equilibrium). Call this propensity to save $a \in (0,1)$. Thus, each cycle, capitalists in department $1$ take their surplus, $s_1y_1(t)$, take $as_1y_1(t)$ of that, and reinvest it. Of this reinvestment, 
\[  \frac{c_1}{c_1+v_1}as_1y_1(t) \] 
is invested in the capital goods department (department 1), while 
\[ \frac{v_1}{c_1+v_1}as_1y_1(t) \]
is invested in wage goods (department 2). Likewise capitalists in department 2 reinvest 
\[\frac{c_2}{c_2+v_2}as_2y_2(t) \] in department $1$ and 
\[\frac{v_2}{c_2+v_2}as_2y_2(t) \]
 in department $2$. Over the course of period $t+1$, surplus reinvested as variable capital will produce surplus. In department $1$ this is the amount 
 \[ e\frac{v_1}{c_1+v_1}as_1y_1(t) = \frac{s_1}{c_1+v_1}as_1y_1(t) \]
and likewise
\[ \frac{s_2}{c_2+v_2}as_2y_2(t) \]
in department $2$. The total value output of the departments after period $t+1$ will then be
\begin{align}
	& y_1(t+1) = y_1(t) + as_1y_1(t) + \frac{s_1}{c_1+v_1}as_1y_1(t) = y_1(t) + \frac{1}{c_1+v_1}as_1y_1(t) \\
	& y_2(t+1) = y_2(t) + as_2y_2(t) + \frac{s_2}{c_2+v_2}as_2y_2(t) = y_2(t) + \frac{1}{c_2+v_2}as_2y_2(t)
\end{align}
Facilitating this growth imposes requirements on the output of society by the end of period $t$. The capital goods output at time $t$ which allows for this growth is 
\begin{align}
	 y_1(t) &= c_1y_1(t) + c_2y_2(t) + c_1\Delta y_1(t) + c_2 \Delta y_2(t) \\
	 		&= c_1y_1(t) + c_2y_2(t) + \frac{c_1}{c_1+v_1}as_1y_1(t) + \frac{c_2}{c_2+v_2}as_2y_2(t)
\end{align}
\begin{align}
	y_2(t) &= v_1y_1(t) + v_2y_2(t) + v_1\Delta y_1(t) + v_2 \Delta y_2(t) + bs_1y_1(t) + bs_2y_2(t) \\
			&= v_1y_1(t) + v_2y_2(t) + \frac{v_1}{c_1+v_1}as_1y_1(t) + \frac{v_2}{c_2+v_2}as_2y_2(t) + bs_1y_1(t) + bs_2y_2(t)
\end{align}
Since $\Delta y_i(t) = y_i(t+1)-y_i(t)$ for both $i$, we also have
\begin{align}
	y_1(t) &= c_1y_1(t) + c_2y_2(t) + c_1y_1(t+1) - c_1y_1(t) + c_2y_2(t+1)-c_2y_2(t) \\
		&= c_1y_1(t+1) + c_2y_2(t+1)
\end{align}
and similarly
\begin{align}
	y_2(t) &= v_1y_1(t+1) + v_2y_2(t+1) + bs_1y_1(t) + bs_2y_2(t)
\end{align}
Therefore we have the matrix equation
\begin{align}
	\begin{pmatrix} y_1(t) \\ y_2(t) \end{pmatrix} &= \begin{pmatrix} c_1 & c_2 \\ v_1 & v_2 \end{pmatrix} \begin{pmatrix} y_1(t+1) \\ y_2(t+1) \end{pmatrix} + \begin{pmatrix} 0 & 0 \\ bs_1 & bs_2 \end{pmatrix} \begin{pmatrix} y_1(t) \\ y_2(t) \end{pmatrix} \\
	&= \begin{pmatrix} c_1 & c_2 \\ \frac{bs_1c_1 + v_1}{1-bs_2} & \frac{bs_1c_2 + v_2}{1-bs_2} \end{pmatrix}\begin{pmatrix} y_1(t+1) \\ y_2(t+1) \end{pmatrix} \\
	&:= \begin{pmatrix} M_{11} & M_{12} \\ M_{21} & M_{22} \end{pmatrix}\begin{pmatrix} y_1(t+1) \\ y_2(t+1) \end{pmatrix}
\end{align}
Defining $\vec{y}(t)$ in terms of $\vec{y}(t+1)$ seems a bit backwards at first glance. However this incorporates both the conditions of growth based on reinvestment as well as the conditions \emph{for} that reinvestment imposed by the required output for that growth. Thus what we have here is a relation between $y_1(t)$ and $y_2(t)$ which must hold for reinvestment of the sort specified to continue uninterrupted. Let $\mu_1$ and $\mu_2$ denote the eigenvalues of the matrix $M$, and 
\[ \vec{m}_1 = \begin{pmatrix} m^1_1 \\ m^1_2 \end{pmatrix}  \hspace{2cm} \begin{pmatrix} m^2_1 \\ m^2_2 \end{pmatrix} \]
be associated eigenvectors. Then by setting the determinant of $M-\mu I$ equal to $0$, solving for $\mu$ and simplifying the contents of the radical after applying the quadratic formula, we obtain the solutions
\[ \mu_1 =  \frac{1}{2}(M_{11}+M_{22}+\sqrt{(M_{11}-M_{22})^2 + 4M_{12}M_{21}}) \]
\[ \mu_2 =  \frac{1}{2}(M_{11}+M_{22}-\sqrt{(M_{11}-M_{22})^2 + 4M_{12}M_{21}}) \]
Since all of the $M_{ij}$s are positive we can see immediately that $\mu_1$ is always positive. Furthermore since the radical is always positive, $\mu_1$ is more than halfway between the smaller and the bigger of $M_{11}$ and $M_{22}$, so it must be bigger than at least one of these. Note that there exists some $u>0$ such that 
\[ \sqrt{(M_{11}-M_{22})^2 + 4M_{12}M_{21}} = |M_{11} - M_{22}| + u \]
since this is the distance plus some extra positive stuff inside the radical. Considering $\mu_2$ with this in mind, we see that we are taking the halfway point between $M_{11}$ and $M_{22}$, and subtracting away more than half the distance between them. Therefore $\mu_2$ may or may not be negative, but is always smaller than  both of $M_{11}$ and $M_{22}$. From the same argument we can also see that $\mu_1$ is \emph{bigger} than $M_{11}$ and $M_{22}$. When $\mu_1$ and $\mu_2$ are both positive, it is of course the case that $\mu_1 > \mu_2$. Suppose that $\mu_2$ is negative and that $\mu_1 < -\mu_2$. Then 
\[ \frac{1}{2}(M_{11}+M_{22}+|M_{11} - M_{22}| + u < -\frac{1}{2}(M_{11}+M_{22}-|M_{11} - M_{22}| - u \]
But this would imply that $M_{11} + M_{22} < 0$, which is a contradiction since all of these entries are positive. Thus we have the following facts known about the eigenvalues:
\begin{itemize}
	\item[(1)] $\mu_1 > 0$.
	\item[(2)] $\mu_1$ is bigger than both $M_{11}$ and $M_{22}$
	\item[(3)] $\mu_2$ is smaller than both $M_{11}$ and $M_{22}$
	\item[(4)] $\mu_1 > |\mu_2|$.
\end{itemize}
With these observations in mind let us turn to the eigenvectors $\vec{m}_1$ and $\vec{m}_2$. These must satisfy the systems
\begin{align}
	& (M_{11} - \mu_1)m^1_1 + M_{12}m^1_2 = 0  \label{e11}\\
	& M_{21}m^1_1 + (M_{22} - \mu_1)m^1_2 = 0 \label{e12}
\end{align}
\begin{align}
	& (M_{11} - \mu_2)m^2_1 + M_{12}m^2_2 = 0  \label{e21}\\
	& M_{21}m^2_1 + (M_{22} - \mu_2)m^2_2 = 0 \label{e22}
\end{align}
We can see here that $M_{11} - \mu_1$ is always negative, and so subtracting this term over and dividing we see that $m^1_1 = xm^2_1$ for some positive ratio $x$. Thus we can take the vector $\vec{m}_1$ as consisting of two positive entries. Namely let $m^1_1 = 1$, so that 
\[ m_2^1 = \frac{\mu_1-M_{11}}{M_{12}}m^1_1 \]
 Since $M_{11} - \mu_2$ is always positive, subtracting the term over and dividing gives that $m^2_1 = -xm^2_2$ for some positive $x$, and therefore we can take $\vec{m}_2$ as having a positive first entry and a negative second entry. In particular, again taking $m^2_1 = 1$, we have
\[ m^2_2 = \frac{\mu_2-M_{11}}{M_{12}}m^2_1 \]-  
  Thus $\vec{m}_1$ points somewhere in quadrant $1$ and $\vec{m}_2$ points somewhere in quadrant $2$, which is enough to be sure they are linearly independent. It follows that $M$ has an eigenbasis, and thus is always diagonalizable. Let 
  \[ P = \begin{pmatrix} m^1_1 & m^2_1 \\ m^1_2 & m^2_2 \end{pmatrix} \implies P^{-1} = \frac{1}{m^1_1m^2_2-m^2_1m^1_2} \begin{pmatrix} m^2_2 & -m^2_1 \\ -m^1_2 & m^1_1 \end{pmatrix} \]
Here seeing $P$ and $P^{-1}$ as change of basis matrices, $P$ takes vectors in the eigenbasis back to the standard basis, while $P^{-1}$ takes vectors from the standard basis into the eigenbasis. Let $\vec{y}(t)$ denote the pair of $y$ values in the standard basis, $\vec{z}(t)$ the vector in the eigenbasis, i.e. 
\[ \vec{z}(t) = P^{-1}\vec{y} \]
Then $M = PDP^{-1}$, where 
\[ D = \begin{pmatrix} \mu_1 & 0 \\ 0 & \mu_2 \end{pmatrix} \] so that
\begin{align}
	& \vec{y}(t) = PDP^{-1}\vec{y}(t+1) = PD\vec{y}_{\mathcal{M}}(t+1) \\
	&\implies \vec{z}(t) = D\vec{z}(t+1) \\
	&\implies \begin{pmatrix} z_1(t) \\ z_2(t) \end{pmatrix} = \begin{pmatrix} \mu_1z_1(t+1) \\ \mu_1z_2(t+1) \end{pmatrix}
\end{align}
Letting $\eta_1 = z_1(0)$ and $\eta_2 = z_2(0)$ represent the initial values of $z_1$ and $z_2$, i.e.
\[ \begin{pmatrix} \eta_1 \\ \eta_2 \end{pmatrix} = \frac{1}{m^1_1m^2_2-m^2_1m^1_2} \begin{pmatrix} m^2_2 & -m^2_1 \\ -m^1_2 & m^1_1 \end{pmatrix} \begin{pmatrix} y_1(0) \\ y_2(0) \end{pmatrix} \]
\[ \eta_1 = \frac{m^2_2y_1(0)-m^2_1y_2(0)}{m^1_1m^2_2-m^2_1m^1_2} = \frac{(\mu_2-c_1)y_1(0)-c_2y_2(0)}{m^1_1(\mu_2-\mu_1)} \]
\[ \eta_2 = \frac{-m^1_2y_1(0)+m^1_1y_2(0)}{m^1_1m^2_2-m^2_1m^1_2} = \frac{-(\mu_1-c_1)y_1(0)+c_2y_2(0)}{m^2_1(\mu_2-\mu_1)} \]
 we have then that $z_i(t+1) = \frac{1}{\mu_i}z_1(t)$ for each $i$, so that 
\[ z_1(t+1) = \frac{1}{\mu_1^t}\eta_1 \]
and likewise
\[ z_2(t+1) = \frac{1}{\mu_2^t}\eta_2 \]
But $\vec{y}(t+1) = P\vec{z}(t+1)$, so 
\[ \begin{pmatrix} y_1(t+1) \\ y_2(t+1) \end{pmatrix} = \begin{pmatrix} m^1_1 & m^2_1 \\ m^1_2 & m^2_2 \end{pmatrix} \begin{pmatrix} \frac{1}{\mu_1^t}\eta_1 \\ \frac{1}{\mu_2^t}\eta_2 \end{pmatrix} = \begin{pmatrix} \eta_1 m_1^1 \left(\frac{1}{\mu_1} \right)^t + \eta_2 m_1^2 \left( \frac{1}{\mu_2} \right)^t \\ \eta_1 m^1_2\left(\frac{1}{\mu_1} \right)^t + \eta_2 m^2_2 \left( \frac{1}{\mu_2} \right)^t \end{pmatrix} \]
Setting $1 + g_i = \frac{1}{\mu_i}$ for both $i$, we finally have the solutions
\begin{align}
	& y_1(t) = \eta_1 m_1^1(1+g_1)^t + \eta_2 m_1^2 (1+g_2)^t \\
	& y_2(t) = \eta_1 m_2^1(1+g_1)^t + \eta_2 m_2^2 (1+g_2)^t
\end{align}
(It seems like we should decrement and have $t-1$'s in the exponent but this doesn't produce functions which line up with $y_1(0)$ and $y_2(0)$ correctly unless you don't, and I don't care enough to consider the matter further than that.)
There are some important observations to make about these solutions to make sense of them. First, since $\mu_1 > |\mu_2|$, we have that 
\[ 1+g_1 < |1+g_2| \]
Moreover we make the following claim:
\begin{lemma}
	$\mu_1 < 1$, always. Furthermore $\mu_2$ is negative iff $k_2 > k_1$ (where $k_i = \frac{c_i}{v_i}$). 
\end{lemma}
\begin{proof}
	If we multiply equation \ref{e11} by $1-bs_1$, multiply equation \ref{e12} by $1-bs_2$, add the two equations together, and foil everything out into individual terms (probably actually a bad idea, see below), we get a big mess, set equal to $0$. Factor out $\mu_1$ from every term it's present in the mess. All remaining terms end up having either an $m^1_1$, an $m^1_2$, or both. Factor these out from their terms. Inside of the expression multiplied by $m^1_1$, without applying any identities it should turn out that it all collapses to $c_1+v_1$. Likewise the stuff multiplied by $m^1_2$ should collapse to $c_2 + v_2$. Move the $\mu_1$ stuff over to the other side of the equation, and distribute the minus sign. What we're left with is the equation 
\begin{align}
	 \mu_1[(1-bs_1)m^1_1 + (1-bs_2)m^1_2] &= (c_1+v_1)m^1_1 + (c_2+v_2)m^1_2 \\
	 &= (1-s_1)m^1_1 + (1-s_2)m^1_2
\end{align}
since $v_i+c_i+s_i=1$. Now since $b<1$, we have that $bs_1 < s_1$, so that $1-bs_1 > 1-s_1$. Thus the left hand side of this equation would be greater than the right hand side if $\mu_1 \geq 0$. Since equality presumably holds it must follow then that $\mu_1 < 0$. \par 
Moving on to $\mu_2$, multiply equation \ref{e21} by $v_1 + bs_1c_1$, equation \ref{e22} by $c_1(1-bs_2)$ (distribute but don't foil!), and subtract the latter from the former. Two terms will be seen to cancel. Pull $\mu_2$ out of the appropriate terms and subtract it over to the other side, then factor $m^2_2$ out of what's left. The expression multiplied by $m^2_2$ will simplify to $v_1c_2 - c_1v_2$, so that we are left with the equation
\[ \mu_2[(v_1+bs_1c_1)m^2_1-c_1(1-bs_2)m^2_2] = (v_1c_2 - c_1v_2)m^2_2 \]
We know that $m^2_1$ is going to be positive while $m^2_2$ is negative. Therefore the whole expression which $\mu_2$ is multiplied by is positive. The significance of this is that the sign of the left hand side is completely determined by $\mu_2$. With this in mind, consider the right hand side. Here, again $m^2_2$ is negative, so this side's sign is the reverse of the sign of the expression $v_1c_2 - c_1v_2$. Thus $\mu_2$ is positive iff $v_1c_2 - c_1v_2$ is negative, and negative iff $v_1c_2 - c_1v_2$ is positive. But since these variables are all themselves positive, we have
\[ v_1c_2 > c_1v_2 \iff \frac{c_2}{v_2} > \frac{c_1}{v_1}  \]
\end{proof}
Since $\mu_1$ is positive and less than $1$, it follows that $1+g_1$ is positive and greater than $1$, i.e. $g_1 > 0$. If the composition of capital for the wage goods industry is greater than that of the capital goods industry, then $\mu_2 < 0$ and so $1+g_2 < 0$ as well. Moreover we will have $1 < 1+g_1 < |1+g_2|$, so we have that $1+g_2 < -1$, i.e. the growth cannot decay over time - it must blow up, as it oscillates every period. 
\begin{corollary}
	
\end{corollary}

\end{document}

