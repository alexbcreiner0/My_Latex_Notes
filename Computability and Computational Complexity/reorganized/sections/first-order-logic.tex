\section{First-Order Logic}
\begin{definition}
    Suppose that there exists a \textit{finite} sequence of of first order expressions $S = (\phi_1,\phi_2,...,\phi_n)$ such that for each expression $\phi_i$ in the sequence, at least one of the following holds:
    \begin{enumerate}
        \item $\phi \in \Lambda$ (i.e. $\phi$ is a basic logical axiom.)
        \item There exist indices $j,k<i$ such that the expressions $\phi_j$ and $\phi_k$ are of the form $\psi$ and $\psi \Rightarrow \phi_i$ respectively. (So $\phi_i$ is 'true' inductively through modus ponens.)
    \end{enumerate}
    Then we call $S$ a \textbf{proof} (or sometimes a \textbf{deduction}) of $\phi_n$, and call $\phi_n$ \textbf{a first order theorem}. As notation, we write $\vdash \phi_n$
\end{definition}
The keyword is finite. \textit{Proofs are finite.} It should also be clear that first order theorems are valid, which we will show formally shortly, in a more general context. We are now ready to address a matter of great philosophical importance.
\par That question is the following: How do we discover truth? Through proof, of course. But first order theorems aren't what we are interested in proving. As we said before, these are vacuous syntactic shells with no real meaning. True meaning comes from discovering truth about \textit{specific models}. For instance, number theorists seek to prove truths about the 'true' model $\mathbb{N}$. But we don't know what this is in it's entirety, for if we did, we wouldn't be attempting to derive truths about it! All we know about $\mathbb{N}$ is that it exists, in some intangible Platonic realm of forms, and we know this because we've been working with it for our entire lives. The way that mathematicians find truths about models that they don't fully understand is through a process called \textit{axiomatization}. What this means is that we come up with a hopefully very safe, hopefully small, set of expressions $\Delta$, called \textbf{axioms}. These represent preconceived notions that we have about the model we are trying to determine truth within. For example, soon we will state the Peano axioms for number theory. This will be a set of expressions $PA$ such that we can generally agree that, whatever or wherever $\mathbb{N}$ actually is, we can all agree uncontroversially that for all $\phi \in PA, \mathbb{N} \models \phi$. Equipped with this set of axioms, we can extend our definition of validity to derive further truths about $\mathbb{N}$. We extend it in the natural way: 
\begin{definition}
    Suppose that there exists a \textit{finite} sequence of first order expressions $S = (\phi_1,\phi_2,...,\phi_n)$ such that for each expression $\phi_i$ in the sequence, at least one of the following holds:
    \begin{enumerate}
        \item Either $\phi \in \Lambda$ or $\phi \in \Delta$
        \item There exist indices $j,k<i$ such that the expressions $\phi_j$ and $\phi_k$ are of the form $\psi$ and $\psi \Rightarrow \phi_i$ respectively. (So $\phi_i$ is 'true' inductively through modus ponens.)
    \end{enumerate}
    Then we call $S$ a \textbf{proof/deduction} of $\phi_n$ through $\Delta$, and call $\phi_n$ a \textbf{$\Delta$-first order theorem}. As notation, we write $\Delta \vdash \phi_n$
\end{definition}
Finally, we say that a set of axioms $\Delta$ \textbf{satisfies} an expression $\phi$ and write $\Delta \models \phi$ if for any model $M$ which satisfies all of the expressions in $\Delta$, we have $M \models \phi$. Before we continue, we should make damn sure that at the very least, provable expressions are true: 
\begin{theorem}[The Soundness Theorem]
    If $\Delta \vdash \phi$, then $\Delta \models \phi$
\end{theorem}
\begin{proof}
    
\end{proof}
In the best case scenario, our set of axioms will be enough to fully characterize a model in the sense that any expression $\phi$ which is true in the model is also provable by our axioms. We formalize this important idea: We say that a model $M$ is \textbf{fully axiomatized} by $\Delta$ if
\[\{\phi: M \models \phi\} = \{\phi: \Delta \models \phi\} \]
To say this is to say that the set of axioms $\Delta$ fully characterizes the model $M$, pinning it down to it's exactness. Of course, any model is trivially axiomatizable - just take all of the expressions that are true as axioms. But this would just as obviously be missing the point. The point is to discover truths about a model in a way that is \textit{systematic}. Thus, we need to place some restrictions on what sets are allowed to be taken as axioms.
\par It is tempting to require that any set of axioms be finite, but this is too weak for anything interesting. For example, when we later list the Peano axioms for number theory, the axiom that essentially says "proofs by induction are a thing" is actually a countably infinite set of axioms, and there is no way around this (at least within first order logic). The next idea would naturally be to allow any set of axioms to be at most countably infinite, but this has the exact opposite problem of being too generous. To see this, just note that the set of all strings in the vocabulary of a language is countable, and therefore nothing at all is off limits. To see what type of restriction truly needs to be placed on a set of axioms, we turn back to computability. Define the following problem:
\begin{problem}
    Given any fixed vocabulary, we have the following problem:
    \begin{center}
        $THEOREMHOOD$: Given an expression $\phi$, is it a first order theorem? That is to say,
        \begin{align}
            THEOREMHOOD = \{\phi: \vdash \phi\} 
        \end{align}
    \end{center}
\end{problem}
\begin{fact}
    $THEOREMHOOD \in \textbf{RE}$
\end{fact}
\begin{proof}
    Any vocabulary consists of an at most countable number of symbols, and therefore the set of all expressions is countable. It is clear that the set of all expressions is recursively enumerable: just have a Turing machine write out each string one at a time in lexicographic order, and check using the inductive, already algorithmic definition to see if the expression is well formed. 
    \par Now, we use this observation to describe a Turing machine which accepts $THEOREMHOOD$. Upon any input $\phi$, the machine begins to list out all finite sequences of well formed expressions such that the final expression is $\phi$, and for each of these it checks to see if this sequence is a valid proof. Our definition of this is clearly by the Church Turing thesis something that can be computably checked - all we have to do is be able to consult our known list of logical axioms, and run over the sequence repeatedly to check for modus ponens. If it turns out that the sequence is indeed a proof, then we accept. Assuming that $\phi$ is a first order theorem, we will eventually find it's proof.
\end{proof}
There are several things to observe about this. Firstly, the assumption that proofs are finite was critical to this argument. If proofs were allowed to be infinite, then there is no way that a Turing machine would able to fully list out the proof and check it in a finite amount of time. Second, it is worth recalling and acknowledging that having this problem be recursively enumerable means that we can let a Turing machine run forever, and know that it will list out every first order theorems - It is an infinite set, but computer generated. Thirdly, we list \textit{another} computational problem, and compare:
\begin{problem}
    \begin{center}
        $VALIDITY$: Given a first order expression $\phi$, is it valid? That is to say
        \begin{align}
            VALIDITY = \{\phi: \models  \phi  \}
        \end{align}
    \end{center}
\end{problem}
This is \textit{not} obviously the same problem! We know by the soundness theorem that first order theorems are valid, but the converse is much less trivial. In fact, it is quite deep. Given a valid expression $\phi$, is there a finite proof of it? Why should there be? One would \textit{hope} that the answer is yes. If the answer is no, then that would either mean that our proof rules are missing something critical, or that mathematical thought in general is kinda broken. Fortunately, the answer is yes, and this comforting result is known as Godel's Completeness Theorem. We will prove it shortly. For now though, it is important that we understand the difference between these two decision problems. It is also important to note that at present we are only talking about first order theorems and valid expressions - there is no mention of a model or axiom system being made yet. The $THEROEMHOOD$ and $VALIDITY$ problems defined above are "empty" in that they are only addressing expressions which are syntactically true - valid, but without substance. We turn to substantial, model specific case next.
\par It will be shown that Godel's Completeness Theorem applies to any set of axioms as well, regardless of size. That is to say, for any model $M$ and any set of axioms $\Delta$ for $M$, it will end up being the case that $\Delta \models \phi \iff \Delta \vdash \phi$. However, framing the discussion around these two computational problems helps us to make precise what we meant earlier by the word \textit{systematic}. Despite the completeness theorem holding for axiom sets of arbitrary size, we would like a set which allows us to systematically derive truth in an exhaustive way. This informal desire is rigorously captured in terms of computability theory, by requiring that the following problems \textit{remain}, like their vacuous counterparts, recursively enumerable:
\begin{problem}
    Fix a set of axioms $\Delta$. Define the following two problems:
    \begin{center}
        $THEOREMHOOD_{\Delta}$: Given an expression $\phi$, is it the case that $\Delta \vdash \phi$? \\
        $VALIDITY_{\Delta}$: Given an expression $\phi$, is it the case that $\Delta \models \phi$?
    \end{center}
\end{problem}
It should be clear that in order for the problem $THEOREMHOOD_{\Delta}$ to be recursively enumerable, we need to require that $\Delta$ \textit{itself} be recursively enumerable. This is the essential requirement that we must impose on any axiom system. This is also why any metamathematical discussion of logic and axiomatic truth is fruitless without an accompanying discussion of computability theory. Since these are notes on computability theory, it should also be expected the dependence will prove itself to be mutual. A true understanding of computability theory relies on the results of formal logic. \textbf{From now on we will assume that any set of axioms $\Delta$ is recursively enumerable, unless otherwise noted.}
\par We now move towards a formal proof of the completeness theorem. Before we can prove it, we need to make sure that some of the essential proof techniques employed regularly by mathematicians carry over into our formal model:
\begin{theorem}[The Deduction Technique]
    Suppose that $\Delta \cup \{\phi\} \vdash \psi$. Then $\Delta \vdash (\phi \Rightarrow \psi)$
\end{theorem}
\begin{proof}
    Consider a proof $S = (\phi_1,\phi_2,...,\phi_n)$ of $\psi$ from $\Delta \cup \{\phi\}$, i.e. $\phi_n  = \psi$. It suffices then to go by induction and show that for each $i = 0,...,n$, $\Delta \vdash (\phi \Rightarrow \phi_i)$. For $i=0$, the claim is true vacuously. (The claim is just that ($\phi \Rightarrow \phi$), a tautology.) Assume then that it is true for all $j < i$, where $i \leq n$. Our proof of $(\phi \Rightarrow \phi_i)$ includes all of the proofs of the expressions $(\phi \Rightarrow \phi_j)$, for $1\leq j < i$, followed by some new ones which will depend on the form of the expression $\phi_i$. First, if $\phi_i \in \Lambda \cup \Delta$, then we add the expresions $\phi_i$, $(\phi_i \Rightarrow (\phi \Rightarrow \phi_i))$, and $(\phi \Rightarrow \phi_i)$. The first expression is legal to add by the inductive hypothesis. The second is a logical axiom (it's Boolean form is a tautology), and the third is legal by modus ponens from the previous two expressions. This clearly constructs a proof. Next, suppose that $\phi_i$ is obtained through modus ponens, i.e. $\phi_j$ and $(\phi_j \Rightarrow \phi_i)$ are somewhere else in the proof. Then by the inductive hypothesis, $(\phi \Rightarrow \phi_j)$ and $(\phi \Rightarrow (\phi_j \Rightarrow \phi_i))$ are already in the proof somewhere. We then add the expressions $((\phi \Rightarrow \phi_j) \Rightarrow ((\phi \Rightarrow (\phi_j \Rightarrow \phi_i)) \Rightarrow (\phi \Rightarrow \phi_i)))$ (the Boolean form of which is a tautology), $(\phi \Rightarrow (\phi_j \Rightarrow \phi_i) \Rightarrow (\phi \Rightarrow \phi_i))$ (legal by two uses of modus ponens - the first thing we added was the implication, and the antecedent of that is assumed to be in $S$ already), and $(\phi \Rightarrow \phi_i)$ (legal by modus ponens again, using the second thing we added and the antecedent already being in $S$). Finally, we have the case that $\phi_i = \phi$, but then we can just add $\phi = \phi$, a tautology. This completes the proof.
\end{proof}
Towards a metatheorem which validates our "proof by contradiction", we define a \textbf{contradiction} to be any expression of the form $\psi \wedge \neg \psi$. Suppose that $\Delta \vdash (\psi \wedge \neg \psi)$ for some $\psi$. If this happens, then $\Delta$ is worthless: \textit{every} expression $\phi$ is satisfied. To see this, suppose $\Delta \vdash (\psi \wedge \neg \psi)$ for some $\psi$. Then $((\psi \wedge \neg \psi) \Rightarrow \psi)$ is a tautology, so by modus ponens we have that $\Delta \vdash \psi$, and similarly we have $\Delta \vdash (\neg \psi)$. Let $\phi$ be an arbitrary expression. Then $\Delta \cup \{\phi\} \vdash \psi$, so by the deduction technique we have $\Delta \vdash (\phi \Rightarrow \psi)$. Next, $(\phi \Rightarrow \psi) \Rightarrow (\neg \psi \Rightarrow \neg \phi))$ is a tautology (every expression implies it's contrapositive), so by modus ponens this yields $(\neg \psi \Rightarrow \neg \phi)$ But $\neg \psi$ has a proof, so we can add the steps of that proof to get $\neg \psi$ and then get $\Delta \vdash (\neg \phi)$ by modus ponens. Now, for the same reason that $\Delta \vdash (\phi \Rightarrow \psi)$, we can also justify that $\Delta \vdash (\neg \phi \Rightarrow \psi)$, which by the same contrapositive argument above gives us $(\neg \psi \Rightarrow \neg \neg \phi)$, which by modus ponens gives $\neg \neg \phi$. But of course, $(\neg \neg \phi \Rightarrow \phi)$ is a tautology, so by modus ponens we get $\phi$, and thus we have $\Delta \vdash \phi$. Let us make a definition:
\begin{definition}
    If $\Delta$ is an axiom system such that there exists a $\psi$ with $\Delta \vdash (\psi \wedge \neg \psi)$, then we say that $\Delta$ is \textbf{inconsistent}. Else, we say that it is \textbf{consistent}.
\end{definition}
By our argument above, with an inconsistent set of axioms, every expression is satisfied, along with it's negation, and so nothing has any meaning. The notion of consistency will have some significance going forward, but for now, it allows us to state prove the following metatheorem which captures our notion of proof by contradiction:
\begin{theorem}[Proof by Contradiction]
    If $\Delta \cup \{\neg \phi\}$ is inconsistent, then $\Delta \vdash \phi$. (Note we are making no mention of $\Delta$'s consistency.)
\end{theorem}
\begin{proof}
    Suppose that $\Delta \cup \{\phi\}$ is inconsistent. Then, along with literally everything else $\Delta \cup \{\neg \phi\} \vdash \phi$ (see above). Thus, by the deduction technique, we know that $\Delta \vdash (\neg \phi \Rightarrow \phi)$, which is equivalent to $\phi$. (Formally, $((\neg \phi \Rightarrow \phi) \Rightarrow \phi)$ is a Boolean tautology, so $\phi$ would follow in a proof by modus ponens.) Thus $\Delta \vdash \phi$.
\end{proof}
Finally, we turn to quantifiers. In mathematics we often hold some object as arbitrary, and show that it has a certain property, and then conclude that since no assumptions were made about the object, every object in a larger set has that property. We give this idea a name, and prove it's corresponding metatheorem:
\begin{theorem}[Justified Generalization]
    Suppose that $\Delta \vdash \phi$, and $x$ is a variable which is not free in any expression of $\Delta$ (i.e. it's unused with respect to any quantifiers that our axioms might be using). Then $\Delta \vdash (\forall x \phi)$. 
\end{theorem}
\begin{proof}
    Consider a proof $S = (\phi_1,...,\phi_n)$ of $\phi$ from $\Delta$, i.e. $\phi_n = \phi$. We proceed just as we did in the deduction metatheorem, by showing via induction that for each $i = 0,...,n$, there is a proof of $\forall x \phi_i$. For the case $i=0$, the statement is vacuously true. (?) Assume it is true for all $j<i$, where $i \leq n$. The proof of $\forall x \phi_i$ will include all of the previous proofs of $\forall x \phi_j$, and some new expressions depending on the form of $\phi_i$. First, if $\phi_i \in \Delta$, then by hypothesis $x$ is not free in $\phi_i$, so we add the logical axiom $\phi_i \Rightarrow \forall x \phi_i$, and then add $\forall x \phi_i$ by modus ponens. (By the structure of this induction $\phi_i$ will already be in $S$, but for the sake of legality we can take the partial proof to be there if it's not already.) If $\phi_i$ was obtained by some $\phi_j$ and $(\phi_j \Rightarrow \phi_i)$ by modus ponens, then by the inductive hypothesis we have a proof of $\forall x \phi_j$ as well as $\forall x (\phi_j \Rightarrow \phi_i)$. To $S$ then, we add $(\forall x (\phi_j \Rightarrow \phi_i) \Rightarrow ((\forall x \phi_j) \Rightarrow (\forall x \phi_i)))$ (the logical axiom corresponding to the distributive property of universal quantifiers),  along with $((\forall x \phi_j) \Rightarrow (\forall x \phi_i))$ (now legal through modus ponens), and finally $\forall x \phi_i$ (again now legal through modus ponens). This completes the proof.
\end{proof}
These three metatheorems equip us to be able to rigorously argue facts about proofs in our metatheory via proof techniques that we are already familiar with. They also equip us with almost everything that we need in order to prove the completeness theorem. Some cleanup is required before we do so.

% I am not sure how important any of this is %
\par First, the justified generalization metatheorem has a technical limitation regarding free variables that hampers it's usefulness. We address this now as a corollary, in a way that should seem obvious, but unfortunately requires a fairly technical lemma:
\begin{lemma}
    Let $\phi$ be an expression, and $z$ a variable not appearing in $\phi$. Then any variable $x$ is substitutible for $z$ in $\phi_z^x$ (regardless of whether or not this substitution is itself legal). Furthermore, $(\phi_z^x)_x^z = \phi$.
\end{lemma}
\begin{proof}
    $\phi_z^x$ is the expression obtained by replacing every free occurrence of the variable $x$ by the variable $z$. Everything is trivial if $x$ doesn't appear in $\phi$ in the first place, so we will assume it is, in which case the hypothesis requires that $z \neq x$. We lazily induct on the form of $\phi$. If $\phi$ is atomic, both claims are trivial. Similarly we have trivial inductive justifications in the cases that $\phi$ is a negation or $\phi$ is a conjunction/disjunction. Assume then that $\phi = \forall y \psi$. Now, if $y = x$, then $\phi_z^x = \forall z \psi_z^x$. Now $x \neq z$, so by definition $x$ is substitutible for $z$ here iff $x$ is substitutible for $x$ in $\psi_z^x$. But this is true by the inductive hypothesis. (The second condition, that $x$ isn't to appear in $z$, doesn't apply because $z$ is a variable.) Finally, if $y \neq x$, then $z$ is substitutible for $x$ in $\phi$. 
\end{proof}
\begin{corollary}
    If $\Delta \vdash \phi_z^x$, where $z$ is not free in $\Delta$, and where $z$ does not occur in $\phi$, then $\Delta \vdash \forall x \phi$.
\end{corollary}
\begin{proof}
    Since $\Delta \vdash \phi_z^x$ and $z$ is not free in $\Delta$, by the justified generalization metatheorem we have $\Delta \vdash \forall z \phi_z^x$. But now, note that $x$ is substitutable for $z$ in $\phi_z^x$. Then, by the lemma, since $z$ never appeared in $\phi$, we have that $x$ is substitutible for $z$ in $\phi_z^x$. Thus from fact 2.6 we have that $\forall z \phi_z^x \Rightarrow (\phi_z^x)_x^z$, but this is just $\phi$. Note that $x$ doesn't appear in $\phi_z^x$, so it certainly isn't free. Thus the \textit{single axiom set} $\{ \forall z \phi_z^x \} \vdash \phi$, and so by the generalization metatheorem $\{ \forall z \phi_z^x \} \vdash \forall x \phi$. But then by our first observation, this gives $\Delta \vdash \forall x \phi$. 
\end{proof}
All this is to say, as long as there is \textit{some} variable for use which is not free in $\Delta$, one can just substitute that wherever it needs to go, and then use the generalization metatheorem safely. 
\begin{lemma}[We can replace letters with other letters]
    Let $\phi$ and $\psi$ be expressions that are identical, except that $\phi$ has free occurrences of a variable $x$ wherever $\psi$ has free occurrences of some other variable $y$. I.e. $\phi_y^x = \psi$ and vice versa. Then $\vdash ((\forall x \phi) \Rightarrow (\forall y \psi))$. Stated simply, if an expression is valid, then so are all of it's alphabetic variants.
\end{lemma}
\begin{proof}
    If $x =y$, then the trivial single expression proof $S = \{\forall x \phi \Rightarrow \forall x \phi\}$ will work, so assume $x \neq y$. By the deduction metatheorem it suffices to show that $\{\forall x \phi \} \vdash \forall y \psi$. (Effectively we are taking $\Delta = \varnothing$.) To start, let $z$ be a new variable which is not free in $\forall x \phi$, and which does not occur in $\psi$. Let the first expression in our proof be $\phi_1 = \forall x \phi$. Next, we add the logical axiom $\phi_2 = \forall x \phi \Rightarrow \phi_z^x$, and note that this consequent is also equal to $\psi_z^y$. Then by modus ponens, we add $\phi_3 = \psi_z^y$. This is a proof that $\{\forall x \phi \} \vdash \psi_z^y$. Then by the above corollary extending the generalization metatheorem, this means that $\{\forall x \phi \} \vdash \forall y \psi$, completing the proof.
\end{proof}
Finally, we need an extremely technical lemma, that anyone reading this is encouraged to read but skip over the proof of:
\begin{fact}
    If $\Delta \vdash \phi$ and $c$ is a constant symbol not appearing in $\Delta$, then there is a variable $y$ not appearing in $\phi$ such that $\Delta \vdash \forall y \phi_y^c$, and furthermore there is a deduction of $\forall y \phi_y^c$ from $\Delta$ in which $c$ does not appear.  
\end{fact}
\begin{proof}
    Let $S = (\phi_1,...,\phi_n)$ be a proof of $\phi$, i.e. $\phi = \phi_n$, and $y$ be a variable not appearing in any of the $\phi_i$. (This is no problem, since our universal set of variables $V$ is countably infinite, and this is a finite set of finite expressions.) We show by induction on $m$ that $((\phi_0)_y^c,...,(\phi_n)_y^c)$ is a proof of $(\phi_m)_y^c$, for all $m \leq n$.
    \par If $\phi_m \in \Delta$, then $c$ does not appear in $\phi_m$, so $(\phi_m)_y^c = \phi_m \in \Delta$. If $\phi_m$ is deduced by modus ponens from some $\phi_i$ and $\phi_j = (\phi_i \Rightarrow \phi_m)$, then $(\phi_j)_y^c$ and $(\phi_i)_y^c \Rightarrow (\phi_m)_y^c$ imply $(\phi_m)_y^c$ by modus ponens. The final case is that $\phi_m \in \Lambda$, in which we do not necessarily have anymore that $c$ doesn't appear anywhere in $\Lambda$. The logical axioms for which this could be an issue are mainly the ones involving quantifiers, so we begin there. Suppose that $\phi_m = \forall x \psi \Rightarrow \psi_t^c$ for some expression $\psi$, and some term $t$ which is substitutible for $x$ in $\psi$. Then $(\phi_m)_y^c = \forall x \psi_y^c \Rightarrow (\psi_t^x)_y^c$. It is straightforward to see that 
    \[ (\psi_t^x)_y^c = (\psi_y^c)_{t_y^c}^x \]
    Meaning that
    \[ (\phi_m)_y^c = \forall x \psi_y^c \Rightarrow (\psi_y^c)_{t_y^c}^x \]
    Now, $x$ occurs free in $\psi_y^c$ wherever it appears free in $\psi$. Also, $t_y^c$ is substitutible for $x$ in $\psi_y^c$, by virtue of $y$ not appearing in $\psi$, and thus certainly not appearing as a quantified variable in $\psi_y^c$. Thus, y is substitutible for $c$ in the entirety of the expression $\phi_m$, giving that $(\phi_m)_y^c \in \Lambda$. The arguments for the other logical axioms involving quantifiers are similar. The arguments for general expressions whose Boolean forms are tautologies, as well as those involving properties of equality, are trivial by the inductive hypothesis and the quantifier cases already shown. This completes the induction on $\phi_m$ and subsequently the proof.
\end{proof}
We are finally ready to state and prove the completeness theorem. We will state it in two equivalent ways:
\begin{theorem}[Godel's Completeness Theorem]
    If $\Delta \models \phi$, then $\Delta \vdash \phi$.
\end{theorem}
\begin{theorem}[Godel's Completeness Theorem, Second Form]
    If $\Delta$ is a consistent set of axioms, then it has a model.
\end{theorem}
Let us see first how these two statements are equivalent. Suppose the second form of the theorem is true, and let $\phi$ be an expression such that $\Delta \models \phi$. Then any model that satisfies all of the expressions in $\Delta$ must also satisfy $\phi$, and hence fail to satisfy $\neg \phi$. Thus, no model, whether one exists or not, can ever satisfy $\Delta \cup \{\neg \phi\}$. Then by the controposative of this second statement, we have that $\Delta$ must be inconsistent. But by the contradiction metatheorem, it must be that $\Delta \vdash \phi$. Conversely, suppose that the first version is true, and suppose that $\Delta \nvdash \phi$. (Need to finish this direction)
\begin{proof}
    We will prove the second version of the theorem. Let $\Delta$ be a consistent set of axioms over a vocabulary $\Sigma$. We need somehow construct a model $M = (U,\mu)$ for $\Delta$. For starters, our universe $U$ will included the collection of \textit{all terms} over the vocabulary $\Sigma$. However, this alone may not be big enough. For instance, if $P$ is a unary relation symbol in $\Sigma$, and $\Delta = \{\exists x P(x)\} \cup \{\neg P(t): \textrm{ t is a term over $\Sigma$}\}$, then $\Delta$ is consistent (nts), but no model whose universe is \textit{only} the terms over $\Sigma$ will be able to satisfy it: To satisfy $\Delta$, we would need to construct a relation $P$ such that $P(t)$ fails on all terms, and yet still have a term $t$ such that $P(t)$ holds, a clear double standard. To deal with this and other possible difficulties, we will do something a little weird. We will \textit{add to $\Sigma$} a countable collection of constant symbols $c_1,c_2,...$ which weren't previously present there, to make a new vocabulary $\Sigma'$. To show that this is okay, we now show that if $\Delta$ is consistent, then it remains consistent as a set of axioms over the new extended vocabulary $\Sigma'$.
    \par By way of contradiction, suppose that $\Delta$ is not consistent as a set of axioms over $\Sigma'$. Then $\Delta \vdash \phi_n$ for some $\phi_n$ of the form $\phi_n = \psi \wedge \neg \psi$, and let $S = (\phi_1,...,\phi_n)$ be a proof of it. Without loss of generality we can assume that there are an infinite number of variables which do not appear anywhere in $\Delta$. (If all of them are used, and we index them one at a time, then we can simply reindex them using only the odd indices. Remember that the set of variables is always assumed to be the same countably infinite set, $V$.) Call these unused variables $x_1,x_2,...$. Then for every instance of a new constant symbol $c_i$, we invoke the extremely technical lemma above to substitute the variable $x_i$. It is clear then by the lemma that this new proof $S'$ is the deduction of a contradiction in the original vocabulary, contradicting our assumption that $\Delta$ was consistent there.
    \par Now, armed with our unused constants, we gradually add new expressions $\phi_i$ into our original set of axioms $\Delta$. We do so inductively, in stages, checking at each stage to make sure that our set is still consistent. First, we enumerate the set of expressions over $\Sigma'$ $\phi_1,\phi_2,...$. Assume $\Delta_0 = \Delta$, and assume inductively that we are at stage $i$ of the construction. We add in expressions one at a time, as follows. 
    \begin{itemize}
        \item If $\Delta_{i-1} \cup \{\phi_i\}$ is consistent, and $\phi_i$ is not of the form $\exists x \psi$ for some $\psi$, then we let $\Delta_i = \Delta_{i-1} \cup \{\phi_i\}$.
        \item If $\Delta_{i-1} \cup \{\phi_i\}$ is consistent, and we don't have the case above, then $\phi_i = \exists x \psi$. For this, we choose one of the constants $c$ which has yet to appear in any of the $\phi_j$ so far for $j<i$, and then let $\Delta_i = \Delta_{i-1} \cup \{\exists x \psi, \psi_c^x\}$. (Equivalently, we could add the single expression $\exists x \psi \Rightarrow \psi_c^x$. These expressions are called in literature \textit{Henkin witnesses}.)
        \item If $\Delta_{i-1} \cup \{\psi_i\}$ is inconsistent, and $\psi_i$ is not of the form $\forall x \psi$, then we let $\Delta_i = \Delta_{i-1} \cup \{\neg \phi_i\}$
        \item If $\Delta_{i-1} \cup \{\psi_i\}$ is inconsistent, and $\psi_i$ \textit{is} of the form above, then we again choose one of the constants $c$ which has yet to appear in the construction, and let $\Delta_i = \Delta_{i-1} \cup \{\neg \forall x \psi, \neg \psi_c^x\}$
    \end{itemize}
    A couple things are of note. Firstly, this construction, as well as many of the arguments made up to this proof, weakly depend on the assumption that our vocabulary $\Sigma$ consists of at most a \textit{countably infinite} set of symbols. In practice, vocabularies in mathematics are always like this, and it is somewhat ridiculous to assume otherwise, especially in the context of computability theory. However, it is worth noting that the completeness theorem remains basically true in the case when $\Sigma$ is uncountable. The reason we bring this up now is because if it were uncountable, we wouldn't have been able to enumerate the set of expressions in $\Sigma'$. To circumvent this, we would have had to add in all of the Henken witnesses in bulk, and then use \textit{Zorn's Lemma} to expand that new set into a maximally consistent set. Our set is maximally consistent trivially already. Thus, the completeness theorem, with a slight alteration of this exact proof, remains true even if the vocabulary is uncountable, \textit{conditionally} on the requirement that we need to invoke the axiom of choice (which is equivalent to Zorn's Lemma). That is the gist of what we are doing though: We are dumping as many expressions as we possibly can into $\Delta$, while (hopefully) keeping the set consistent. We confirm that next.
    \begin{fact}
        For all $i \geq 0$, $\Delta_i$ is consistent.
    \end{fact}
    \begin{proof}
        We go by induction on $i$. The base case is done already. Suppose that $\Delta_{i-1}$ is consistent. If $\Delta_i$ is determined by the first case, then yeah duh $\Delta_i$ is consistent. Similarly yeah duh for the third case, except we are only allowed to say yeah duh by the contradiction metatheorem. (The metatheorem says that because $\Delta_{i-1} \cup \{\phi_i\}$ is inconsistent, $\Delta_{i-1} \vdash \neg \phi_i \Rightarrow \Delta_{i-1} \models \neg \phi_i$ by the soundness theorem, so it makes no difference whether or not we add this in.)
        \par Now we reach the nontrivial cases. Assume we are dealing with the second case, i.e. $\Delta_{i-1} \cup \{\phi_i\}$ is incosistent and $\phi_i$ is of the form $\exists x \psi$. Suppose that $\Delta_{i-1} \cup \{\exists x \psi \Rightarrow \psi_c^x\}$ is inconsistent (remember this is equivalent to what we were adding in case 2.) We will show that this implies $\Delta_{i-1}$ is itself inconsistent, a contradiction of the inductive hypothesis. Note that for any expressions $\alpha$ and $\beta$, it is always the case that $\{\neg \alpha\} \vdash (\alpha \Rightarrow \beta)$. (This is easily shown via our deduction metatheorem.) Consequently it is just as easy to show that $\{\neg(\alpha \Rightarrow \beta)\} \vdash \alpha$. Now by the hypothesis and the contradiction metatheorem we have to have that $\Delta_{i-1} \vdash \neg (\exists x \psi \Rightarrow \psi_c^x)$, and therefore by the simple observation we just made, we know that $\Delta_{i-1} \vdash \exists x \psi$. Similarly, since $\{\beta\} \vdash (\alpha \Rightarrow \beta)$ by a logical axiom, we have by a contrapositive type argument using our deduction metatheorem that $\{\neg(\alpha \Rightarrow \beta)\} \vdash \neg \beta$, so an extension of our last observation gives $\Delta_{i-1} \vdash \neg \psi_c^x$. Now, by hypothesis, this particular $c$ doesn't appear in any of the expressions in $\Delta_{i-1}$, so we can employ our extremely technical lemma to assume that $\Delta_{i-1} \vdash \forall z \neg (\psi_c^x)_z^c$ for some variable $z$ not already appearing in $\psi$, and furthermore since $c$ never occurs in $\psi$, we have that $(\psi_c^x)_z^c = \psi_z^x$. Thus $\Delta_{i-1} \vdash \forall z\neg \psi_z^x$. Since $z$ did not occur in $\psi$, $z$ occurs free in $\psi_z^x$ exactly where $x$ occurred free in $\psi$, and so by our baby lemma about alphabetic variants, $x$ is substitutible for $z$ in $\psi_z^x$. Thus have that $\forall z \neg \psi_z^x \Rightarrow \neg (\psi_z^x)_x^z = \psi$, and thus $\forall z \neg \psi_z^x\ \Rightarrow \neg \psi$. Since $x$ is not free in $\forall z \neg \psi_z^x$ (it isn't even there! I hate this shit so much) we have $\{\forall z \neg \psi_z^x\} \vdash \neg \psi$. Thus $\Delta_{i-1} \vdash \forall x \neg \psi$, but this is a problem, because earlier we said that $\Delta_{i-1} \vdash \exists x \psi$, which is equivalent of course to $\forall x \neg \psi$. We have arrived at our contradiction. The final case is similar. 
    \end{proof}
    Now, finally, let $\Delta' = \bigcup_{i=0}^{\infty}\Delta_i$. First, we confirm that $\Delta'$ is consistent. Thankfully, this is much easier: If we were to assume that it wasn't, and if $S$ were a proof of some contradiction in $\Delta'$, then since proofs are finite, and every expression in $S$ is in some $\Delta_i$, this $S$ would be a proof of a contradiction at the largest of these levels, call it $j = \max{i: \textrm{ i indexes one of the expressions of $S$}}$ meaning that $\Delta_j$ is itself inconsistent, a contradiction. $\Delta'$ isn't just consistent though - it is \textit{complete}. By construction, for any expression $\phi$, we have that either $\phi \in \Delta'$ or $\neg \phi \in \Delta'$. Finally, we have that $\Delta'$ is \textit{closed}. That is to say, for any expression of the form $\exists x \phi$, we are guarunteed to also have an expression of the form $\phi_c^x$ for some constant $c$. This is due to the so called Henken witnesses.
    \par Now, we begin to actually construct the model $M$. Let $T$ be the set of all of the terms in $\Sigma'$. We define an equivalence relation $\equiv$ on $T$, by $t \equiv t'$ iff the expression $t = t' \in \Delta'$. (i.e. the atomic expression $=(t,t') \in \Delta'$. It might be worth a reminder that because $\Delta'$ is complete, all of these expressions are either in $\Delta'$, or their negations are.
    \begin{fact}
        $\equiv$ is an equivalence relation on $T$
    \end{fact}
    \begin{proof}
        First, reflexivity itself of the equality symbol is a logical axiom, so $\equiv$ inherits the reflexivity of that: $\vdash (t = t) \Rightarrow (t = t) \vdash \Delta' \Rightarrow t \equiv t$ for any term $t$.
        For symmetry, note that for any model $M$ satisfying $\Delta'$, $M \models \Delta' \Rightarrow M \models (t = u) \Rightarrow t^M = u^M \Rightarrow u^M = t^M \Rightarrow M \models (u = t)$, and since $M$ models $\Delta'$ and every expression is either in $\Delta'$ or it's negation is and $\Delta'$ is consistent, we must conclude that $(u = t) \in \Delta'$, so $u \equiv t$, i.e. $\equiv$ is reflexive.
        \par Finally, for transitivity, suppose that $t_1 \equiv t_2$ and $t_2 \equiv t_3$. We can apply logical axiom 2c to get this, as this allows us to use $(t_2 = t_3 \wedge t_1 = t_1) \Rightarrow (=(t_1,t_2) \Rightarrow =(t_1,t_3)$, and then by modus ponens twice we get that $\Delta' \vdash (t_1 = t_3)$, meaning that $(t_1 = t_3) \in \Delta'$. Thus $\equiv$ is reflexive, symmetric, and transitive, so it is an equivalence relation. (Note that more generally this shows that regarding $=$ as an equivalence relation is something that can always be taken as a first order theorem.)
    \end{proof}
    Now define the universe of our model $U$ to be the \textit{set of all equivalence classes for $\equiv$}. So $[t] \in U$ means that $[t] = \{t': t' \equiv t\}$. Next, we begin to define our function $\mu$. To the variables $x$, we let $x^M = [x]$. To the constants $c$, we let $c^M = [c]$. If $R$ is a $k$-ary relation symbol, then we let $R^M([t_1],...,[t_k])$ iff $R(t_1,...,t_k) \in \Delta'$, and if $f$ is a $k$-ary function symbol, then we let $f^M([t_1],...,[t_k]) = [f(t_1,...,t_k)]$. Next of course, we must show that these definitions are well founded. To see this, suppose that $t_1 \equiv u_1, t_2 \equiv u_2,..., t_k \equiv u_k$. We need to show that $f(t_1,..,t_k) \equiv f(u_1,...,u_k)$, i.e. the output of $f^M$ is independent of our choice of representatives for the equivalence relation. Note that $t_i \equiv u_i$ for all $i$ implies that $(t_i = u_i) \in \Delta'$ for all $i$, and hence by one of our logical axioms 2a we have that $\Delta' \vdash (f(t_1,...,t_k) = f(u_1,...,u_k))$. Since $\Delta'$ is both complete and consistent, we must then have that this expression is itself an element of $\Delta'$, and therefore $f^M([t_1],...,[t_k]) = [f(t_1,...,t_k)]$. The well foundedness of relations is an identical argument, making use of logical axiom 2c instead of 2b. Thus, finally, we have a well defined structure $M$. It remains to show that this structure truly models $\Delta'$. Note that once we have shown this, then we are done, because satisfying every expression in $\Delta'$ means we satisfy every expression in $\Delta$, by virtue of $\Delta \subseteq \Delta'$.
    \par We go by induction on the structure of $\phi$ to show that $M \models \phi \iff \Delta' \models \phi$. Firstly, if $\phi$ is atomic, that is, $\phi = R(t_1,...,t_k)$. Then by the definition of $R^M$, it is clearly the case that $R^M(t_1,...,t_k) \iff \phi \in \Delta'$. Next we address the Boolean connective cases. 
    If $\phi = \neg \psi$ for some $\psi$, then $M \models \phi \iff M \nvDash \psi$. By the inductive hypothesis, $M \nvDash \psi \iff \Delta' \nvDash \psi$. The case for $\vee$ and $\wedge$ are equally immediate. Finally, for the quantifier case $\phi = \forall x \psi$. First we show that if $\phi \in \Delta'$, then $M \models \phi$. 
    That is to say, for any term $[t]$, we have that $M_{x = [t]} \models \psi$. Let $\psi'$ be an alphabetic variant of $\psi$, such that $t$ is substitutible for $x$ in $\psi'$. (I.e. just replace any problematic variables with ones that aren't, and note that this alphabetic variant is equivalent to the original.) 
    Then $M_{x = t} \models \psi \iff M_{x = t} \models \psi_t^{'x} \iff \psi_t^{'x} \in \Delta'$. (Where the last $\iff$ is by the inductive hypothesis.) 
    But since $\forall x \psi \in \Delta'$, it's alphabetic variant $\forall x \psi' \in \Delta'$. But since $t$ is substitutible for $x$ in $\psi'$, we can invoke our subsitution logical axiom to obtain $\forall x \psi' \Rightarrow \psi_t^{'x}$, and so $\psi_t^{'x} \in \Delta'$. Thus $M \models \forall x \psi$. 
    Conversely, suppose that $M \models \forall x \psi$. We seek to show that $\forall x \psi \in \Delta'$. Suppose it isn't. Then by completeness of $\Delta'$, it must be that $\neg \forall x \psi \in \Delta'$, which is of course equivalent to $\exists x \neg \psi$. But then this means that at some stage of the construction of $\Delta'$, we added the Henkin formula $\exists x \psi \Rightarrow \psi_c^x$. Thus by modus ponens we have $\Delta' \vdash \neg \psi_c^x$, where $c$ is a new constant symbol which doesn't appear in $\psi$. Since $M \models \forall x \psi$, we have that for all $[c] \in U$, $M_{x = [c]} \models \psi_c^x$. Thus $\psi_c^x \in \Delta'$. By the inductive hypothesis though this means that $\psi_c^x \in \Delta'$, a contradiction.  
    \par Thus, we've shown that $M$ models our expanded, maximal set of axioms $\Delta'$, and consequently also satisfies $\Delta$. The proof is complete.
\end{proof}
The completeness theorem has a wealth of extremely important corollaries. The most important one for us is this:
\begin{corollary}
    For any set of axioms $\Delta$ over any vocabulary, $VALIDITY_{\Delta} \in \textbf{RE}$ 
\end{corollary}
\begin{proof}
    We now know that $VALIDITY_{\Delta} = SATISFIABILITY_{\Delta}$, which we already knew was in $\textbf{RE}$!
\end{proof}
This is good news, but don't think for a second that it's the full story. We now know that we can have a computer generate all statements that follow from a set of axioms. What we still \textit{don't} know is whether or not every model is fully axiomatizable. To this end, the discussion turns back towards computability theory, as we make our way towards some truly amazing results. 
